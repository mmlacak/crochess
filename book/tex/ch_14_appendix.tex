
% Appendix chapter ----------------------------------------------------
\chapter*{Appendix}
\addcontentsline{toc}{chapter}{Appendix}
\label{ch:Appendix}

\section*{Odd variants}
\addcontentsline{toc}{section}{Odd variants}
\label{sec:Appendix/Odd variants}

Odd variants ...

Note that in Odd Classical Game, since it's played on 7 x 7 board,
there is no en-passant move. This is so because of very small board
there is no room for a Pawn to perform 2-field initial move without,
at the same time, preventing opponent to do the same at the same file.

\section*{Sides of chess board}
\addcontentsline{toc}{section}{Sides of chess board}
\label{sec:Appendix/Sides of chess board}

Sides of chess board ... light, dark, left, right.

\section*{Initial setups}
\addcontentsline{toc}{section}{Initial setups}
\label{sec:Appendix/Initial setups}
Initial setups ...

% Initial setup for Light player is (mirrored for Dark one):
% \texttt{PPPPPPPPPP \\
%         RGNBQKBNGR}, \\
% or more conveniently, as seen in this image:

% Initial setup for Light player is (mirrored for Dark one):
% \texttt{PPPPPPPPPPPP \\
%         RGANBQKBNAGR}, \\
% or more conveniently, as seen in this image:

% Initial setup for Light player is (mirrored for Dark one):
% \texttt{PPPPPPPPPPPPPP \\
%         RGAUNBQKBNUAGR}, \\
% or more conveniently, as seen in this image:

% Initial setup for Light player is (mirrored for Dark one):
% \texttt{PPPPPPPPPPPPPPPP \\
%         RGAUWNBQKBNWUAGR}, \\
% or more conveniently, as seen in this image:

% Initial setup for Light player is (mirrored for Dark one):
% \texttt{PPPPPPPPPPPPPPPPPP \\
%         TRGAUWNBQKBNWUAGRT}, \\
% or more conveniently, as seen in this image:

% Initial setup for Light player is (mirrored for Dark one):
% \texttt{PPPPPPPPPPPPPPPPPPPP \\
%         TRGAUWCNBQKBNCWUAGRT}, \\
% or more conveniently, as seen in this image:

% Initial setup for Light player is (mirrored for Dark one):
% \texttt{PPPPPPPPPPPPPPPPPPPPPP \\
%         TRGAUWCSNBQKBNSCWUAGRT}, \\
% or more conveniently, as seen in this image:

% Initial setup for Light player is (mirrored for Dark one):
% \texttt{PPPPPPPPPPPPPPPPPPPPPPPP \\
%         TRGAHUWCSNBQKBNSCWUHAGRT}, \\
% or more conveniently, as seen in this image:

% Initial setup for Light player is (mirrored for Dark one):
% \texttt{PPPPPPPPPPPPPPPPPPPPPPPP \\
%         TRGAHUWCSNBQKBNSCWUHAGRT}, \\
% or more conveniently, as seen in this image:

% Initial setup for Light player is (mirrored for Dark one):
% \texttt{PPPPPPPPPPPPPPPPPPPPPPPPPP \\
%         TRGAHIUWCSNBQKBNSCWUIHAGRT}, \\
% or more conveniently, as seen in this image:

\section*{Passive pieces}
\addcontentsline{toc}{section}{Passive pieces}
\label{sec:Appendix/Passive pieces}

Passive pieces are ...

\subsection*{Activating passive piece with Pawn}
\addcontentsline{toc}{subsection}{Activating passive piece with Pawn}
\label{subsec:Appendix/Passive pieces/Activating passive piece with Pawn}

Activating passive piece with Pawn ... capture-fields vs. step-fields.

\section*{Converting opponent's Rook}
\addcontentsline{toc}{section}{Converting opponent's Rook}
\label{sec:Appendix/Converting opponent's Rook}

Converting opponent's Rook in own Rook initial position ... regarding castling.

\section*{Converting opponent's Pawn}
\addcontentsline{toc}{section}{Converting opponent's Pawn}
\label{sec:Appendix/Converting opponent's Pawn}

Converting opponent's Pawn in own Pawn initial position ... regarding en-passant, promotion.

\section*{Activating Pawn, en passant}
\addcontentsline{toc}{section}{Activating Pawn, en passant}
\label{sec:Appendix/Activating Pawn, en passant}

Activating Pawn in its' initial position ... en passant.

\section*{Movement of Wave}
\addcontentsline{toc}{section}{Movement of Wave}
\label{sec:Appendix/Movement of Wave}

Movement of Wave ... as multi-step piece activating it  ...

\section*{Monolith initial positions}
\addcontentsline{toc}{section}{Monolith initial positions}
\label{sec:Appendix/Monolith initial positions}

Monolith initial positions ...

\section*{Royal powers}
\addcontentsline{toc}{section}{Royal powers}
\label{sec:Appendix/Royal powers}

Royal powers ...

    Refers to unique attributes of the king: being subject to check and checkmate, the inability to be captured, and the ability to castle.
    Protection from effects/actions induced/performed by non-active pieces, i.e. Pyramid, Wave, Star, Monolith.

% https://en.wikipedia.org/wiki/Glossary_of_chess#Royal_powers

\section*{Promote-to figures}
\addcontentsline{toc}{section}{Promote-to figures}
\label{sec:Appendix/Promote-to figures}

Pawn can be promoted to any piece except Pawn, King, Star or Monolith.

\clearpage % ..........................................................
% ---------------------------------------------------- Appendix chapter
