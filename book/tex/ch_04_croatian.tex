
% Copyright (c) 2015 - 2020 Mario Mlačak, mmlacak@gmail.com
% Public Domain work, under CC0 1.0 Universal Public Domain Dedication. See LICENSING, COPYING files for details.

% Croatian Ties chapter ===============================================
\chapter*{Croatian Ties}
\addcontentsline{toc}{chapter}{Croatian Ties}
\label{ch:Croatian Ties}

\begin{flushright}
\parbox{0.7\textwidth}{
\emph{Secrecy is the first essential in affairs of the State.\newline
\hspace*{\fill}{\textasciitilde{} De Richelieu} } }
\end{flushright}

\noindent
Croatian Ties is chess variant which is played on 10 $\times$ 10 board,
with light grey and red fields and dark gray and dark red pieces.
A new piece is introduced, Pegasus.

\clearpage % ..........................................................
% Pegasus *************************************************************

\section*{Pegasus}
\addcontentsline{toc}{section}{Pegasus}
\label{sec:Croatian Ties/Pegasus}

\noindent
\begin{wrapfigure}[6]{l}{0.4\textwidth}
\centering
\includegraphics[width=0.4\textwidth, keepaspectratio=true]{pieces/07_pegasus.png}
\caption{Pegasus}
\label{fig:07_pegasus}
\end{wrapfigure}
Pegasus moves similarly to Knight, but it can continue its jumpy movement
until another piece is encountered, or it runs out of board. Note that once
in movement, Pegasus cannot change its heading.

\vspace{5.0\baselineskip}
\subsection*{Movement}
\addcontentsline{toc}{subsection}{Movement}
\label{sec:Croatian Ties/Pegasus/Movement}

\noindent
\begin{wrapfigure}[15]{l}{0.5\textwidth}
\centering
\includegraphics[width=0.5\textwidth, keepaspectratio=true]{examples/04_ct/scn_ct_01_pegasus_initial.png}
\caption{Pegasus initial step}
\label{fig:scn_ct_01_pegasus_initial}
\end{wrapfigure}
In the example on the left we have Pegasus with all valid initial moves marked.
These all are the same as valid moves for Knight.

Pegasus' movement is not hampered by a piece placed on any unmarked field.
Pegasus can "jump" over it just as Knight would.

\clearpage % ..........................................................

\noindent
\begin{figure}[!h]
\includegraphics[width=1.0\textwidth, keepaspectratio=true]{examples/04_ct/scn_ct_02_pegasus_direction.png}
\caption{Pegasus move direction}
\label{fig:scn_ct_02_pegasus_direction}
\end{figure}

Once direction is chosen Pegasus can continue its movement performing one jump
after another in order from nearest field to furthest. Here, this is marked
with green arrows. Accessible fields are marked 1 to 4, in order of accessibility,
from nearest to furthest. Again, once direction is chosen it can't be changed anymore.
For instance, after reaching field 2 it's illegal to change direction to 2f (or
any other red arrow).

\clearpage % ..........................................................

\subsection*{Steps, step-fields, capture-fields, ply}
\addcontentsline{toc}{subsection}{Steps, step-fields, capture-fields, ply}
\label{sec:Croatian Ties/Pegasus/Steps, step-fields, capture-fields, ply}

\noindent
\begin{figure}[!h]
\vspace{-1.0\baselineskip}
\includegraphics[width=1.0\textwidth, keepaspectratio=true]{examples/04_ct/scn_ct_03_define_step_ply.png}
\caption{Step-fields, capture-fields, ply}
\label{fig:scn_ct_03_define_step_ply}
\end{figure}

Above, field 3 is chosen as destination for Pegasus' movement. Move along arrow is a step.
Field at which arrow points to is a step-field. Here, each step-field is also capture-field,
Pegasus would be able to capture opponent's piece on it. Completed movement of Pegasus,
from its starting position to its destination field 3 is a ply.

\clearpage % ..........................................................

\subsection*{Movement (cont.)}
\addcontentsline{toc}{subsection}{Movement (cont.)}
\label{sec:Croatian Ties/Pegasus/Movement (cont.)}

\noindent
\begin{figure}[!h]
\vspace{-1.2\baselineskip}
\includegraphics[width=1.0\textwidth, keepaspectratio=true]{examples/04_ct/scn_ct_04_pegasus_movement.png}
\caption{Pegasus moves}
\label{fig:scn_ct_04_pegasus_movement}
\end{figure}

Pegasus can "jump" over pieces on non-step-fields, Rooks in example above. Numbers
here enumerate directions of movement. Own piece on step-field stops Pegasus at
preceding step-field, see direction 2. Opponent's piece on step-field can be captured
(blue arrow). Just as with any other piece that would finish the move, meaning Pegasus
would have to stop at captured field, see direction 1.

% ************************************************************* Pegasus
\clearpage % ..........................................................

\section*{Rush, en passant}
\addcontentsline{toc}{section}{Rush, en passant}
\label{sec:Croatian Ties/Rush, en passant}

\noindent
\begin{wrapfigure}{l}{0.4\textwidth}
\centering
\includegraphics[width=0.33\textwidth, keepaspectratio=true]{en_passants/04_croatian_ties_en_passant.png}
\caption{En passant}
\label{fig:04_croatian_ties_en_passant}
\end{wrapfigure}
Rush is Pawn's longer initial movement, i.e. from its starting position, for at least
2 fields forward.

Rush and en passant are identical to those in Classic Chess, only difference is that Pawn
can now move longer on initial turn, up to 3 fields in this instance.

In the example on the left, rush fields are numbered. Longer rush also opens more opportunity
for opponent to perform en passant or block it, entirely or partially. For discussion on the
topic see:
\href{https://en.wikipedia.org/wiki/En\_passant}{https://en.wikipedia.org/wiki/En\_passant}.

\clearpage % ..........................................................

\section*{Castling}
\addcontentsline{toc}{section}{Castling}
\label{sec:Croatian Ties/Castling}

\vspace*{-0.3\baselineskip}
Castling is the same as in Classical Chess, only difference is that King can move either 2 or 3
fields across. All other constraints from Classical Chess still applies, described in detail here:
\href{https://en.wikipedia.org/wiki/Castling}{https://en.wikipedia.org/wiki/Castling}.

\vspace*{-0.3\baselineskip}
\noindent
\begin{figure}[!h]
\includegraphics[width=1.0\textwidth, keepaspectratio=true]{castlings/04_ct/croatian_ties_castling.png}
\vspace*{-1.4\baselineskip}
\caption{Castling}
\label{fig:croatian_ties_castling}
\end{figure}

\vspace*{-0.3\baselineskip}
In example above, all valid King's castling moves are numbered. Regardless if castling is long or short,
Rook always ends up on the opposite side of King on the field immediately next to it, i.e. one field closer
to center.

\vspace*{-0.3\baselineskip}
\noindent
\begin{figure}[!h]
\includegraphics[width=1.0\textwidth, keepaspectratio=true]{castlings/04_ct/croatian_ties_castling_left_03.png}
\vspace*{-1.4\baselineskip}
\caption{Castling long left}
\label{fig:croatian_ties_castling_left_03}
\end{figure}

\vspace*{-0.7\baselineskip}
\noindent
\begin{figure}[!h]
\includegraphics[width=1.0\textwidth, keepaspectratio=true]{castlings/04_ct/croatian_ties_castling_right_02.png}
\vspace*{-1.4\baselineskip}
\caption{Castling short right}
\label{fig:croatian_ties_castling_right_02}
\end{figure}

\vspace*{-0.3\baselineskip}
In examples above initial King's position is marked with "K". In both cases, Rook ends up at the
inside field, immediately next to the King.

\clearpage % ..........................................................

\section*{Initial setup}
\addcontentsline{toc}{section}{Initial setup}
\label{sec:Croatian Ties/Initial setup}

Compared to initial setup of Classical Chess, Pegasus is inserted between Rook and Knight
symmetrically, on both sides of chessboard. This can be seen in the image below:

\noindent
\begin{figure}[h]
\includegraphics[width=1.0\textwidth, keepaspectratio=true]{boards/04_croatian_ties.png}
\caption{Croatian Ties board}
\label{fig:04_croatian_ties}
\end{figure}

\clearpage % ..........................................................
% =============================================== Croatian Ties chapter
