
% Tamoanchan Revisited chapter ----------------------------------------
\chapter*{Tamoanchan Revisited}
\addcontentsline{toc}{chapter}{Tamoanchan Revisited}

\begin{flushright}
\parbox{0.6\textwidth}{
\emph{I dream, therefore I exist. \\
\hspace*{\fill}{\textperiodcentered \textperiodcentered \textperiodcentered \hspace*{0.2em} August Strindberg} } }
\end{flushright}

\noindent
Tamoanchan Revisited is chess variant which is played on 22 x 22 board,
with bright cyan and blue fields and light green and dark blue pieces.
Star colors are bright red and yellow. In algebraic notation, columns
are enumerated from 'a' to 'v', and rows are enumerated from '1' to '22'.
A new piece is introduced, Serpent.

\textbf{\huge{TODO :: Star colors !!!}} % TODO :: FIX ME !!!

\clearpage % ..........................................................

\section*{Serpent}
\addcontentsline{toc}{section}{Serpent}

\noindent
\begin{wrapfigure}{l}{0.4\textwidth}
\centering
\includegraphics[width=0.4\textwidth, keepaspectratio=true]{pieces/13_serpent.png}
\caption{Serpent}
\label{fig:13_serpent}
\end{wrapfigure}

\clearpage % ..........................................................

\section*{Initial setup}
\addcontentsline{toc}{section}{Initial setup}

Initial setup can be seen in image below:

\noindent
% \begin{figure}[t]
\begin{figure}[h]
\includegraphics[width=1.0\textwidth, keepaspectratio=true]{boards/16_tamoanchan_revisited.png}
\caption{Tamoanchan Revisited board}
\label{fig:16_tamoanchan_revisited}
% \centering
\end{figure}

\clearpage % ..........................................................
% ---------------------------------------- Tamoanchan Revisited chapter
