
% Miranda's veil chapter ----------------------------------------------
\chapter*{Miranda's veil}
\addcontentsline{toc}{chapter}{Miranda's veil}

\begin{flushright}
\parbox{0.8\textwidth}{
\emph{Under all that we think, lives all we believe, like the ultimate veil of our spirits. \\
\hspace*{\fill}{\textperiodcentered \textperiodcentered \textperiodcentered \hspace*{0.2em} Antonio Machado} } }
\end{flushright}

\noindent
Miranda's veil is chess variant which is played on 16 x 16 board, with
white and dark violet fields and light magenta and indigo pieces. In
algebraic notation, columns are enumerated from 'a' to 'p', and rows
are enumerated from '1' to '16'. A new piece is introduced, Wave.

\clearpage % ..........................................................

\section*{Wave}
\addcontentsline{toc}{section}{Wave}

\noindent
\begin{wrapfigure}[12]{l}{0.4\textwidth}
\includegraphics[width=0.4\textwidth, keepaspectratio=true]{pieces/10_wave.png}
\caption{Wave}
\label{fig:10_wave}
% % \centering
\end{wrapfigure}
Wave is passive piece, it has to be activated before it can move.
Activation is done in the same way as with Pyramid. Own piece
has to capture field at which Wave is located before Wave can
move.

Unlike Pyramid, Wave can be activated even if activating piece
has no momentum, i.e. with all its' momentum spent for moving.
Wave is not limited by received momentum.

After activation Wave moves like piece that just activated it.
Additionally, Wave can move over multiple step-fields, in the same
fashion, even if activating piece can move for only one.

\textbf{\huge{TODO :: link to "activated by/moves as" table}} % TODO :: FIX ME !!!

Pawn can activate Wave not just at its' capture-fields but also on its'
step-fields. Wave is not blocked by any piece, it can continue its'
movement freely as if there are no pieces on board at all.

Wave cannot capture any piece. Consequenly, Wave cannot check opponent's
King, and hence can't participate in a checkmate.

Wave can activate any own piece, except King. Wave can also activate
opponent's Wave.

In algebraic notation symbol for Wave is 'W'.

\clearpage % ..........................................................

\subsection*{Activation}
\addcontentsline{toc}{subsection}{Activation}

\noindent
% \begin{figure}[t]
\begin{figure}[h]
\includegraphics[width=1.0\textwidth, keepaspectratio=true]{examples/10_move_wave_init.png}
\caption{Wave activation}
\label{fig:10_move_wave_init}
% \centering
\end{figure}

Above, Knight is about to activate Wave. After activation, Wave could
continue movement the same way Knight moves over multiple step-fields,
thus moving as if activated by Pegasus. Wave does not spend received
momentum while moving, and would transfer it entirely to any piece it
activates.

\clearpage % ..........................................................

\noindent
% \begin{figure}[t]
\begin{figure}[h]
\includegraphics[width=1.0\textwidth, keepaspectratio=true]{examples/10_move_wave_activated.png}
\caption{Wave activated}
\label{fig:10_move_wave_activated}
% \centering
\end{figure}

Once activated, Wave can move unhindered by surrounding pieces (green arrows),
even if they occupy step-fields between starting and destination field. Wave
can also activate pieces (red arrows) obscured by others, for instance light
Pyramid which would otherwise be out of reach for regular piece, e.g. Pegasus.
Note, Wave cannot activate Kings nor opponent's pieces (grey arrows), except
dark Wave.

\clearpage % ..........................................................

\noindent
% \begin{figure}[t]
\begin{figure}[h]
\includegraphics[width=1.0\textwidth, keepaspectratio=true]{examples/10_move_wave_finished.png}
\caption{Wave finished}
\label{fig:10_move_wave_finished}
% \centering
\end{figure}

Activated Bishop now continues the move according to own rules of movement,
i.e. diagonally. Note that it's restricted by momentum received, and thus
can move for only one field.

\clearpage % ..........................................................

\subsection*{Cascading Waves}
\addcontentsline{toc}{subsection}{Cascading Waves}

\noindent
\begin{wrapfigure}[13]{l}{0.5625\textwidth}
\includegraphics[width=0.5625\textwidth, keepaspectratio=true]{examples/10_move_wave_cascading_rook.png}
\caption{Rook starting cascade}
\label{fig:10_move_wave_cascading_rook}
% % \centering
\end{wrapfigure}
Cascading Waves refers to a move in which two or more Waves have been displaced.
For example, Wave can activate another Wave. Wave can also activate active
piece (or Pyramid), which can then activate another Wave.

On the left, Rook is about to activate Wave 1, giving it momentum of 4.

\vspace*{0.05\textheight}
\noindent
\begin{wrapfigure}[12]{r}{0.5625\textwidth}
\includegraphics[width=0.5625\textwidth, keepaspectratio=true]{examples/10_move_wave_cascading_wave_1.png}
\caption{Wave 1 cascading}
\label{fig:10_move_wave_cascading_wave_1}
% % \centering
\end{wrapfigure}
Activated Wave 1 inherits rules of movement from activating piece, and so now
moves as Rook would. It's not obstructed with any piece in its' way, nor it's
limited by received momentum, i.e. it can move further than just 4 fields away.
Wave 1 can also activate Wave 2.

\clearpage % ..........................................................

\noindent
\begin{wrapfigure}[13]{l}{0.5625\textwidth}
\includegraphics[width=0.5625\textwidth, keepaspectratio=true]{examples/10_move_wave_cascading_wave_2.png}
\caption{Wave 2 cascading}
\label{fig:10_move_wave_cascading_wave_2}
% % \centering
\end{wrapfigure}
Activated Wave 2 inherits rules of movement from activating piece (in this
case Wave 1), and so moves too as Rook would. Wave 2 also received momentum
of 4. Again, it's not obstructed by any piece, nor it's limited by received
momentum. Wave 2 can also activate either Queen or Rook.

\vspace*{0.075\textheight}
\noindent
\begin{wrapfigure}[9]{r}{0.5625\textwidth}
\includegraphics[width=0.5625\textwidth, keepaspectratio=true]{examples/10_move_wave_cascading_rook_b.png}
\caption{Rook, 2nd cascading}
\label{fig:10_move_wave_cascading_rook_b}
% % \centering
\end{wrapfigure}
Rook is now activated, but it is limited by momentum received, i.e. it can
move at most 4 fields away. Naturaly, Rook is obstructed by surrounding pieces,
i.e. it can't move past Wave 1. Rook can activate Wave 1.

\clearpage % ..........................................................

\noindent
\begin{wrapfigure}[10]{l}{0.5625\textwidth}
\includegraphics[width=0.5625\textwidth, keepaspectratio=true]{examples/10_move_wave_cascading_wave_1_b.png}
\caption{Wave 1, 2nd cascading}
\label{fig:10_move_wave_cascading_wave_1_b}
% % \centering
\end{wrapfigure}
Activated Wave 1 received momentum of 2 and rules of movement from Rook. Again,
Wave 1 is not obstructed by any piece, nor it is limited by received momentum.
Wave 1 can also activate either Queen or Wave 2.

\vspace*{0.155\textheight}
\noindent
\begin{wrapfigure}[10]{r}{0.5625\textwidth}
\includegraphics[width=0.5625\textwidth, keepaspectratio=true]{examples/10_move_wave_cascading_queen.png}
\caption{Queen cascading}
\label{fig:10_move_wave_cascading_queen}
% % \centering
\end{wrapfigure}
Activated Queen received momentum of 2, it can move at most 2 fields away.
Queen can also activate Wave 2, giving it momentum of 0.

\clearpage % ..........................................................

\noindent
\begin{wrapfigure}[7]{l}{0.5625\textwidth}
\includegraphics[width=0.5625\textwidth, keepaspectratio=true]{examples/10_move_wave_cascading_wave_2_b.png}
\caption{Wave 2, 2nd cascading}
\label{fig:10_move_wave_cascading_wave_2_b}
% % \centering
\end{wrapfigure}
Activated Wave 2 received rules of movement from Queen, and 0 momentum.
Note, since Wave 2 has no momentum it can't activate Rook, only Wave 1.

\vspace*{0.245\textheight}
\noindent
\begin{wrapfigure}[9]{r}{0.5625\textwidth}
\includegraphics[width=0.5625\textwidth, keepaspectratio=true]{examples/10_move_wave_cascading_wave_1_c.png}
\caption{Wave 1, 3rd cascading}
\label{fig:10_move_wave_cascading_wave_1_c}
% % \centering
\end{wrapfigure}
Activated Wave 1 inherits rules of movement from activating piece (Wave 2),
meaning Wave 1 too now moves as Queen would. Due to no momentum, Wave 1 can't
activate neither Queen nor Rook.

\clearpage % ..........................................................

\noindent
\begin{wrapfigure}[3]{l}{0.5625\textwidth}
\includegraphics[width=0.5625\textwidth, keepaspectratio=true]{examples/10_move_wave_cascading_end.png}
\caption{Wave 1, end cascading}
\label{fig:10_move_wave_cascading_end}
% % \centering
\end{wrapfigure}
Wave 1 end this rather long cascade by settling past Rook.

\vspace*{0.345\textheight}
In cascade, Wave is not limited by received momentum, nor it's obstructed by
other pieces on board.

Wave inherts rules of movement from activating piece. All other pieces moves
according to their own rules, and are restricted by momentum received.

\textbf{\huge{TODO :: link to "activated by/moves as" table}} % TODO :: FIX ME !!!

During cascade, after each ply (movement of a piece) activation takes place
according to current positioning of pieces on the board, just as it would
at the beginning of move.

This makes it possible, in the same cascade, to activate piece which started
it all, Rook in this example.

\clearpage % ..........................................................

\subsection*{Cascading opponent}
\addcontentsline{toc}{subsection}{Cascading opponent}

\noindent
\begin{wrapfigure}[10]{l}{0.5625\textwidth}
\includegraphics[width=0.5625\textwidth, keepaspectratio=true]{examples/10_move_wave_opponent_light_queen.png}
\caption{Light Queen starting cascade}
\label{fig:10_move_wave_opponent_light_queen}
% % \centering
\end{wrapfigure}
Praesent in sapien. Lorem ipsum dolor sit amet, consectetuer adipiscing elit.
Duis fringilla tristique neque. Sed interdum libero ut metus. Pellentesque placerat.
Nam rutrum augue a leo. Morbi sed elit sit amet ante lobortis sollicitudin.

\vspace*{0.175\textheight}
\noindent
\begin{wrapfigure}[10]{r}{0.5625\textwidth}
\includegraphics[width=0.5625\textwidth, keepaspectratio=true]{examples/10_move_wave_opponent_light_wave.png}
\caption{Light Wave}
\label{fig:10_move_wave_opponent_light_wave}
% % \centering
\end{wrapfigure}
Praesent in sapien. Lorem ipsum dolor sit amet, consectetuer adipiscing elit.
Duis fringilla tristique neque. Sed interdum libero ut metus. Pellentesque placerat.
Nam rutrum augue a leo. Morbi sed elit sit amet ante lobortis sollicitudin.

\clearpage % ..........................................................

\noindent
\begin{wrapfigure}[10]{l}{0.5625\textwidth}
\includegraphics[width=0.5625\textwidth, keepaspectratio=true]{examples/10_move_wave_opponent_dark_wave.png}
\caption{Dark Wave}
\label{fig:10_move_wave_opponent_dark_wave}
% % \centering
\end{wrapfigure}
Praesent in sapien. Lorem ipsum dolor sit amet, consectetuer adipiscing elit.
Duis fringilla tristique neque. Sed interdum libero ut metus. Pellentesque placerat.
Nam rutrum augue a leo. Morbi sed elit sit amet ante lobortis sollicitudin.

\vspace*{0.175\textheight}
\noindent
\begin{wrapfigure}[10]{r}{0.5625\textwidth}
\includegraphics[width=0.5625\textwidth, keepaspectratio=true]{examples/10_move_wave_opponent_dark_wave.png}
\caption{Dark Queen}
\label{fig:10_move_wave_opponent_dark_queen}
% % \centering
\end{wrapfigure}
Praesent in sapien. Lorem ipsum dolor sit amet, consectetuer adipiscing elit.
Duis fringilla tristique neque. Sed interdum libero ut metus. Pellentesque placerat.
Nam rutrum augue a leo. Morbi sed elit sit amet ante lobortis sollicitudin.


\clearpage % ..........................................................

\subsection*{Activation by Pawn}
\addcontentsline{toc}{subsection}{Activation by Pawn}

\noindent
\begin{figure}[!h]
% \begin{figure}[!t]
\includegraphics[width=1.0\textwidth, keepaspectratio=true]{examples/10_move_wave_activation_by_pawn.png}
\caption{Wave activation by Pawns}
\label{fig:10_move_wave_activation_by_pawn}
% \centering
\end{figure}

Pawns can activate Waves on own capture-field giving it 1 momentum (Pawn 1).
Pawns can also activate Waves on step-fields, giving it count of traveled-to
step-fields as momentum. For Pawn 2 that would be 1 momentum, while for Pawn 3
that would be 3.

\clearpage % ..........................................................

\section*{Initial setup}
\addcontentsline{toc}{section}{Initial setup}

Initial setup can be seen in image below:

\noindent
% \begin{figure}[t]
\begin{figure}[h]
\includegraphics[width=1.0\textwidth, keepaspectratio=true]{boards/10_miranda_s_veil.png}
\caption{Miranda's veil board}
\label{fig:10_miranda_s_veil}
% \centering
\end{figure}

\clearpage % ..........................................................
% ---------------------------------------------- Miranda's veil chapter
