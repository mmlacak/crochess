
% Copyright (c) 2015 - 2020 Mario Mlačak, mmlacak@gmail.com
% Public Domain work, under CC0 1.0 Universal Public Domain Dedication. See LICENSING, COPYING files for details.

% Miranda's veil chapter ==============================================
\chapter*{Miranda's veil}
\addcontentsline{toc}{chapter}{Miranda's veil}
\label{ch:Miranda's veil}

\begin{flushright}
\parbox{0.8\textwidth}{
\emph{Under all that we think, lives all we believe, like the ultimate veil of our spirits.\newline
\hspace*{\fill}{\textasciitilde{} Antonio Machado} } }
\end{flushright}

\noindent
Miranda's veil is chess variant which is played on 16~$\times$~16 board,
with white and dark violet fields and light magenta and indigo pieces.
A new piece is introduced, Wave.

\clearpage % ..........................................................
% Wave ****************************************************************

\section*{Wave}
\addcontentsline{toc}{section}{Wave}
\label{sec:Miranda's veil/Wave}

\vspace*{-1.4\baselineskip}
\noindent
\begin{wrapfigure}[11]{l}{0.4\textwidth}
\centering
\includegraphics[width=0.4\textwidth, keepaspectratio=true]{pieces/10_wave.png}
\caption{Wave}
\label{fig:10_wave}
\end{wrapfigure}
Wave is passive piece, it has to be activated before it can move. Activation is done
with own piece capturing a field at which Wave is located, before Wave can move.

Wave inherits step- and capture-fields from activating piece. Once initial step is
made, Wave cannot change its direction. Activated Wave can continue moving in the
initially chosen direction until the end of a chessboard is reached, even if
activated by a single-step piece (e.g. a Knight).

Wave does not use received momentum for moving, and so Wave can be activated even
with no momentum. Wave can activate any own piece, except King and Pyramid, if it
has momentum. Wave can also activate other Wave, own or opponent's, even if it has
no momentum. Wave transfers all of received momentum to a piece it activates.

Wave is transparent; all other pieces can move past (step over) Wave, as if it
isn't present on a chessboard. Other pieces are transparent to Wave; Wave can move
past (step over) any piece, as if it isn't there. Transparency makes activation of
Wave optional, and also prevents Wave to be pinned.

Wave cannot capture any piece; and so cannot neither check nor checkmate opponent's
King.

\clearpage % ..........................................................
% Activation ----------------------------------------------------------

\subsection*{Activation}
\addcontentsline{toc}{subsection}{Activation}
\label{sec:Miranda's veil/Wave/Activation}

\vspace*{-1.5\baselineskip}
\noindent
\begin{figure}[!h]
\includegraphics[width=1.0\textwidth, keepaspectratio=true]{examples/10_mv/scn_mv_001_wave_activation_init.png}
\vspace*{-1.4\baselineskip}
\caption{Activating Wave}
\label{fig:scn_mv_001_wave_activation_init}
\end{figure}

\vspace*{-0.5\baselineskip}
A piece can activate own Wave by simply capturing a field at which that Wave stands.
Activation is optional, a piece could just as well move past Wave. Activated Wave
receives any momentum activating piece had.

Here, Pegasus has opportunity to activate Wave, with 5 momentum.

\clearpage % ..........................................................

\vspace*{-2.1\baselineskip}
\noindent
\begin{figure}[!h]
\includegraphics[width=1.0\textwidth, keepaspectratio=true]{examples/10_mv/scn_mv_002_wave_activated.png}
\vspace*{-1.4\baselineskip}
\caption{Wave activated}
\label{fig:scn_mv_002_wave_activated}
\end{figure}

\vspace*{-0.5\baselineskip}
Activated Wave inherits way of moving from the activating piece; more specifically,
Wave inherits all initial step- and capture-fields which activating piece would
have when starting a ply. Activated Wave does not spend received momentum for moving,
and so Wave can be activated even if activating piece has no momentum.

Here, Wave activated by Pegasus (now "in the air") moves like one, i.e. along one
chosen semi-diagonal.

\clearpage % ..........................................................
% Activating pieces ...................................................

\subsubsection*{Activating pieces}
\addcontentsline{toc}{subsubsection}{Activating pieces}
\label{sec:Miranda's veil/Wave/Activation/Activating pieces}

\vspace*{-1.5\baselineskip}
\noindent
\begin{figure}[!h]
\includegraphics[width=1.0\textwidth, keepaspectratio=true]{examples/10_mv/scn_mv_003_pawn_pass_through.png}
\vspace*{-1.4\baselineskip}
\caption{Passing opponent's Pawn}
\label{fig:scn_mv_003_pawn_pass_through}
\end{figure}

\vspace*{-0.5\baselineskip}
Wave in its movement is not obstructed by any piece on chessboard, it can move
past (step over) any piece, as if it's not there. In short, other pieces are
all transparent to Wave, and all activations are optional.

Here, Wave cannot activate opponent's Pawn on its step-field, but it's not
hindered by that Pawn, and can reach fields behind it, which would be out of
reach for Pegasus.

\clearpage % ..........................................................

\vspace*{-2.1\baselineskip}
\noindent
\begin{figure}[!h]
\includegraphics[width=1.0\textwidth, keepaspectratio=true]{examples/10_mv/scn_mv_004_wave_activating_rook.png}
\vspace*{-1.3\baselineskip}
\caption{Activating Rook}
\label{fig:scn_mv_004_wave_activating_rook}
\end{figure}

\vspace*{-0.3\baselineskip}
Wave can activate any own piece, except King, if it has momentum. Wave can also
activate any other Wave, own or opponent's, even if it doesn't have any momentum.
Wave does not spend received momentum while moving, and would transfer it entirely
to any piece it activates.

Here, Wave can activate own Rook, even though it's positioned behind opponent's
Pawn, and transfer to it all of 5 received momentum.

\clearpage % ..........................................................

\vspace*{-2.1\baselineskip}
\noindent
\begin{figure}[!h]
\includegraphics[width=1.0\textwidth, keepaspectratio=true]{examples/10_mv/scn_mv_005_rook_activated.png}
\vspace*{-1.3\baselineskip}
\caption{Rook activated}
\label{fig:scn_mv_005_rook_activated}
\end{figure}

\vspace*{-0.3\baselineskip}
Material is any piece, except Wave. Activated material moves the same as it would
in a normal move, i.e. if not activated. The only difference is that activated
material is limited by received momentum, i.e. can't move for more fields than
momentum it received.

Here, activated Rook (now "in the air") can choose one of horizontals or verticals
as its new direction. Rook can reach at most 5 fields, because that's the momentum
it received.

\clearpage % ..........................................................

\vspace*{-2.1\baselineskip}
\noindent
\begin{figure}[!h]
\includegraphics[width=1.0\textwidth, keepaspectratio=true]{examples/10_mv/scn_mv_006_rook_captures.png}
% \vspace*{-1.3\baselineskip}
\caption{Rook captures}
\label{fig:scn_mv_006_rook_captures}
\end{figure}

Activated material piece can also capture opponent's piece, if it's within reach,
and not obstructed by other pieces.

Here, activated Rook can capture dark Knight; it can't capture dark Bishop since
own light Pawn is in the way. Light Rook can't capture dark Pegasus since it's out
of reach.

% ................................................... Activating pieces
\clearpage % ..........................................................
% Activating Pawn .....................................................

\subsubsection*{Activating Pawn}
\addcontentsline{toc}{subsubsection}{Activating Pawn}
\label{sec:Miranda's veil/Wave/Activation/Activating Pawn}

\vspace*{-1.4\baselineskip}
\noindent
\begin{figure}[!h]
\includegraphics[width=1.0\textwidth, keepaspectratio=true]{examples/10_mv/scn_mv_007_activating_rush_pawn_init.png}
\vspace*{-1.3\baselineskip}
\caption{Activating Pawns}
\label{fig:scn_mv_007_activating_rush_pawn_init}
\end{figure}

\vspace*{-0.3\baselineskip}
Image above and the next one both have two examples presented in parallel; on the left,
and to the right.

Activating Pawn in its initial position gives it ability to capture opponent's
piece or rush, i.e. perform longer initial movement. Pawn can be rushed only for
momentum received, but no more than longest rush move available, in this variant
up to (and including) 6 fields.

\clearpage % ..........................................................

\vspace*{-2.1\baselineskip}
\noindent
\begin{figure}[!h]
\includegraphics[width=1.0\textwidth, keepaspectratio=true]{examples/10_mv/scn_mv_008_activating_rush_pawn_end.png}
\caption{Pawns activated}
\label{fig:scn_mv_008_activating_rush_pawn_end}
\end{figure}

Pawn 1 received 4 momentum, and so when rushing it the furthest 2 fields are out
of reach. Pawn 2 had 13 momentum, but could use only 6 for rush, since this is the
longest rush movement available in this variant.

% ..................................................... Activating Pawn
\clearpage % ..........................................................
% Activating Pyramid ..................................................

\subsubsection*{Activating Pyramid}
\addcontentsline{toc}{subsubsection}{Activating Pyramid}
\label{sec:Miranda's veil/Wave/Activation/Activating Pyramid}

\vspace*{-0.9\baselineskip}
\noindent
\begin{wrapfigure}[7]{l}{0.4\textwidth}
\centering
\includegraphics[width=0.39375\textwidth, keepaspectratio=true]{examples/10_mv/scn_mv_009_not_activating_pyramid_by_wave.png}
\vspace*{-1.4\baselineskip}
\caption{Wave cannot activate Pyramid}
\label{fig:scn_mv_009_not_activating_pyramid_by_wave}
\end{wrapfigure}
Wave cannot activate Pyramid.

Here, light Wave is about to be activated by light Bishop with 3 momentum.
Light Wave can reach light Pyramid, but cannot interact with it; Wave can
use its transparency, and move past Pyramid.

\vspace*{7.7\baselineskip}
\noindent
\begin{wrapfigure}[10]{l}{0.4\textwidth}
\centering
\includegraphics[width=0.39375\textwidth, keepaspectratio=true]{examples/10_mv/scn_mv_010_activating_pyramid_cascade_pawn.png}
\vspace*{-1.4\baselineskip}
\caption{Activating Pyramid in a cascade}
\label{fig:scn_mv_010_activating_pyramid_cascade_pawn}
\end{wrapfigure}
Pyramid can be activated in a cascade by any material piece (i.e. any
piece other than Wave).

Here, light Wave is about to be activated by light Bishop with 3 momentum,
which can then activate light Pawn; Pawn can then activate Pyramid on its
capture-field, with 2 momentum.

% .................................................. Activating Pyramid
\clearpage % ..........................................................
% Wave is transparent .................................................

\subsubsection*{Wave is transparent}
\addcontentsline{toc}{subsubsection}{Wave is transparent}
\label{sec:Miranda's veil/Wave/Cascading Waves/Wave is transparent}

\vspace*{-1.4\baselineskip}
\noindent
\begin{figure}[!h]
\includegraphics[width=1.0\textwidth, keepaspectratio=true]{examples/10_mv/scn_mv_011_wave_is_transparent.png}
\vspace*{-1.4\baselineskip}
\caption{Wave is transparent}
\label{fig:scn_mv_011_wave_is_transparent}
\end{figure}

\vspace*{-0.5\baselineskip}
Just as other pieces are transparent to Wave, so is Wave transparent for all the
other pieces. Any interaction with a Wave is optional; a piece could activate own
Wave, it could capture opponent's Wave, or it could move past all Waves in its path,
and e.g. capture opponent's piece behind a Wave.\newline
\indent
Here, light Queen could capture dark Wave, activate own Wave, or capture dark
Pegasus; dark Pawn is shielded by its Pegasus.

\clearpage % ..........................................................

\vspace*{-2.1\baselineskip}
\noindent
\begin{figure}[!h]
\includegraphics[width=1.0\textwidth, keepaspectratio=true]{examples/10_mv/scn_mv_012_wave_cant_be_pinned.png}
\vspace*{-1.3\baselineskip}
\caption{Wave is not pinned}
\label{fig:scn_mv_012_wave_cant_be_pinned}
\end{figure}

\vspace*{-0.5\baselineskip}
Since it's transparent Wave cannot be pinned, i.e. a piece can ignore (step over)
Wave placed on its capture-field, and still check opponent's King.

Here, dark Pegasus checks light King, even though light Wave is on dark Pegasus'
capture-field. Any other piece positioned instead of light Wave would be
\href{https://en.wikipedia.org/wiki/Pin_(chess)#Absolute_pin}{hard-pinned},
and light King wouldn't be in check.

% ................................................. Wave is transparent
\clearpage % ..........................................................

\subsubsection*{Wave and castling}
\addcontentsline{toc}{subsubsection}{Wave and castling}
\label{sec:Miranda's veil/Wave/Cascading Waves/Wave and castling}

\vspace*{-0.4\baselineskip}
Wave is transparent, so it doesn't block castling if positioned between castling
pieces and their destination fields. Wave cannot be activated by castling piece,
so it blocks any castling which uses its field as a destination.

\vspace*{-0.4\baselineskip}
\noindent
\begin{figure}[!h]
\includegraphics[width=1.0\textwidth, keepaspectratio=true]{examples/10_mv/scn_mv_013_wave_no_block_castling_king.png}
\vspace*{-1.4\baselineskip}
\caption{Castling King}
\label{fig:scn_mv_013_wave_no_block_castling_king}
\end{figure}

\vspace*{-1.4\baselineskip}
\noindent
\begin{figure}[!h]
\includegraphics[width=1.0\textwidth, keepaspectratio=true]{examples/10_mv/scn_mv_014_wave_no_block_castling_rook.png}
\vspace*{-1.4\baselineskip}
\caption{Castling Rook}
\label{fig:scn_mv_014_wave_no_block_castling_rook}
\end{figure}

\vspace*{-0.4\baselineskip}
In previous example King initiated castling by moving onto a field past Wave, Rook
can now finish castling onto empty field between the two, so the whole castling move
is legal.

\vspace*{-0.4\baselineskip}
\noindent
\begin{figure}[!h]
\includegraphics[width=1.0\textwidth, keepaspectratio=true]{examples/10_mv/scn_mv_015_wave_block_castling_rook.png}
\vspace*{-1.4\baselineskip}
\caption{Castling Rook blocked}
\label{fig:scn_mv_015_wave_block_castling_rook}
\end{figure}

\vspace*{-0.4\baselineskip}
If in previous example King initiated castling by moving onto a field just past Wave,
Rook cannot castle onto field occupied by Wave, so the whole castling move is illegal.

\clearpage % ..........................................................

\subsubsection*{Single-step pieces and transparency}
\addcontentsline{toc}{subsubsection}{Single-step pieces and transparency}
\label{sec:Miranda's veil/Wave/Cascading Waves/Single-step pieces and transparency}

\vspace*{-0.7\baselineskip}
\noindent
\begin{wrapfigure}[13]{l}{0.4\textwidth}
\centering
\includegraphics[width=0.39375\textwidth, keepaspectratio=true]{examples/10_mv/scn_mv_016_pawn_blocked_by_opponents_wave.png}
\vspace*{-1.4\baselineskip}
\caption{Pawns blocked by Waves}
\label{fig:scn_mv_016_pawn_blocked_by_opponents_wave}
\end{wrapfigure}
Image on the left and the next one both have two examples presented in parallel,
each started by a labeled Pawn.\newline
\indent
To use transparency, a piece has to be able to reach a Wave, and then make another
step away from it, in the same ply. So, single-step piece (i.e. a Pawn, Knight, King,
or Unicorn) has to activate own, or capture opponent's Wave; otherwise it's blocked
by a Wave, regardless if own, or opponent's.

\vspace*{1.1\baselineskip}
\noindent
\begin{wrapfigure}[16]{l}{0.4\textwidth}
\centering
\includegraphics[width=0.39375\textwidth, keepaspectratio=true]{examples/10_mv/scn_mv_017_pawn_not_blocked_by_opponents_wave.png}
\vspace*{-1.4\baselineskip}
\caption{Pawns not blocked by Waves}
\label{fig:scn_mv_017_pawn_not_blocked_by_opponents_wave}
\end{wrapfigure}
In previous example light Pawn B is blocked from moving straight forward by dark
Wave on its step-field; other Waves can be either activated, or captured. No Wave
can be stepped over, since both Pawns can make only one step in a ply.\newline
\indent
Here, both Pawns can rush, thus making more than one step in a single ply. Opponent's,
dark Wave is blocking light Pawn B from settling onto step-field it occupies. Even so,
both Waves are transparent to light Pawns; each Pawn can now step over corresponding
Wave, and reach step-fields behind it.

\clearpage % ..........................................................

\subsubsection*{Piece blocked}
\addcontentsline{toc}{subsubsection}{Piece blocked}
\label{sec:Miranda's veil/Wave/Activation/Piece blocked}

\vspace*{-1.4\baselineskip}
\noindent
\begin{figure}[h]
\includegraphics[width=1.0\textwidth, keepaspectratio=true]{examples/10_mv/scn_mv_018_wave_no_activating_blocked_piece.png}
\caption{Piece blocked}
\label{fig:scn_mv_018_wave_no_activating_blocked_piece}
\end{figure}

Wave cannot activate blocked pieces, even if it has momentum. Here, Pawn is blocked
from moving forward by own Bishop, and there are no opponent's pieces on its
diagonal capture-fields. So, Wave cannot activate Pawn, even though it has one
momentum received from Knight.

% ---------------------------------------------------------- Activation
\clearpage % ..........................................................
% Movement ------------------------------------------------------------

\subsection*{Movement}
\addcontentsline{toc}{subsection}{Movement}
\label{sec:Miranda's veil/Wave/Movement}

\vspace*{-1.5\baselineskip}
\noindent
\begin{figure}[h]
\includegraphics[width=1.0\textwidth, keepaspectratio=true]{examples/10_mv/scn_mv_019_bishop_activating_wave.png}
\vspace*{-1.4\baselineskip}
\caption{Bishop activating Wave}
\label{fig:scn_mv_019_bishop_activating_wave}
\end{figure}

\vspace*{-0.5\baselineskip}
Activated Wave inherits choice of initial step-fields and way of movement from
activating piece; so, movement of activated Wave is the same as activating piece
starting a new ply. Wave activated by pieces which can move for only one field
(such as Pawn, Knight, King, and Unicorn) can move over multiple fields, by
repeating initial step multiple times. Again, activating Wave is optional,
activating piece could continue its movement past Wave.

\clearpage % ..........................................................

\vspace*{-2.1\baselineskip}
\noindent
\begin{figure}[!h]
\includegraphics[width=1.0\textwidth, keepaspectratio=true]{examples/10_mv/scn_mv_020_wave_activated_by_bishop.png}
\vspace*{-1.4\baselineskip}
\caption{Wave activated by Bishop}
\label{fig:scn_mv_020_wave_activated_by_bishop}
\end{figure}

\vspace*{-0.5\baselineskip}
Activated Wave is not limited by received momentum, and can move past (step over)
any piece as if not present.

Here, Wave (now "in the air") activated by Bishop moves like one, i.e. along one chosen
diagonal. Activated light Wave cannot activate dark Knight, but can activate own Pawn.
Wave is not obstructed by neither Pawn nor Knight, and can move past them. Wave is not
limited by 4 received momentum, and can reach edge of chessboard.

\clearpage % ..........................................................
% Activated by Knight .................................................

\subsubsection*{Activated by Knight}
\addcontentsline{toc}{subsubsection}{Activated by Knight}
\label{sec:Miranda's veil/Wave/Movement/Activated by Knight}

\vspace*{-1.4\baselineskip}
\noindent
\begin{figure}[h]
\includegraphics[width=1.0\textwidth, keepaspectratio=true]{examples/10_mv/scn_mv_021_knight_activating_wave.png}
\caption{Knight activating Wave}
\label{fig:scn_mv_021_knight_activating_wave}
\end{figure}

Wave can make multiple steps in a ply, even if activated by a piece which can make only
one step. Activated Wave can take one chosen direction, which cannot be changed later.

Here, Knight is about to activate Wave, and transfer to it one momentum.

\clearpage % ..........................................................

\vspace*{-2.1\baselineskip}
\noindent
\begin{figure}[!h]
\includegraphics[width=1.0\textwidth, keepaspectratio=true]{examples/10_mv/scn_mv_022_wave_activated_by_knight.png}
\caption{Wave activated by Knight}
\label{fig:scn_mv_022_wave_activated_by_knight}
\end{figure}

Here, Wave (now "in the air") activated by light Knight can choose one semi-diagonal
(corresponding to steps Knight can make), and then move over multiple step-fields, up
to the edge of chessboard. So, Wave activated by Knight moves like a Pegasus.

% ................................................. Activated by Knight
\clearpage % ..........................................................
% Activated by King ...................................................

\subsubsection*{Activated by King}
\addcontentsline{toc}{subsubsection}{Activated by King}
\label{sec:Miranda's veil/Wave/Movement/Activated by King}

\vspace*{-1.4\baselineskip}
\noindent
\begin{figure}[h]
\includegraphics[width=1.0\textwidth, keepaspectratio=true]{examples/10_mv/scn_mv_023_king_activating_wave.png}
\caption{King activating Wave}
\label{fig:scn_mv_023_king_activating_wave}
\end{figure}

Similarly, Wave activated by King can choose one direction along diagonals, horizontal
or vertical lines (corresponding to steps King can make).

\clearpage % ..........................................................

\vspace*{-2.1\baselineskip}
\noindent
\begin{figure}[!h]
\includegraphics[width=1.0\textwidth, keepaspectratio=true]{examples/10_mv/scn_mv_024_wave_activated_by_king.png}
% \vspace*{-1.3\baselineskip}
\caption{Wave activated by King}
\label{fig:scn_mv_024_wave_activated_by_king}
\end{figure}

Then, Wave (now "in the air") activated by King can move over multiple step-fields,
up to the edge of chessboard. Direction taken by activated Wave cannot be changed
for duration of a ply. So, Wave activated by King moves like a Queen.

% ................................................... Activated by King
\clearpage % ..........................................................
% Activated by stepping Pawn ..........................................

\subsubsection*{Activated by stepping Pawn}
\addcontentsline{toc}{subsubsection}{Activated by stepping Pawn}
\label{sec:Miranda's veil/Wave/Movement/Activated by stepping Pawn}

\vspace*{-1.5\baselineskip}
\noindent
\begin{figure}[!h]
\includegraphics[width=1.0\textwidth, keepaspectratio=true]{examples/10_mv/scn_mv_025_wave_activation_by_step_pawn.png}
\vspace*{-1.4\baselineskip}
\caption{Pawn activates Wave on step-field}
\label{fig:scn_mv_025_wave_activation_by_step_pawn}
\end{figure}

\vspace*{-0.5\baselineskip}
Pawn can activate Wave on its step-fields, from which Wave inherits its steps, a
straight forward step and two diagonal steps, towards opponent's initial positions.

Even though \hyperref[fig:scn_mv_011_wave_is_transparent]{Wave is transparent},
single-step piece (like Pawn here) has to activate Wave encountered on its step-field.
Rushing Pawn (just like other multi-step pieces) does not have to activate Wave, and
can continue rushing further instead.

\clearpage % ..........................................................

\vspace*{-2.1\baselineskip}
\noindent
\begin{figure}[!h]
\includegraphics[width=1.0\textwidth, keepaspectratio=true]{examples/10_mv/scn_mv_026_wave_activated_by_step_pawn.png}
\vspace*{-1.4\baselineskip}
\caption{Wave activated on Pawn's step-field}
\label{fig:scn_mv_026_wave_activated_by_step_pawn}
\end{figure}

\vspace*{-0.5\baselineskip}
Unlike Pawn, Wave can choose any step, including diagonal, even if that step-field
is empty. Wave can continue its movement in a chosen direction until the end of a
chessboard. Activated Wave (now "in-the-air") is transparent, so it can move past
any pieces on its step-fields (e.g. light Pyramid). Wave cannot change direction
once it starts moving; so here, light Bishop is out of reach.\newline
\indent
Wave activated by rushing Pawn behaves the same, except it receives more momentum.

% .......................................... Activated by stepping Pawn
\clearpage % ..........................................................
% Activated by capturing Pawn .........................................

\subsubsection*{Activated by capturing Pawn}
\addcontentsline{toc}{subsubsection}{Activated by capturing Pawn}
\label{sec:Miranda's veil/Wave/Movement/Activated by capturing Pawn}

\vspace*{-1.5\baselineskip}
\noindent
\begin{figure}[!h]
\includegraphics[width=1.0\textwidth, keepaspectratio=true]{examples/10_mv/scn_mv_027_wave_activation_by_capture_pawn.png}
\vspace*{-1.4\baselineskip}
\caption{Pawn activates Wave on capture-field}
\label{fig:scn_mv_027_wave_activation_by_capture_pawn}
\end{figure}

\vspace*{-0.5\baselineskip}
Wave activated by Pawn on its capture-field behaves the same as the one
\hyperref[fig:scn_mv_025_wave_activation_by_step_pawn]{activated on its step-field}.

Activated Wave inherits the same steps Pawn is having; i.e. a step straight forward
or diagonally, towards opponent's initial positions.\newline
\indent
Here, light Pawn is about to activate Wave on its capture-field, giving Wave 1
momentum.

\clearpage % ..........................................................

\vspace*{-2.1\baselineskip}
\noindent
\begin{figure}[!h]
\includegraphics[width=1.0\textwidth, keepaspectratio=true]{examples/10_mv/scn_mv_028_wave_activated_by_capture_pawn.png}
\vspace*{-1.4\baselineskip}
\caption{Wave activated on Pawn's capture-field}
\label{fig:scn_mv_028_wave_activated_by_capture_pawn}
\end{figure}

\vspace*{-0.5\baselineskip}
As before, Wave can choose any step, including diagonal, even if that step-field is
empty. Wave can continue its movement in initially chosen direction until the end of
a chessboard. \hyperref[fig:scn_mv_011_wave_is_transparent]{Wave is transparent}, so
it can step over any pieces on its step-fields (here, e.g. light Pyramid). Activated
Wave (now "in-the-air") cannot change direction once it starts moving; here, light
Bishop is out of reach.

% ......................................... Activated by capturing Pawn
\clearpage % ..........................................................
% Activated by Unicorn ................................................

\subsubsection*{Activated by Unicorn}
\addcontentsline{toc}{subsubsection}{Activated by Unicorn}
\label{sec:Miranda's veil/Wave/Movement/Activated by Unicorn}

\vspace*{-0.7\baselineskip}
\noindent
\begin{wrapfigure}[10]{l}{0.45\textwidth}
\centering
\includegraphics[width=0.4375\textwidth, keepaspectratio=true]{examples/10_mv/scn_mv_029_wave_same_color.png}
\vspace*{-0.3\baselineskip}
\caption{Wave short jump}
\label{fig:scn_mv_029_wave_same_color}
\end{wrapfigure}
Wave, activated by Unicorn on a field with the same color as Wave, has the same step-fields
as Knight has.

Wave activated on a field in opposite color can jump much longer, and has the same step-fields
as Unicorn has. For comparison, short steps are also numbered (grey).

\vspace*{0.7\baselineskip}
\noindent
\begin{wrapfigure}[18]{l}{0.7\textwidth}
\centering
\includegraphics[width=0.6875\textwidth, keepaspectratio=true]{examples/10_mv/scn_mv_030_wave_opposite_color.png}
\vspace*{-0.3\baselineskip}
\caption{Wave long jump}
\label{fig:scn_mv_030_wave_opposite_color}
\end{wrapfigure}
On two initial steps, Wave can freely choose any marked fields, regardless if it's long or short step.
If Wave was positioned on same-color field, first step would be short, and second one long; vice versa
if Wave started on opposite-color field. On all subsequent steps, Wave has to keep alternating between
the two initially chosen steps, for the remainder of a ply.

\clearpage % ..........................................................

\vspace*{-2.1\baselineskip}
\noindent
\begin{figure}[!h]
\includegraphics[width=1.0\textwidth, keepaspectratio=true]{examples/10_mv/scn_mv_031_wave_activation_by_unicorn_first_step.png}
\vspace*{-1.3\baselineskip}
\caption{Unicorn activates Wave}
\label{fig:scn_mv_031_wave_activation_by_unicorn_first_step}
\end{figure}

\vspace*{-0.3\baselineskip}
Here, light Wave is activated by Unicorn on the same-color (light) field, so all available
step-fields are short jumps, i.e. the same as Knight. For first step, Wave can choose any of
marked step-fields, including the one occupied by own piece (light Pawn on field 2). Normally,
own piece could be activated, leaving Wave in its position. In this particular case, light Pawn
is blocked from moving, so it can't be activated. Light Wave can still choose field 2 as a first
step, only it has to move past light Pawn on it.

\clearpage % ..........................................................

\vspace*{-2.1\baselineskip}
\noindent
\begin{figure}[!h]
\includegraphics[width=1.0\textwidth, keepaspectratio=true]{examples/10_mv/scn_mv_032_wave_activation_by_unicorn_second_step.png}
\caption{Wave activated by Unicorn, step 1}
\label{fig:scn_mv_032_wave_activation_by_unicorn_second_step}
\end{figure}

Here, after first step, light Wave is located on opposite-color (dark) field, so all available
step-fields are long jumps, which are the same as those of Unicorn. Dark Pawn on field 3 can't be
activated, because it's opponent's piece. Just as with light Pawn in previous example, that does
not prevent light Wave to choose field 3 as its second step, only it has to move over dark Pawn
on it, and continue moving further.

\clearpage % ..........................................................

\vspace*{-2.1\baselineskip}
\noindent
\begin{figure}[!h]
\includegraphics[width=1.0\textwidth, keepaspectratio=true]{examples/10_mv/scn_mv_033_wave_activation_by_unicorn_complete.png}
\caption{Wave activated by Unicorn, complete ply}
\label{fig:scn_mv_033_wave_activation_by_unicorn_complete}
\end{figure}

After second step is chosen, complete movement of Wave consists of alternating between the two initially
chosen steps, which Wave for the rest of a ply has to follow, e.g. after reaching field 4, it cannot move
to any other step-field (red). Light Wave could also activate dark Wave, in which case it would end it's
ply on dark Wave's field, and dark Wave would move away. Pieces on all other non-step fields are ignored
(Pawns).

\clearpage % ..........................................................

\subsubsection*{Out of board steps}
\addcontentsline{toc}{subsubsection}{Out of board steps}
\label{sec:Miranda's veil/Wave/Movement/Out of board steps}

\vspace*{-1.4\baselineskip}
\noindent
\begin{figure}[!h]
\includegraphics[width=1.0\textwidth, keepaspectratio=true]{examples/10_mv/scn_mv_034_wave_off_board.png}
\caption{Wave off-board steps}
\label{fig:scn_mv_034_wave_off_board}
\end{figure}

Here, light grey fields are virtual fields extending existing chessboard.
For Wave, it's legal to step outside of a board, and all subsequent steps
are also legal, as long as its ply ends on a board. So, Wave activated by
Unicorn can reach fields 1 and 2, even though it stepped outside of the
board. It is illegal for any piece, including Wave, to end its ply outside
of a board.

% ................................................ Activated by Unicorn
% ------------------------------------------------------------ Movement
\clearpage % ..........................................................
% Cascading Waves -----------------------------------------------------

\subsection*{Cascading Waves}
\addcontentsline{toc}{subsection}{Cascading Waves}
\label{sec:Miranda's veil/Wave/Cascading Waves}

\vspace*{-1.4\baselineskip}
\noindent
\begin{figure}[h]
\includegraphics[width=1.0\textwidth, keepaspectratio=true]{examples/10_mv/scn_mv_035_wave_cascading_init.png}
\caption{Cascade start}
\label{fig:scn_mv_035_wave_cascading_init}
\end{figure}

A Wave can also activate other Wave; movement of an activated Wave is the same as
activating Wave. Generally, activated Wave inherits way of movement from activating
piece.

Here, Wave B moves like a Bishop, because activating Wave A moved like a Bishop,
since it was activated by one.

\clearpage % ..........................................................

\vspace*{-2.1\baselineskip}
\noindent
\begin{figure}[h]
\includegraphics[width=1.0\textwidth, keepaspectratio=true]{examples/10_mv/scn_mv_036_wave_cascading_steps.png}
\vspace*{-1.4\baselineskip}
\caption{Active piece cascaded}
\label{fig:scn_mv_036_wave_cascading_steps}
\end{figure}

\vspace*{-0.4\baselineskip}
When piece activated in a cascade is not a Wave, it has its own rules of movement, and
Waves activated afterwards inherit them from that activating piece; such a piece is
called activator.

Here, Waves activated after Knight moves like multi-step Knight (i.e. Pegasus), since
Waves are not restricted to only one step, even if activator is. For Waves A, and B
activator is Bishop, while for Waves C, and D activator is Knight.

\clearpage % ..........................................................

\vspace*{-2.1\baselineskip}
\noindent
\begin{figure}[h]
\includegraphics[width=1.0\textwidth, keepaspectratio=true]{examples/10_mv/scn_mv_037_wave_cascading_end.png}
\vspace*{-1.3\baselineskip}
\caption{Cascade end}
\label{fig:scn_mv_037_wave_cascading_end}
\end{figure}

\vspace*{-0.3\baselineskip}
First piece in a cascade gathers momentum over step-fields traveled. All pieces
transfer all of momentum remaining after movement to the next piece in a cascade.
Wave doesn't spend received momentum for movement, but all other pieces do.

Here, numbers in lower, left corner are received momentum. Bishop gathered 3 momentum,
1 has been spent by Knight, and so activated Pawn can be rushed for only 2 fields.

\clearpage % ..........................................................

\subsubsection*{No momentum}
\addcontentsline{toc}{subsubsection}{No momentum}
\label{sec:Miranda's veil/Wave/Cascading Waves/No momentum}

\vspace*{-1.4\baselineskip}
\noindent
\begin{figure}[h]
\includegraphics[width=1.0\textwidth, keepaspectratio=true]{examples/10_mv/scn_mv_038_wave_no_momentum_no_activating.png}
\caption{No momentum}
\label{fig:scn_mv_038_wave_no_momentum_no_activating}
\end{figure}

Wave can be activated with no momentum, if so it can activate only other Waves, but
cannot activate material pieces. Here, one momentum originating from Bishop has been
already spent by Knight, so Wave B is activated with no momentum, and so it cannot
activate Pawn. Wave B can step over Pawn, and activate dark Wave, also with no momentum.

\clearpage % ..........................................................

\subsubsection*{Single-step piece and momentum}
\addcontentsline{toc}{subsubsection}{Single-step piece and momentum}
\label{sec:Miranda's veil/Wave/Cascading Waves/Single-step piece and momentum}

\vspace*{-1.5\baselineskip}
\noindent
\begin{figure}[h]
\includegraphics[width=1.0\textwidth, keepaspectratio=true]{examples/10_mv/scn_mv_039_single_step_piece_momentum.png}
\vspace*{-1.4\baselineskip}
\caption{Single-step piece and momentum}
\label{fig:scn_mv_039_single_step_piece_momentum}
\end{figure}

\vspace*{-0.5\baselineskip}
All pieces can receive any amount of momentum, and transfer all of unspent momentum
after movement to the next piece in a cascade; this includes pieces which can only
make single step in a ply, like Knight.\newline
\indent
Here, numbers in lower right corner are received momentum; Knight received 4 momentum,
and transferred remaining 3 to next Wave in a cascade, even though it can make only one
step in a ply.

\clearpage % ..........................................................
% Reactivating pieces .................................................

\subsubsection*{Reactivating pieces}
\addcontentsline{toc}{subsubsection}{Reactivating pieces}
\label{sec:Miranda's veil/Wave/Cascading Waves/Reactivating pieces}

\vspace*{-1.4\baselineskip}
\noindent
\begin{figure}[!h]
\includegraphics[width=1.0\textwidth, keepaspectratio=true]{examples/10_mv/scn_mv_043_reactivating_piece_init.png}
% \vspace*{-1.3\baselineskip}
\caption{Start reactivating piece}
\label{fig:scn_mv_043_reactivating_piece_init}
\end{figure}

% \vspace*{-0.3\baselineskip}
During cascade, after each ply activation takes place according to current position
of pieces on a chessboard, just as it would at the beginning of a move. Every piece
activated in a cascade can choose any legal direction of movement independently of
any previous choice.

\clearpage % ..........................................................

\vspace*{-2.1\baselineskip}
\noindent
\begin{figure}[!h]
\includegraphics[width=1.0\textwidth, keepaspectratio=true]{examples/10_mv/scn_mv_044_reactivating_piece_steps.png}
\vspace*{-1.3\baselineskip}
\caption{Reactivating piece steps}
\label{fig:scn_mv_044_reactivating_piece_steps}
\end{figure}

\vspace*{-0.3\baselineskip}
It's possible to re-activate piece which already participated in the same cascade;
reactivation takes place on a field occupied by piece at the beginning of that ply.

Here, Wave C (now "in the air") is about to reactivate Wave A, which can then e.g.
cascade Pegasus. Since Wave A has already been moved in cascade from its initial
position "a", so reactivation takes place on a changed position, i.e. current at
the beginning of reactivating ply.

% ................................................. Reactivating pieces
\clearpage % ..........................................................
% Cascading pinned piece ..............................................

\subsubsection*{Cascading pinned piece}
\addcontentsline{toc}{subsubsection}{Cascading pinned piece}
\label{sec:Miranda's veil/Wave/Cascading Waves/Cascading pinned piece}

\vspace*{-1.5\baselineskip}
\noindent
\begin{figure}[!h]
\includegraphics[width=1.0\textwidth, keepaspectratio=true]{examples/10_mv/scn_mv_045_pinned_piece_cascaded_init.png}
\vspace*{-1.3\baselineskip}
\caption{Light Queen is hard-pinned}
\label{fig:scn_mv_045_pinned_piece_cascaded_init}
\end{figure}

\vspace*{-0.5\baselineskip}
A piece hard-pinned to its King cannot move in a normal, non-cascading move, since
that would leave King checked. Whether King is checked or checkmated is determined
only after a move (a cascade) has been finished. So, in a cascade one could replace
hard-pinned piece with any other material piece; Wave can't be used since
\hyperref[fig:scn_mv_011_wave_is_transparent]{it's transparent}.

Here, light Queen is hard-pinned.

\clearpage % ..........................................................

\vspace*{-2.1\baselineskip}
\noindent
\begin{figure}[!h]
\includegraphics[width=1.0\textwidth, keepaspectratio=true]{examples/10_mv/scn_mv_046_pinned_piece_cascaded_1.png}
\vspace*{-1.3\baselineskip}
\caption{Cascading pinned piece}
\label{fig:scn_mv_046_pinned_piece_cascaded_1}
\end{figure}

\vspace*{-0.5\baselineskip}
During cascade, Wave can be the only piece on a pinning path (i.e. on opponent's
capture-field), as long as any material piece is pinned after that cascade has been
finished. Pinned piece doesn't have to be replaced at the same field; blocking any
capture-field which leads to own King will do.

Here, light Wave A is the only piece on dark Pegasus' capture-path after it activates
Queen; this is fine, Knight will be pinned after light player's move has been finished.

\clearpage % ..........................................................

\vspace*{-2.3\baselineskip}
\noindent
\begin{figure}[!h]
\includegraphics[width=1.0\textwidth, keepaspectratio=true]{examples/10_mv/scn_mv_048_pinned_piece_cascaded_2.png}
% \vspace*{-1.3\baselineskip}
\caption{Pinned piece starts a cascade}
\label{fig:scn_mv_048_pinned_piece_cascaded_2}
\end{figure}

% \vspace*{-0.5\baselineskip}
It's possible for pinned piece to start a cascade, leaving opponent's capture-path
empty. Similarly to previous example, this is also fine, as long as some other
material piece is pinned after that same cascade has been finished.

% .............................................. Cascading pinned piece
\clearpage % ..........................................................
% Activating piece, check, and checkmate ..............................

\subsubsection*{Activating piece, check, and checkmate}
\addcontentsline{toc}{subsubsection}{Activating piece, check, and checkmate}
\label{sec:Miranda's veil/Wave/Cascading Waves/Activating piece, check, and checkmate}

\vspace*{-0.7\baselineskip}
\noindent
\begin{wrapfigure}[15]{l}{0.4\textwidth}
\centering
\includegraphics[width=0.39375\textwidth, keepaspectratio=true]{examples/10_mv/scn_mv_049_activating_piece_check_init.png}
\vspace*{-1.4\baselineskip}
\caption{King not in check}
\label{fig:scn_mv_049_activating_piece_check_init}
\end{wrapfigure}
Opportunity to check (or checkmate) after activation is still not
an attack on opponent's King.\newline
\indent
Opponent's King can be checked (or checkmated) only if it's located
on a capture-field of active piece when move is finished.\newline
\indent
Passive pieces
\hyperref[fig:scn_ma_19_pyramid_vs_king]{cannot neither check nor checkmate}
opponent's King, even if they have capture-fields (e.g. Pyramid).\newline
\indent
Here, dark King is not in check yet, even if it would be located on light
Rook's capture-field after activation and repositioning.

% \vspace*{3.4\baselineskip}
\noindent
\begin{wrapfigure}[11]{l}{0.4\textwidth}
\centering
\includegraphics[width=0.39375\textwidth, keepaspectratio=true]{examples/10_mv/scn_mv_050_activating_piece_check_end.png}
\vspace*{-1.4\baselineskip}
\caption{King checked}
\label{fig:scn_mv_050_activating_piece_check_end}
\end{wrapfigure}
Only after all pieces in cascade has settled, and move has finished, can a piece
check (or checkmate) opponent's King.\newline
\indent
On the left, grey arrows show path traveled over by a piece they point to.
Light player's move has been finished, now is dark player's turn. Dark King
is now positioned on light Rook's capture-field, so now it's in check.

% .............................. Activating piece, check, and checkmate
\clearpage % ..........................................................
% Cascade check, checkmate ............................................

\subsubsection*{Cascade check, checkmate}
\addcontentsline{toc}{subsubsection}{Cascade check, checkmate}
\label{sec:Miranda's veil/Wave/Cascading Waves/Cascade check, checkmate}

\vspace*{-1.4\baselineskip}
\noindent
\begin{figure}[!h]
\includegraphics[width=1.0\textwidth, keepaspectratio=true]{examples/10_mv/scn_mv_051_cascaded_piece_check_init.png}
\caption{Activating Queen}
\label{fig:scn_mv_051_cascaded_piece_check_init}
\end{figure}

Again, King is checked (or checkmated) only after cascade has finished; just like
after normal, non-cascaded move. During cascade, piece can be reactivated on a
temporary field, and thus repositioned to a new field. So, a piece can temporary
be located on a field where it would check (or checkmate) King, if that would be
its final destination for that cascade. When piece is repositioned from its
temporary location, King is not affected, and game continues.

\clearpage % ..........................................................

\vspace*{-2.3\baselineskip}
\noindent
\begin{figure}[!h]
\includegraphics[width=1.0\textwidth, keepaspectratio=true]{examples/10_mv/scn_mv_052_cascaded_piece_check_end.png}
\vspace*{-1.3\baselineskip}
\caption{Reactivating Queen}
\label{fig:scn_mv_052_cascaded_piece_check_end}
\end{figure}

\vspace*{-0.3\baselineskip}
Here, light Queen after being positioned on a temporary field does not attack dark
King, because all of its momentum has been transferred to Wave B, then C and finally
D. Wave D is now "in the air", about to reactivate light Queen with remaining 5
momentum; grey arrows show path traveled over by piece they point to. After
reactivation, light Queen still won't attack dark King, even though it's located on
light Queen's capture-field, and within range. This is so because being checked (or
checkmated) is a status of a position, after all pieces had settled down, and move
has been done. This means, cascade would have to finish with activated light Queen
settling onto e.g. field 2, or 4, for that Queen to check dark King.

% ............................................ Cascade check, checkmate
% ----------------------------------------------------- Cascading Waves
\clearpage % ..........................................................
% Cascading opponent --------------------------------------------------

\subsection*{Cascading opponent}
\addcontentsline{toc}{subsection}{Cascading opponent}
\label{sec:Miranda's veil/Wave/Cascading opponent}

\vspace*{-1.4\baselineskip}
\noindent
\begin{figure}[h]
\includegraphics[width=1.0\textwidth, keepaspectratio=true]{examples/10_mv/scn_mv_065_wave_cascading_opponent.png}
% \vspace*{-1.3\baselineskip}
\caption{Cascading opponent}
\label{fig:scn_mv_065_wave_cascading_opponent}
\end{figure}

% \vspace*{-0.3\baselineskip}
Own Wave can activate opponent's Wave, and vice versa, opponent's Wave can activate
own Wave. In both cases activated Wave moves the same way activating Wave does.
Opponent's Wave can also activate any other opponent's piece, except King. Note,
color of the first piece in a cascade matches color of a player who started that
cascade, thus determines which pieces are own and which are opponent's.

% \hyperref[fig:scn_mv_037_wave_cascading_end]{cascade with all own pieces}

\clearpage % ..........................................................

\vspace*{-2.1\baselineskip}
\noindent
\begin{figure}[h]
\includegraphics[width=1.0\textwidth, keepaspectratio=true]{examples/10_mv/scn_mv_066_cascaded_opponent_capturing.png}
\caption{Cascaded opponent capturing piece}
\label{fig:scn_mv_066_cascaded_opponent_capturing}
\end{figure}

Opponent's pieces, activated in own cascade, keep all of their behavior as if in a
normal move, for instance capturing their opponent's (in own cascade, that would be
own!) pieces.

Here, dark Knight, in a cascade started by light player, is not (and cannot be)
activating light Wave, it's just capturing it.

\clearpage % ..........................................................

\vspace*{-2.1\baselineskip}
\noindent
\begin{figure}[h]
\includegraphics[width=1.0\textwidth, keepaspectratio=true]{examples/10_mv/scn_mv_067_cascaded_opponent_promoting.png}
\caption{Cascaded opponent promoting Pawn}
\label{fig:scn_mv_067_cascaded_opponent_promoting}
\end{figure}

Opponent's Pawn in own cascade can be promoted only to other opponent's pieces,
this includes opponent's Pawns tagged for promotion.

Here, dark Pawn, in a cascade started by light player, is not (and cannot be)
promoted to light piece, it's being promoted to dark piece, e.g. dark Queen.

\clearpage % ..........................................................
% Cascade self-checkmate ..............................................

\subsubsection*{Cascade self-checkmate}
\addcontentsline{toc}{subsubsection}{Cascade self-checkmate}
\label{sec:Miranda's veil/Wave/Cascading opponent/Cascade self-checkmate}

\vspace*{-1.5\baselineskip}
\noindent
\begin{figure}[h]
\includegraphics[width=1.0\textwidth, keepaspectratio=true]{examples/10_mv/scn_mv_068_cascade_self_checkmate_init.png}
\vspace*{-1.4\baselineskip}
\caption{Cascading opponent's Rook}
\label{fig:scn_mv_068_cascade_self_checkmate_init}
\end{figure}

\vspace*{-0.4\baselineskip}
Opponent's piece, activated in own cascade, can be positioned on a field where
it would check own King if cascade has finished; this is fine as long as that piece
is reactivated and
\hyperref[fig:scn_mv_051_cascaded_piece_check_init]{moved away from its temporary position}.
Opponent's piece left in a position from where it does check own King after own cascade
is finished leads to immediate self-induced checkmate, since now it's opponent's turn,
and own King can't be removed from check.

\clearpage % ..........................................................

\vspace*{-2.1\baselineskip}
\noindent
\begin{figure}[h]
\includegraphics[width=1.0\textwidth, keepaspectratio=true]{examples/10_mv/scn_mv_069_cascade_self_checkmate_end.png}
\vspace*{-1.4\baselineskip}
\caption{Cascaded self-checkmate}
\label{fig:scn_mv_069_cascade_self_checkmate_end}
\end{figure}

\vspace*{-0.4\baselineskip}
Here, light player's cascade has finished; notice, first piece in a cascade is light
Bishop. Light player has left dark Rook in a position to check light King; grey arrows
show path traveled over by a piece they point to. Now is dark player's turn; since
light King can't be moved out of a check, this is immediate self-checkmate.

% .............................................. Cascade self-checkmate
\clearpage % ..........................................................
% Double checkmate ....................................................

\subsubsection*{Double checkmate}
\addcontentsline{toc}{subsubsection}{Double checkmate}
\label{sec:Miranda's veil/Wave/Cascading opponent/Double checkmate}

\vspace*{-1.5\baselineskip}
\noindent
\begin{figure}[h]
\includegraphics[width=1.0\textwidth, keepaspectratio=true]{examples/10_mv/scn_mv_070_cascade_double_checkmate_init.png}
\vspace*{-1.4\baselineskip}
\caption{Cascading own and opponent's pieces}
\label{fig:scn_mv_070_cascade_double_checkmate_init}
\end{figure}

\vspace*{-0.4\baselineskip}
Again, King is checkmated only
\hyperref[fig:scn_mv_051_cascaded_piece_check_init]{after move has finished}, and
all pieces have settled onto their final destinations. In a single turn player can
checkmate self in addition to checkmating opponent; the result is a draw.\newline
\indent
Here, it's light player's turn; which is about to move light Queen and activate
light Wave, dark King isn't checkmated yet since the move is still ongoing. Light
Wave would then activate dark Wave, which would activate dark Rook.

\clearpage % ..........................................................

\vspace*{-2.1\baselineskip}
\noindent
\begin{figure}[h]
\includegraphics[width=1.0\textwidth, keepaspectratio=true]{examples/10_mv/scn_mv_071_cascade_double_checkmate_end.png}
\vspace*{-1.4\baselineskip}
\caption{Cascaded checkmate and self-checkmate}
\label{fig:scn_mv_071_cascade_double_checkmate_end}
\end{figure}

\vspace*{-0.4\baselineskip}
Here, light player's move has finished, now it's dark player's turn; grey arrows
show path traveled over by a piece they point to. Both Kings are checkmated;
light King since it's in check and it's dark player's turn; dark King since there
are no more legal moves left.

The result of double checkmate is always a draw; regardless which player made a
move, regardless which King was encountered first in a cascade.

% .................................................... Double checkmate
\clearpage % ..........................................................
% Wave blocked ........................................................

\subsubsection*{Wave blocked}
\addcontentsline{toc}{subsubsection}{Wave blocked}
\label{sec:Miranda's veil/Wave/Cascading opponent/Wave blocked}

\vspace*{-1.5\baselineskip}
\noindent
\begin{figure}[h]
\includegraphics[width=1.0\textwidth, keepaspectratio=true]{examples/10_mv/scn_mv_072_wave_blocked_init.png}
\vspace*{-1.4\baselineskip}
\caption{Activating Wave}
\label{fig:scn_mv_072_wave_blocked_init}
\end{figure}

\vspace*{-0.4\baselineskip}
Wave cannot activate opponent's pieces, except for Waves. Activated Wave which movement
is completely blocked is oblationed, i.e. is removed from chessboard as if captured by
opponent.

Here, dark Wave B is about to be activated with one momentum.

\clearpage % ..........................................................

\vspace*{-2.1\baselineskip}
\noindent
\begin{figure}[h]
\includegraphics[width=1.0\textwidth, keepaspectratio=true]{examples/10_mv/scn_mv_073_wave_blocked_end.png}
\vspace*{-1.4\baselineskip}
\caption{Activated Wave blocked}
\label{fig:scn_mv_073_wave_blocked_end}
\end{figure}

\vspace*{-0.4\baselineskip}
Here, dark Wave B is activated without committing its movement yet (it's "in the air").
All accessible step-fields are blocked by opponent's light pieces, which cannot be
activated by dark Wave, even though it has one momentum.
Note, Wave (just like any other piece) has to move away from its starting position,
it cannot stay and re-activate piece that has activated it (here, light Wave 2).
Thus, dark Wave B is oblationed, i.e. removed from chessboard.

% ........................................................ Wave blocked
\clearpage % ..........................................................
% Activating opponent's Wave ..........................................

\subsubsection*{Activating opponent's Wave}
\addcontentsline{toc}{subsubsection}{Activating opponent's Wave}
\label{sec:Miranda's veil/Wave/Movement/Activating opponent's Wave}

\vspace*{-1.5\baselineskip}
\noindent
\begin{figure}[!h]
\includegraphics[width=1.0\textwidth, keepaspectratio=true]{examples/10_mv/scn_mv_078_activating_opponents_wave.png}
\vspace*{-1.4\baselineskip}
\caption{Dark Pawn activating light Wave}
\label{fig:scn_mv_078_activating_opponents_wave}
\end{figure}

\vspace*{-0.5\baselineskip}
Opponent's Wave can be activated indirectly, and it would inherit from its
\hyperref[sec:Terms/Activator]{activator} all capture and movement steps,
\hyperref[fig:scn_mv_002_wave_activated]{the same as before}. When
\hyperref[fig:scn_mv_027_wave_activation_by_capture_pawn]{activating piece is Pawn},
opponent's Wave would also have movement direction swapped backwards, i.e. towards
own initial positions.

Here, dark Pawn is about to activate light Wave, indirectly, on its capture-field.

\clearpage % ..........................................................

\vspace*{-2.1\baselineskip}
\noindent
\begin{figure}[!h]
\includegraphics[width=1.0\textwidth, keepaspectratio=true]{examples/10_mv/scn_mv_079_activated_opponents_wave.png}
\vspace*{-1.4\baselineskip}
\caption{Dark Pawn activated light Wave}
\label{fig:scn_mv_079_activated_opponents_wave}
\end{figure}

\vspace*{-0.5\baselineskip}
Here, activated light Wave (now "in-the-air") can choose any dark Pawn's initial
step as its direction of movement, i.e. either straight downwards, or diagonally,
towards light player's initial positions; compare this pattern to light Wave
\hyperref[fig:scn_mv_028_wave_activated_by_capture_pawn]{activated by own, light Pawn}.\newline
\indent
Note also that activated light Wave retains its behavior, and cannot activate dark
pieces, only light ones; with the only exception being dark Wave. Wave can always
activate any other Wave, regardless of colors.

% .......................................... Activating opponent's Wave
% -------------------------------------------------- Cascading opponent
\clearpage % ..........................................................
% En passant ----------------------------------------------------------

\subsection*{En passant}
\addcontentsline{toc}{subsection}{En passant}
\label{sec:Miranda's veil/Wave/En passant}

Until this variant the only difference from Classical Chess was the possibility
for a Pawn to rush for more than two fields; otherwise, rush and en passant were
the same as in Classical Chess, and with the same rules still in place:\newline
\href{https://en.wikipedia.org/wiki/En\_passant}{https://en.wikipedia.org/wiki/En\_passant}.

Now, due to transparency of Waves, ability to cascade own and opponent's pieces, it's
possible for a Pawn capturing en passant to encounter pieces on its capture-field,
and between a capture-field and a rushed Pawn, immediately after opponent's cascade
has finished.

En passant turns into ordinary capture if a piece on Pawn's capture-field can be
captured, in which case rushed Pawn can't be captured en passant anymore.
En passant is blocked if a piece on a capture-field can't be captured, e.g. if
it's in the same color as capturing Pawn. En passant is denied if rushed Pawn was
(re)activated and moved away from its rushing destination, in the very same
cascade.

In all other cases en passant is still possible; for instance, if a piece on a
capture-field can be activated, or if there are pieces between a capture-field
and a rushed Pawn.

Note, Pawn can capture en passant if it started a move, or on its first ply in a
cascade, if activated. Pawn cannot capture en passant if it has already moved in
the same cascade.

\clearpage % ..........................................................
% En passant turned capture ...........................................

\subsubsection*{En passant turned capture}
\addcontentsline{toc}{subsubsection}{En passant turned capture}
\label{sec:Miranda's veil/Wave/En passant/En passant turned capture}

\vspace*{-0.7\baselineskip}
\noindent
\begin{wrapfigure}[15]{l}{0.4\textwidth}
\centering
\includegraphics[width=0.39375\textwidth, keepaspectratio=true]{examples/10_mv/scn_mv_080_en_passant_rushing_cascade.png}
\vspace*{-1.4\baselineskip}
\caption{Rushing cascade}
\label{fig:scn_mv_080_en_passant_rushing_cascade}
\end{wrapfigure}
Rushing Pawn can start a cascade in which other own piece is placed onto opponent's
en passant capture-field.
Opponent's Pawn can then capture a piece, but cannot capture rushing Pawn en passant
anymore.\newline
\indent
On the left, dark Pawn is positioned so that it would gain en passant opportunity,
should light Pawn rush over field E. Rushing Pawn is about to start a cascade by
activating light Wave, which can then activate Bishop, which can then activate
Pyramid.

% \TODO :: fix lmodern
% \clearpage % ..........................................................

% \vspace*{-2.1\baselineskip}
\noindent
\begin{wrapfigure}[7]{l}{0.4\textwidth}
\centering
\includegraphics[width=0.39375\textwidth, keepaspectratio=true]{examples/10_mv/scn_mv_081_en_passant_turning_capture.png}
\vspace*{-1.4\baselineskip}
\caption{Setting-up a figure}
\label{fig:scn_mv_081_en_passant_turning_capture}
\end{wrapfigure}
Grey arrows show path traveled over by a piece they point to; taken together they
show first part of the cascade, which is already done.\newline
\indent
Here, activated light Pyramid (now "in the air") is about to move to its destination
field E.

\clearpage % ..........................................................

\vspace*{-2.1\baselineskip}
\noindent
\begin{wrapfigure}[11]{l}{0.4\textwidth}
\centering
\includegraphics[width=0.39375\textwidth, keepaspectratio=true]{examples/10_mv/scn_mv_082_en_passant_turned_capture.png}
\vspace*{-1.4\baselineskip}
\caption{Capturing figure instead}
\label{fig:scn_mv_082_en_passant_turned_capture}
\end{wrapfigure}
On the left, since activated Pyramid has settled onto field E, rushing light Pawn
cascade has finished, and now it's dark player's turn.\newline
\indent
Even though it rushed over field E on the very previous turn, light Pawn cannot
be captured en passant, because dark Pawn is capturing light Pyramid instead,
since that is positioned on its capture-field.

% \TODO :: fix lmodern

% \clearpage % ..........................................................

\vspace*{1.7\baselineskip}
\noindent
\begin{wrapfigure}[13]{l}{0.4\textwidth}
\centering
\includegraphics[width=0.39375\textwidth, keepaspectratio=true]{examples/10_mv/scn_mv_083_en_passant_wave_captured.png}
\vspace*{-1.4\baselineskip}
\caption{Capturing Wave instead}
\label{fig:scn_mv_083_en_passant_wave_captured}
\end{wrapfigure}
Since \hyperref[fig:scn_mv_011_wave_is_transparent]{Wave is transparent}, rushing
Pawn can step over it, and can continue rushing further. Any piece captured instead
of a rushed Pawn prevents that Pawn being captured en passant.

In a new example on the left similar to previous one, grey arrows show path traveled
over by rushing light Pawn, from its starting field P. Now on dark player's turn,
dark Pawn cannot realize en passant opportunity, and can only capture light Wave
instead, since Wave is positioned on its capture-field E.

Rushing Pawn could also be activated, capturing a figure prevents en passant all
the same; only difference is that momentum gathered has to be enough to also drive
rushing Pawn, and a figure onto en passant capture-field.

% ........................................... En passant turned capture
\clearpage % ..........................................................
% En passant blocked ..................................................

\subsubsection*{En passant blocked}
\addcontentsline{toc}{subsubsection}{En passant blocked}
\label{sec:Miranda's veil/Wave/En passant/En passant blocked}

\vspace*{-0.7\baselineskip}
\noindent
\begin{wrapfigure}[15]{l}{0.4\textwidth}
\centering
\includegraphics[width=0.39375\textwidth, keepaspectratio=true]{examples/10_mv/scn_mv_084_rushing_cascade_opponent.png}
\vspace*{-1.4\baselineskip}
\caption{Rushing cascade}
\label{fig:scn_mv_084_rushing_cascade_opponent}
\end{wrapfigure}
Rushing Pawn can start a cascade in which opponent's piece is placed onto opponent's
Pawn capture-field.
Opponent's Pawn is then blocked by its own piece, and cannot capture rushing Pawn en
passant anymore.

On the left, dark Pawn is positioned so that it would gain en passant opportunity,
should light Pawn rush over field E. Rushing Pawn is about to start a cascade by
activating light Wave A, continuing to Bishop, light Wave B, dark Wave, ending
with dark Rook.

% \TODO :: fix lmodern

% \vspace*{13.7\baselineskip}
\noindent
\begin{wrapfigure}[7]{l}{0.4\textwidth}
\centering
\includegraphics[width=0.39375\textwidth, keepaspectratio=true]{examples/10_mv/scn_mv_085_blocking_en_passant.png}
\vspace*{-1.4\baselineskip}
\caption{Blocking en passant}
\label{fig:scn_mv_085_blocking_en_passant}
\end{wrapfigure}
Grey arrows show path traveled over by a piece they point to; taken together they
show first part of the cascade, which is already done.\newline
\indent
Here, activated dark Rook (now "in the air") is about to move to its destination
field E.

\clearpage % ..........................................................

\vspace*{-2.1\baselineskip}
\noindent
\begin{wrapfigure}[10]{l}{0.4\textwidth}
\centering
\includegraphics[width=0.39375\textwidth, keepaspectratio=true]{examples/10_mv/scn_mv_086_blocked_en_passant.png}
\vspace*{-1.4\baselineskip}
\caption{Blocked en passant}
\label{fig:scn_mv_086_blocked_en_passant}
\end{wrapfigure}
On the left, since activated Rook has settled onto field E, rushing light Pawn
cascade has finished, and now it's dark player's turn.

Even though it rushed over field E on the very previous turn, light Pawn cannot
be captured en passant, because dark Pawn is blocked by its own dark Rook, and
so it cannot move to its capture-field.

% .................................................. En passant blocked
% \clearpage % ..........................................................
% En passant denied ...................................................

\vspace*{3.1\baselineskip}
\subsubsection*{En passant denied}
\addcontentsline{toc}{subsubsection}{En passant denied}
\label{sec:Miranda's veil/Wave/En passant/En passant denied}

\vspace*{-0.7\baselineskip}
\noindent
\begin{wrapfigure}[10]{l}{0.4\textwidth}
\centering
\includegraphics[width=0.39375\textwidth, keepaspectratio=true]{examples/10_mv/scn_mv_087_en_passant_denied_init.png}
\vspace*{-1.4\baselineskip}
\caption{Rushing light Pawn}
\label{fig:scn_mv_087_en_passant_denied_init}
\end{wrapfigure}
Rushing Pawn can be (re)activated and moved away from its rushing destination, all
in the very same cascade, thus denying en passant opportunity to opponent.

Here, light Pawn is about to rush forward and activate light Wave A, which would
then activate light Bishop, which would activate light Wave C.

\clearpage % ..........................................................

% \vspace*{1.7\baselineskip}
\vspace*{-2.1\baselineskip}
\noindent
\begin{wrapfigure}[13]{l}{0.4\textwidth}
\centering
\includegraphics[width=0.39375\textwidth, keepaspectratio=true]{examples/10_mv/scn_mv_088_en_passant_denied_pawn_activated.png}
\vspace*{-1.4\baselineskip}
\caption{Activating light Pawn}
\label{fig:scn_mv_088_en_passant_denied_pawn_activated}
\end{wrapfigure}
Rushed Pawn is \hyperref[fig:scn_cc_09_tags_rushing]{tagged as en passant opportunity}
immediately after its rushing ply is finished. Tag is a link between a piece and its
field; so, tag (and opportunity it represents) is lost when a piece is moved away.\newline
\indent
Here, light Pawn has rushed in previous ply, now is tagged as en passant opportunity.
Activated light Wave C (now "in-the-air") is about to activate light Pawn, which would
then move a field forward.

% \clearpage % ..........................................................

% \vspace*{-2.1\baselineskip}
\noindent
\begin{wrapfigure}[10]{l}{0.4\textwidth}
\centering
\includegraphics[width=0.39375\textwidth, keepaspectratio=true]{examples/10_mv/scn_mv_089_en_passant_denied_end.png}
\vspace*{-1.4\baselineskip}
\caption{En passant denied}
\label{fig:scn_mv_089_en_passant_denied_end}
\end{wrapfigure}
Here, light Pawn has moved away from its rushing destination, and so it also lost
en passant opportunity tag, all in a previous light player's cascade.

Now is dark player's turn, but it's illegal for a dark Pawn to try capture light
Pawn en passant, since it doesn't have its tag anymore.

% ................................................... En passant denied
\clearpage % ..........................................................

\subsubsection*{Activation after en passant}
\addcontentsline{toc}{subsubsection}{Activation after en passant}
\label{sec:Miranda's veil/Wave/En passant/Activation after en passant}

\vspace*{-0.7\baselineskip}
\noindent
\begin{wrapfigure}[8]{l}{0.4\textwidth}
\centering
\includegraphics[width=0.39375\textwidth, keepaspectratio=true]{examples/10_mv/scn_mv_090_activation_after_en_passant_init.png}
\vspace*{-1.4\baselineskip}
\caption{Rushing light Pawn}
\label{fig:scn_mv_090_activation_after_en_passant_init}
\end{wrapfigure}
Opponent's Wave is also transparent to a rushing Pawn; while Pawn cannot interact
with opponent's Wave on its step-fields, Pawn can step over it, and continue
rushing.

Here, light Pawn is rushing from field P, over dark Wave, onto its destination
field R.

\vspace*{2.7\baselineskip}
\noindent
\begin{wrapfigure}[10]{l}{0.4\textwidth}
\centering
\includegraphics[width=0.39375\textwidth, keepaspectratio=true]{examples/10_mv/scn_mv_091_activation_after_en_passant_end.png}
\vspace*{-1.4\baselineskip}
\caption{Activation after en passant}
\label{fig:scn_mv_091_activation_after_en_passant_end}
\end{wrapfigure}
Activating a piece does not prevent a Pawn from capturing another en passant, since
activation is continuation of a cascade, not side-effect of a movement.

Here, dark Pawn is about to activate dark Wave on its en passant capture-field,
light Pawn would be captured en passant, and dark Wave would continue a cascade.

\clearpage % ..........................................................
% En passant not blocked ..............................................

\subsubsection*{En passant not blocked}
\addcontentsline{toc}{subsubsection}{En passant not blocked}
\label{sec:Miranda's veil/Wave/En passant/En passant not blocked}

\vspace*{-0.7\baselineskip}
\noindent
\begin{wrapfigure}[8]{l}{0.4\textwidth}
\centering
\includegraphics[width=0.39375\textwidth, keepaspectratio=true]{examples/10_mv/scn_mv_092_en_passant_not_blocked_init.png}
\vspace*{-1.4\baselineskip}
\caption{Rushing cascade}
\label{fig:scn_mv_092_en_passant_not_blocked_init}
\end{wrapfigure}
Pieces on fields other than en passant capture-field do not block capture, regardless
if light or dark.

Here, light Pawn is about to rush forward and activate light Wave, which would
then activate light Bishop, which would activate light Pyramid.

\vspace*{3.7\baselineskip}
\noindent
\begin{wrapfigure}[11]{l}{0.4\textwidth}
\centering
\includegraphics[width=0.39375\textwidth, keepaspectratio=true]{examples/10_mv/scn_mv_093_en_passant_not_blocked_step_2.png}
\vspace*{-1.4\baselineskip}
\caption{Continuing cascade}
\label{fig:scn_mv_093_en_passant_not_blocked_step_2}
\end{wrapfigure}
Here, light Pawn rushed forward in the very first ply, after which it has been
tagged as en passant opportunity (green marker). Other pieces have advanced in
their plies, too; grey arrows now show path traveled over by a piece they point
to.

Light Pyramid (now "in-the-air") is about to land on a field between rushed light
Pawn and en passant capture-field E.

\clearpage % ..........................................................

\vspace*{-2.1\baselineskip}
\noindent
\begin{wrapfigure}[9]{l}{0.4\textwidth}
\centering
\includegraphics[width=0.39375\textwidth, keepaspectratio=true]{examples/10_mv/scn_mv_094_en_passant_not_blocked_end.png}
\vspace*{-1.4\baselineskip}
\caption{En passant not blocked}
\label{fig:scn_mv_094_en_passant_not_blocked_end}
\end{wrapfigure}
Here, light player's cascade has been finished, now it's dark player turn. Rushed
light Pawn hasn't been moved from its destination, so it retains its tag.

Dark Pawn can capture light Pawn en passant, regardless of light Pyramid in-between
its capture-field E and light Pawn.

% .............................................. En passant not blocked
% \clearpage % ..........................................................

\vspace*{2.7\baselineskip}
\subsubsection*{En passant legal}
\addcontentsline{toc}{subsubsection}{En passant legal}
\label{sec:Miranda's veil/Wave/En passant/En passant legal}

\vspace*{-0.7\baselineskip}
\noindent
\begin{wrapfigure}[13]{l}{0.4\textwidth}
\centering
\includegraphics[width=0.39375\textwidth, keepaspectratio=true]{examples/10_mv/scn_mv_096_en_passant_legal_end.png}
\vspace*{-1.4\baselineskip}
\caption{Activated Dark Pawn capturing en passant}
\label{fig:scn_mv_096_en_passant_legal_end}
\end{wrapfigure}
Pawn can be activated, and capture opponent's Pawn en passant on its first ply
in a cascade.

Here, light Pawn has rushed in previous turn, and it's tagged as en passant
opportunity (green marker); grey arrows show path traveled over by the Pawn.\newline
\indent
Now it's dark player's turn; dark Bishop is about to activate dark Wave, which
will then activate dark Pawn, which can capture light Pawn en passant.

\clearpage % ..........................................................
% En passant illegal ..................................................

\subsubsection*{En passant illegal}
\addcontentsline{toc}{subsubsection}{En passant illegal}
\label{sec:Miranda's veil/Wave/En passant/En passant illegal}

\vspace*{-0.7\baselineskip}
\noindent
\begin{wrapfigure}[25]{l}{0.4\textwidth}
\centering
\includegraphics[width=0.39375\textwidth, keepaspectratio=true]{examples/10_mv/scn_mv_097_en_passant_illegal_init.png}
\vspace*{-1.4\baselineskip}
\caption{Dark Pawn rushing after light one}
\label{fig:scn_mv_097_en_passant_illegal_init}
\end{wrapfigure}
Again, Pawn can capture another en passant only on its first ply in a move
regardless if it started a move, or was activated.\newline
\indent
Pawn cannot capture en passant if it was reactivated, i.e. if it has already
moved in the same cascade.\newline
\indent
So, capturing Pawn has to be the only piece moving in a turn, or it has to capture
en passant on its first ply in a move; and cannot capture on its second, third,
... ply in the same cascade.

Here, light Pawn has rushed to its destination, and now is tagged as en passant
opportunity (green marker), grey arrows show path traveled over by the Pawn.\newline
\indent
Now it's dark player turn, dark Pawn is about to rush forward and activate dark
Wave A, which would then activate dark Queen, which would activate dark Wave C.

Next images of development in the example here will show only middle part of an
chessboard, since there is nothing interesting elsewhere.

\clearpage % ..........................................................

\vspace*{-2.1\baselineskip}
\noindent
\begin{wrapfigure}[8]{l}{0.4\textwidth}
\centering
\includegraphics[width=0.39375\textwidth, keepaspectratio=true]{examples/10_mv/scn_mv_098_en_passant_illegal_pawn_activated.png}
\vspace*{-1.4\baselineskip}
\caption{Activating dark Pawn}
\label{fig:scn_mv_098_en_passant_illegal_pawn_activated}
\end{wrapfigure}
In the first ply dark Pawn has rushed, after which it was tagged as en passant
opportunity (also, green marker).\newline
\indent
Here, activated dark Wave C (now "in-the-air") is about to step over dark Wave B
and activate dark Pawn, with 5 momentum.

\vspace*{-0.3\baselineskip}
\noindent
\begin{wrapfigure}[7]{l}{0.4\textwidth}
\centering
\includegraphics[width=0.39375\textwidth, keepaspectratio=true]{examples/10_mv/scn_mv_099_en_passant_illegal_queen_reactivated.png}
\vspace*{-1.4\baselineskip}
\caption{Reactivating dark Queen}
\label{fig:scn_mv_099_en_passant_illegal_queen_reactivated}
\end{wrapfigure}
Here, activated dark Pawn (now "in-the-air") has lost its en passant opportunity
tag (and, green marker), and is about to activate dark Wave B, which would then
activate dark Queen, which would activate dark Wave A.

\vspace*{0.7\baselineskip}
\noindent
\begin{wrapfigure}[11]{l}{0.4\textwidth}
\centering
\includegraphics[width=0.39375\textwidth, keepaspectratio=true]{examples/10_mv/scn_mv_100_en_passant_illegal_pawn_reactivated.png}
\vspace*{-1.4\baselineskip}
\caption{Reactivating dark Pawn}
\label{fig:scn_mv_100_en_passant_illegal_pawn_reactivated}
\end{wrapfigure}
Here, activated dark Wave A (now "in-the-air") is activating dark Pawn, with 4
momentum.\newline
\indent
Even though dark Pawn is about to be activated with enough momentum and can reach
en passant capture-field E, it cannot capture light Pawn en passant anymore since
dark Pawn was activated, and so that ply is not the very first one in the move.

As shown here, if Pawns already moved in a cascade could capture en passant in their
second, third, ... ply, it would be possible to capture from their initial positions;
that certainly would not be "in passing".

% .................................................. En passant illegal
% ---------------------------------------------------------- En passant
% **************************************************************** Wave
\clearpage % ..........................................................

\section*{Rush, en passant}
\addcontentsline{toc}{section}{Rush, en passant}
\label{sec:Miranda's veil/Rush, en passant}

\noindent
\begin{wrapfigure}[5]{l}{0.4\textwidth}
\centering
\includegraphics[width=0.20625\textwidth, keepaspectratio=true]{en_passants/10_miranda_s_veil_en_passant.png}
\caption{En passant}
\label{fig:10_miranda_s_veil_en_passant}
\end{wrapfigure}
Rush and en passant are identical to those in Classic Chess, only difference
is that Pawn can now move longer on initial turn, up to 6 fields in this
variant.

% \clearpage % ..........................................................

\vspace*{9.0\baselineskip}
\section*{Promotion}
\addcontentsline{toc}{section}{Promotion}
\label{sec:Miranda's veil/Promotion}

Promotion is non enforced, delayed variety, i.e. it's the same as in
\hyperref[sec:Age of Aquarius/Promotion]{previous chess variant}, Age of Aquarius.

\clearpage % ..........................................................

\section*{Castling}
\addcontentsline{toc}{section}{Castling}
\label{sec:Miranda's veil/Castling}

Castling is the same as in Classical Chess, only difference is that King can move between 2 and 6 fields across.
All other constraints from Classical Chess still applies.

\vspace*{-0.7\baselineskip}
\noindent
\begin{figure}[!h]
\includegraphics[width=1.0\textwidth, keepaspectratio=true]{castlings/10_mv/miranda_s_veil_castling_00.png}
\caption{Castling}
\label{fig:miranda_s_veil_castling_00}
\end{figure}

Above, all valid King's castling moves are numbered.

\vspace*{-0.7\baselineskip}
\noindent
\begin{figure}[!h]
\includegraphics[width=1.0\textwidth, keepaspectratio=true]{castlings/10_mv/miranda_s_veil_castling_right_15_05.png}
\caption{Castling long right}
\label{fig:miranda_s_veil_castling_right_15_05}
\end{figure}

In this example King was castling long to the right. Initial King's position is marked with "K".
After castling is finished, right Rook ends up at field immediately left to the King.

Again, \hyperref[fig:scn_mv_011_wave_is_transparent]{Wave is transparent}, so it does
not block castling if it's positioned between castling pieces and their destination
fields.
Wave cannot be activated by castling pieces, so Wave blocks both King and Rook from castling
\hyperref[fig:scn_mv_015_wave_block_castling_rook]{onto a field it occupies}.

\clearpage % ..........................................................

\section*{Initial setup}
\addcontentsline{toc}{section}{Initial setup}
\label{sec:Miranda's veil/Initial setup}

Compared to initial setup of Age of Aquarius, Wave is inserted between Knight and Unicorn
symmetrically, on both sides of chessboard. This can be seen in the image below:

\noindent
\begin{figure}[h]
\includegraphics[width=1.0\textwidth, keepaspectratio=true]{boards/10_miranda_s_veil.png}
\caption{Miranda's veil board}
\label{fig:10_miranda_s_veil}
\end{figure}

\clearpage % ..........................................................
% ============================================== Miranda's veil chapter
