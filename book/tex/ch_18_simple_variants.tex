
% Copyright (c) 2025 Mario Mlačak, mmlacak@gmail.com
% Public Domain work, under CC0 1.0 Universal Public Domain Dedication. See LICENSING, COPYING files for details.

% Simple variants chapter =============================================
\chapter*{Simple variants}
\addcontentsline{toc}{chapter}{Simple variants}
\label{ch:Simple variants}

Simple variants described here are meant to make playing on large chessboards
more approachable to enthusiasts and casual players.

% \clearpage % ..........................................................
% Classical Chess variants ********************************************

\section*{Classical Chess variants}
\addcontentsline{toc}{section}{Classical Chess variants}
\label{sec:Simple variants/Classical Chess variants}

Classical Chess variants are just a simple enlargement of a Classical Chess,
with the same pieces and the same rules, with two simple enhancements to the
already existing rules; those are extended rushing and castling.

New variants are made by taking Rook, Bishop, Knight and their accompanying
Pawns on both flanks, and adding them all together in the same order once,
twice, and finally thrice to their respective flank.

To maximize playing area all chessboard in all new variants are squares,
resulting in variants which are played on a 14~$\times$~14, 20~$\times$~20,
and finally 26~$\times$~26 chessboards.\newline
\indent
New variants are referred to by their size, that is Classical 14, Classical 20
and Classical 26 Chess.

\clearpage % ..........................................................

\subsection*{Initial setups}
\addcontentsline{toc}{subsection}{Initial setups}
\label{sec:Simple variants/Classical Chess variants/Initial setups}

Compared to initial setup of Clasical Chess, additional Rooks, Bishops and Knights
are appended on both sides of Classical 14 chessboard:

\noindent
\begin{figure}[h]
\includegraphics[width=1.0\textwidth, keepaspectratio=true]{boards/24_classic14.png}
\caption{Classical 14 board}
\label{fig:24_classic14}
\end{figure}

\vfill{}


% ******************************************** Classical Chess variants
\clearpage % ..........................................................
% Croatian Ties variants **********************************************

\section*{Croatian Ties variants}
\addcontentsline{toc}{section}{Croatian Ties variants}
\label{sec:Simple variants/Croatian Ties variants}

\TODO

% ********************************************** Croatian Ties variants
\clearpage % ..........................................................
% ============================================= Simple variants chapter
