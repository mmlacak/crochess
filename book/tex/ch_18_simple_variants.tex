
% Copyright (c) 2025 Mario Mlačak, mmlacak@gmail.com
% Public Domain work, under CC0 1.0 Universal Public Domain Dedication. See LICENSING, COPYING files for details.

% Simple variants chapter =============================================
\chapter*{Simple variants}
\addcontentsline{toc}{chapter}{Simple variants}
\label{ch:Simple variants}

Simple variants described here are meant to make playing on large chessboards
more approachable to enthusiasts and casual players.

% \clearpage % ..........................................................
% Classical Chess variants ********************************************

\section*{Classical Chess variants}
\addcontentsline{toc}{section}{Classical Chess variants}
\label{sec:Simple variants/Classical Chess variants}

Classical Chess variants are just a simple enlargement of a Classical Chess,
with the same pieces and the same rules, with two simple enhancements to the
already existing rules; those are extended rushing and castling.

New variants are made by taking Rook, Bishop, Knight and their accompanying
Pawns on both flanks, and adding them all together in the same order once,
twice, and finally thrice to their respective flank.

To maximize playing area all chessboards in all new variants are squares,
resulting in variants which are played on a 14~$\times$~14, 20~$\times$~20,
and finally 26~$\times$~26 chessboards.\newline
\indent
New variants are referred to by their size, that is Classical 14, Classical 20
and Classical 26 Chess.

\clearpage % ..........................................................

\subsection*{Rush, en passant}
\addcontentsline{toc}{subsection}{Rush, en passant}
\label{sec:Simple variants/Classical Chess variants/Rush, en passant}

\noindent
\begin{wrapfigure}{l}{0.4\textwidth}
\centering
\includegraphics[width=0.235714286\textwidth, keepaspectratio=true]{en_passants/24_classic14_en_passant.png}
\caption{Rush, en passant, Classical 14 Chess}
\label{fig:24_classic14_en_passant}
\end{wrapfigure}
Rushing and en passant are
\href{https://en.wikipedia.org/wiki/En\_passant}{the same as in Classical Chess},
except on its initial move Pawn can now move forward for more than just two
fields.

Pawns in all new variants can rush to any field on their side of a chessboard;
for light player that means onto any field on bottom half of a chessboard, for
dark player upper half.

So, Pawn can rush up to, and including, 5 fields forward in Classical 14 Chess;
up to (and including) 8 fields forward in Classical 20 Chess; and up to (and
including) 11 fields forward in Classical 26 Chess.

% \clearpage % ..........................................................

\subsection*{Castling}
\addcontentsline{toc}{subsection}{Castling}
\label{sec:Simple variants/Classical Chess variants/Castling}

Castling is
\href{https://en.wikipedia.org/wiki/Castling}{the same as in Classical Chess},
except on its initial move King can now move sideways for more than just two
fields.

In all new variants, King can castle only with a Rook that is the closest to the
King on the chosen side of castling, King's or Queen's. King can castle to any
field towards Rook on King's side of a chessboard, and one field short on Queen's
side.

So, King can castle up to, and including, 5 fields in Classical 14 Chess;
up to (and including) 8 fields in Classical 20 Chess; and up to (and
including) 11 fields in Classical 26 Chess.

\noindent
\begin{figure}[!h]
\includegraphics[width=1.0\textwidth, keepaspectratio=true]{castlings/24_c14/classic14_castling_03.png}
\vspace*{-1.4\baselineskip}
\caption{Castling short, Classical 14 Chess}
\label{fig:classic14_castling_03}
\end{figure}

. . .

\noindent
\begin{figure}[!h]
\includegraphics[width=1.0\textwidth, keepaspectratio=true]{castlings/24_c14/classic14_castling_00.png}
\vspace*{-1.4\baselineskip}
\caption{Castling long, Classical 14 Chess}
\label{fig:classic14_castling_00}
\end{figure}

. . .

\noindent
\begin{figure}[!h]
\includegraphics[width=1.0\textwidth, keepaspectratio=true]{castlings/26_c20/classic20_castling_00.png}
\vspace*{-1.4\baselineskip}
\caption{Castling long, Classical 20 Chess}
\label{fig:classic20_castling_00}
\end{figure}

. . .

\noindent
\begin{figure}[!h]
\includegraphics[width=1.0\textwidth, keepaspectratio=true]{castlings/28_c26/classic26_castling_00.png}
\vspace*{-1.4\baselineskip}
\caption{Castling long, Classical 26 Chess}
\label{fig:classic26_castling_00}
\end{figure}

. . .

\clearpage % ..........................................................

\subsection*{Initial setups}
\addcontentsline{toc}{subsection}{Initial setups}
\label{sec:Simple variants/Classical Chess variants/Initial setups}

Compared to initial setup of Classical Chess, additional Rooks, Bishops and Knights
are added on both sides of Classical 14 chessboard:

\noindent
\begin{figure}[h]
\includegraphics[width=1.0\textwidth, keepaspectratio=true]{boards/24_classic14.png}
\caption{Classical 14 board}
\label{fig:24_classic14}
\end{figure}

\vfill{}

\clearpage % ..........................................................

Compared to initial setup of Classical Chess, two sets of additional Rooks, Bishops
and Knights are added on both sides of Classical 20 chessboard:

\noindent
\begin{figure}[h]
\includegraphics[width=1.0\textwidth, keepaspectratio=true]{boards/26_classic20.png}
\caption{Classical 20 board}
\label{fig:26_classic20}
\end{figure}

\vfill{}

\clearpage % ..........................................................

Compared to initial setup of Classical Chess, three sets of additional Rooks, Bishops
and Knights are added on both sides of Classical 26 chessboard:

\noindent
\begin{figure}[h]
\includegraphics[width=1.0\textwidth, keepaspectratio=true]{boards/28_classic26.png}
\caption{Classical 26 board}
\label{fig:28_classic26}
\end{figure}

\vfill{}

% ******************************************** Classical Chess variants
\clearpage % ..........................................................
% Croatian Ties variants **********************************************

\section*{Croatian Ties variants}
\addcontentsline{toc}{section}{Croatian Ties variants}
\label{sec:Simple variants/Croatian Ties variants}

\TODO

% ********************************************** Croatian Ties variants
\clearpage % ..........................................................
% Summary *************************************************************

\section*{Summary}
\addcontentsline{toc}{section}{Summary}
\label{sec:Simple variants/Summary}

\TODO

% ************************************************************* Summary
\clearpage % ..........................................................
% ============================================= Simple variants chapter
