
% Copyright (c) 2025 Mario Mlačak, mmlacak@gmail.com
% Public Domain work, under CC0 1.0 Universal Public Domain Dedication. See LICENSING, COPYING files for details.

% Simple variants chapter =============================================
\chapter*{Simple variants}
\addcontentsline{toc}{chapter}{Simple variants}
\label{ch:Simple variants}

Simple variants described here are meant to make playing on large chessboards
more approachable to enthusiasts and casual players.

% \clearpage % ..........................................................
% Classical Chess variants ********************************************

\section*{Classical Chess variants}
\addcontentsline{toc}{section}{Classical Chess variants}
\label{sec:Simple variants/Classical Chess variants}

Classical Chess variants are just a simple enlargement of a Classical Chess,
with the same pieces and the same rules, with two simple enhancements to the
already existing rules; those are extended rushing and castling.

New variants are made by taking Rook, Bishop, Knight and their accompanying
Pawns on both flanks, and adding them all together in the same order once,
twice, and finally thrice to their respective flank.

To maximize playing area all chessboards in all new variants are squares,
resulting in variants which are played on a 14~$\times$~14, 20~$\times$~20,
and finally 26~$\times$~26 chessboards.\newline
\indent
New variants are referred to by their size, that is Classical 14, Classical 20
and Classical 26 Chess.

\clearpage % ..........................................................

\subsection*{Rush, en passant}
\addcontentsline{toc}{subsection}{Rush, en passant}
\label{sec:Simple variants/Classical Chess variants/Rush, en passant}

\noindent
\begin{wrapfigure}{l}{0.4\textwidth}
\centering
\includegraphics[width=0.235714286\textwidth, keepaspectratio=true]{en_passants/24_classic14_en_passant.png}
\caption{Rush, en passant, Classical 14 Chess}
\label{fig:24_classic14_en_passant}
\end{wrapfigure}
Rushing and en passant are
\href{https://en.wikipedia.org/wiki/En\_passant}{the same as in Classical Chess},
except on its initial move Pawn can now move forward for more than just two
fields.

Pawns in all new variants can rush to any field on their side of a chessboard;
for light player that means onto any field on bottom half of a chessboard, for
dark player upper half.

So, Pawn can rush up to, and including, 5 fields forward in Classical 14 Chess;
up to (and including) 8 fields forward in Classical 20 Chess; and up to (and
including) 11 fields forward in Classical 26 Chess.

\vfill{}

\clearpage % ..........................................................

\subsection*{Castling}
\addcontentsline{toc}{subsection}{Castling}
\label{sec:Simple variants/Classical Chess variants/Castling}

Castling is
\href{https://en.wikipedia.org/wiki/Castling}{the same as in Classical Chess},
except on its initial move King can now move sideways for more than just two
fields.

In all new variants, King can castle only with a Rook that is the closest to the
King on the chosen side of castling, King's or Queen's. King can castle onto any
field towards Rook (up to, and including, the field right next to the Rook) on
King's side of a chessboard, and one field short on Queen's side.

So, King can castle up to, and including, 5 fields in Classical 14 Chess;
up to (and including) 8 fields in Classical 20 Chess; and up to (and
including) 11 fields in Classical 26 Chess.

\noindent
\begin{figure}[!h]
\includegraphics[width=1.0\textwidth, keepaspectratio=true]{castlings/24_c14/classic14_castling_03.png}
\vspace*{-1.4\baselineskip}
\caption{Castling short, Classical 14 Chess}
\label{fig:classic14_castling_03}
\end{figure}

Castling with Rook closest to the King is exactly the same as in Classical Chess,
since that part is carried over.

\noindent
\begin{figure}[!h]
\includegraphics[width=1.0\textwidth, keepaspectratio=true]{castlings/24_c14/classic14_castling_00.png}
\vspace*{-1.4\baselineskip}
\caption{Castling long, Classical 14 Chess}
\label{fig:classic14_castling_00}
\end{figure}

Castling with distant Rook just adds more destinations for King to choose from.

\vfill{}

\clearpage % ..........................................................

In any case, after castling, Rook ends up at field right next to King; on the left
if castling was at King's side, on the right if Queen's side.

\noindent
\begin{figure}[!h]
\includegraphics[width=1.0\textwidth, keepaspectratio=true]{castlings/24_c14/classic14_castling_right_13_02.png}
\vspace*{-1.4\baselineskip}
\caption{Castling short right, Classical 14 Chess}
\label{fig:classic14_castling_right_13_02}
\end{figure}

All Rooks, in all new variants, can castle with King ending just 2 fields away from
its initial position. Here, King castled short with distant Rook; should have been
present, near Rook could do the same.

\noindent
\begin{figure}[!h]
\includegraphics[width=1.0\textwidth, keepaspectratio=true]{castlings/24_c14/classic14_castling_left_00_05.png}
\vspace*{-1.4\baselineskip}
\caption{Castling long left, Classical 14 Chess}
\label{fig:classic14_castling_left_00_05}
\end{figure}

Only distant Rooks can castle with long castling King.

\clearpage % ..........................................................

\subsection*{Initial setups}
\addcontentsline{toc}{subsection}{Initial setups}
\label{sec:Simple variants/Classical Chess variants/Initial setups}

Compared to initial setup of Classical Chess, additional Rooks, Bishops and Knights
are added on both sides of Classical 14 chessboard:

\noindent
\begin{figure}[h]
\includegraphics[width=1.0\textwidth, keepaspectratio=true]{boards/24_classic14.png}
\caption{Classical 14 board}
\label{fig:24_classic14}
\end{figure}

\vfill{}

\clearpage % ..........................................................

Compared to initial setup of Classical Chess, two sets of additional Rooks, Bishops
and Knights are added on both sides of Classical 20 chessboard:

\noindent
\begin{figure}[h]
\includegraphics[width=1.0\textwidth, keepaspectratio=true]{boards/26_classic20.png}
\caption{Classical 20 board}
\label{fig:26_classic20}
\end{figure}

\vfill{}

\clearpage % ..........................................................

Compared to initial setup of Classical Chess, three sets of additional Rooks, Bishops
and Knights are added on both sides of Classical 26 chessboard:

\noindent
\begin{figure}[h]
\includegraphics[width=1.0\textwidth, keepaspectratio=true]{boards/28_classic26.png}
\caption{Classical 26 board}
\label{fig:28_classic26}
\end{figure}

\vfill{}

% ******************************************** Classical Chess variants
\clearpage % ..........................................................
% Croatian Ties variants **********************************************

\section*{Croatian Ties variants}
\addcontentsline{toc}{section}{Croatian Ties variants}
\label{sec:Simple variants/Croatian Ties variants}

Croatian Ties variants are the same as
\hyperref[sec:Simple variants/Classical Chess variants]{Classical Chess variants},
with only two changes; every Knight is replaced by Pegasus, and all Pawns can now
move sideways.

New variants are referred to by their size, that is Croatian Ties 14, Croatian
Ties 20 and Croatian Ties 26 Chess.

% \subsection*{Rush, en passant}
% \addcontentsline{toc}{subsection}{Rush, en passant}
% \label{sec:Simple variants/Croatian Ties variants/Rush, en passant}

% Rushing and en passant are exactly
% \hyperref[sec:Simple variants/Classical Chess variants/Rush, en passant]{the same as in Classical Chess variants}.

% \subsection*{Castling}
% \addcontentsline{toc}{subsection}{Castling}
% \label{sec:Simple variants/Croatian Ties variants/Castling}

% Castling is exactly
% \hyperref[sec:Simple variants/Classical Chess variants/Castling]{the same as in Classical Chess variants}.

\subsection*{Sideways Pawns}
\addcontentsline{toc}{subsection}{Sideways Pawns}
\label{sec:Simple variants/Croatian Ties variants/Sideways Pawns}

\noindent
\begin{wrapfigure}{l}{0.4\textwidth}
\centering
\includegraphics[width=0.378571429\textwidth, keepaspectratio=true]{examples/30_ct14/scn_ct14_01_sideways_pawns.png}
\caption{Sideways Pawns, Croatian Ties 14 Chess}
\label{fig:scn_ct14_01_sideways_pawns}
\end{wrapfigure}
In addition to going forward, Pawns in all Croatian Ties variants can also move
sideways, onto an empty field immediately to the left or to the right.

The same as going forward, Pawns moving sideways cannot capture opponent's pieces,
and are blocked by any piece standing immediately left or right to a Pawn.

\vfill{}

\clearpage % ..........................................................

\subsection*{Initial setups}
\addcontentsline{toc}{subsection}{Initial setups}
\label{sec:Simple variants/Croatian Ties variants/Initial setups}

Initial setup on Croatian Ties 14 chessboard is
\hyperref[fig:24_classic14]{the same as Classical 14 Chess}, except all Knights
are replaced by Pegasuses:

\noindent
\begin{figure}[h]
\includegraphics[width=1.0\textwidth, keepaspectratio=true]{boards/30_croatian14.png}
\caption{Croatian Ties 14 board}
\label{fig:30_croatian14}
\end{figure}

\vfill{}

\clearpage % ..........................................................


Initial setup on Croatian Ties 20 chessboard is
\hyperref[fig:26_classic20]{the same as Classical 20 Chess}, except all Knights
are replaced by Pegasuses:

\noindent
\begin{figure}[h]
\includegraphics[width=1.0\textwidth, keepaspectratio=true]{boards/32_croatian20.png}
\caption{Croatian Ties 20 board}
\label{fig:32_croatian20}
\end{figure}

\vfill{}

\clearpage % ..........................................................

Initial setup on Croatian Ties 26 chessboard is
\hyperref[fig:28_classic26]{the same as Classical 26 Chess}, except all Knights
are replaced by Pegasuses:

\noindent
\begin{figure}[h]
\includegraphics[width=1.0\textwidth, keepaspectratio=true]{boards/34_croatian26.png}
\caption{Croatian Ties 26 board}
\label{fig:34_croatian26}
\end{figure}

\vfill{}

% ********************************************** Croatian Ties variants
\clearpage % ..........................................................
% Summary *************************************************************

\section*{Summary}
\addcontentsline{toc}{section}{Summary}
\label{sec:Simple variants/Summary}

\TODO

% ************************************************************* Summary
\clearpage % ..........................................................
% ============================================= Simple variants chapter
