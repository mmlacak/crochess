
% Conquest of Tlalocan chapter ========================================
\chapter*{Conquest of Tlalocan}
\addcontentsline{toc}{chapter}{Conquest of Tlalocan}

\begin{flushright}
\parbox{0.8\textwidth}{
\emph{The human mind is inspired enough when it comes to inventing
horrors; it is when it tries to invent a Heaven that it shows itself
cloddish. \\
\hspace*{\fill}{\textperiodcentered \textperiodcentered \textperiodcentered \hspace*{0.2em} Evelyn Waugh} } }
\end{flushright}

\noindent
Conquest of Tlalocan is chess variant which is played on 24 x 24 board,
with bright red and cyan fields, and dark red and light green pieces.
Star colors are bright red and bright blue. In algebraic notation, columns
are enumerated from 'a' to 'x', and rows are enumerated from '1' to '24'.
A new piece is introduced, Shaman.

\clearpage % ..........................................................
% Shaman **************************************************************

\section*{Shaman}
\addcontentsline{toc}{section}{Shaman}

\noindent
\begin{wrapfigure}[11]{l}{0.4\textwidth}
\centering
\includegraphics[width=0.4\textwidth, keepaspectratio=true]{pieces/14_shaman.png}
\caption{Shaman}
\label{fig:14_shaman}
\end{wrapfigure}
Shaman moves like sort-of cross between Knight and long-jump Unicorn,
where one figure provides step-fields, and the other capture-fields.

For light Shaman, step-fields are provided by the Knight, while capture-fields
are provided by long-range Unicorn. For dark Shaman, it's the opposite.

Shaman can activate both Wave and Pyramid on its' capture-fields, while only
Wave can be activated on step-fields. In both cases, momentum given is 1.

Alternative move for Shaman is a trance-journey.

Shaman symbol in algebraic notation is 'H', to avoid confusion with Serpent.

% \vspace*{0.05\textheight}
\noindent
\begin{wrapfigure}{l}{0.4\textwidth}
\centering
\includegraphics[width=0.4\textwidth, keepaspectratio=true]{pieces/star/18_conquest_of_tlalocan.png}
\caption{Star}
\label{fig:star/18_conquest_of_tlalocan}
\end{wrapfigure}
Star colors in this variant are presented on the left.


% :: TODO ::
\clearpage % ..........................................................
% \vspace*{0.25\textheight}
\huge{TODO :: step-ply, capture-ply}
\normalsize{ -move ? }
% :: TODO ::


\clearpage % ..........................................................
% Movement ------------------------------------------------------------

\subsection*{Movement}
\addcontentsline{toc}{subsection}{Movement}

\noindent
\begin{figure}[!h]
% \begin{figure}[!t]
\includegraphics[width=1.0\textwidth, keepaspectratio=true]{examples/18_cot/scn_cot_01_shaman_movement.png}
\caption{Shaman movement}
\label{fig:scn_cot_01_shaman_movement}
% \centering
\end{figure}

Here, step-fields are marked green, while capture-fields are marked blue.
Note, movement of Shaman does not depend on color of field on which it
stands.

\clearpage % ..........................................................

\subsubsection*{Activating Pyramid, Wave}
\addcontentsline{toc}{subsubsection}{Activating Pyramid, Wave}

% \vspace*{0.05\textheight}
\noindent
\begin{wrapfigure}{l}{0.4\textwidth}
\centering
\includegraphics[width=0.375\textwidth, keepaspectratio=true]{examples/18_cot/scn_cot_02_activating_passives.png}
\caption{Activating Pyramid, Wave}
\label{fig:scn_cot_02_activating_passives}
\end{wrapfigure}
Here, Pyramids can be activated only on capture-fields (blue), while Waves
can also be activated on step-fields (green).

Since Pyramid 1 is standing on a step-field, it is the only one which can't
be activated (red).

In any case, momentum of 1 would be transferred to activated piece.

% ------------------------------------------------------------ Movement
\clearpage % ..........................................................
% Trance-journey ------------------------------------------------------

\subsection*{Trance-journey}
\addcontentsline{toc}{subsection}{Trance-journey}

...

% ------------------------------------------------------ Trance-journey
% ************************************************************** Shaman
\clearpage % ..........................................................

\section*{Promotion}
\addcontentsline{toc}{section}{Promotion}

Promotion is non enforced, delayed variety, i.e. it's the same as in
\hyperref[sec:Age of Aquarius/Promotion]{previous chess variant}, Age of Aquarius.

Promotion in this variant is polygamous, more than one Queen in the same color
can be present on chessboard at any given time.

\clearpage % ..........................................................

\section*{En passant}
\addcontentsline{toc}{section}{En passant}

\noindent
\begin{wrapfigure}{l}{0.4\textwidth}
\centering
\includegraphics[width=0.125\textwidth, keepaspectratio=true]{en_passants/18_conquest_of_tlalocan_en_passant.png}
\caption{En passant}
\label{fig:18_conquest_of_tlalocan_en_passant}
\end{wrapfigure}
Rush and en passant are identical to those in Classic Chess, only difference
is that Pawn can now move longer on initial turn, up to 9 fields in this
variant.

\clearpage % ..........................................................

\section*{Castling}
\addcontentsline{toc}{section}{Castling}

Castling is the same as in Classical Chess, only difference is that King can move between 2 and 8 fields across.
All other constraints from Classical Chess still applies.

\noindent
\begin{figure}[!h]
% \begin{figure}[!t]
\includegraphics[width=1.0\textwidth, keepaspectratio=true]{castlings/18_cot/conquest_of_tlalocan_castling.png}
\caption{Castling}
\label{fig:conquest_of_tlalocan_castling}
% \centering
\end{figure}

In example above, all valid King's castling moves are numbered.

\noindent
\begin{figure}[!h]
% \begin{figure}[!t]
\includegraphics[width=1.0\textwidth, keepaspectratio=true]{castlings/18_cot/conquest_of_tlalocan_castling_right_08.png}
\caption{Castling long right}
\label{fig:conquest_of_tlalocan_castling_right_08}
% \centering
\end{figure}

In this example King was castling long to the right. Initial King's position is marked with "K".
After castling is finished, right Rook ends up at field immediately left to the King.

\clearpage % ..........................................................

\section*{Initial setup}
\addcontentsline{toc}{section}{Initial setup}

Compared to initial setup of Tamoanchan Revisited, Shaman is inserted between Unicorn and Pyramid
symmetrically, on both sides of chessboard. This can be seen in the image below:

\noindent
% \begin{figure}[t]
\begin{figure}[h]
\includegraphics[width=1.0\textwidth, keepaspectratio=true]{boards/18_conquest_of_tlalocan.png}
\caption{Conquest of Tlalocan board}
\label{fig:18_conquest_of_tlalocan}
% \centering
\end{figure}

\clearpage % ..........................................................
% ======================================== Conquest of Tlalocan chapter
