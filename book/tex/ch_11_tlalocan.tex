
% Conquest of Tlalocan chapter ----------------------------------------
\chapter*{Conquest of Tlalocan}
\addcontentsline{toc}{chapter}{Conquest of Tlalocan}

\begin{flushright}
\parbox{0.8\textwidth}{
\emph{The human mind is inspired enough when it comes to inventing
horrors; it is when it tries to invent a Heaven that it shows itself
cloddish. \\
\hspace*{\fill}{\textperiodcentered \textperiodcentered \textperiodcentered \hspace*{0.2em} Evelyn Waugh} } }
\end{flushright}

\noindent
Conquest of Tlalocan is chess variant which is played on 24 x 24 board,
with bright cyan and red fields and light green and dark red pieces.
In algebraic notation, columns are enumerated from 'a' to 'x', and rows
are enumerated from '1' to '24'. A new piece is introduced, Shaman.

\clearpage

\section*{Shaman}
\addcontentsline{toc}{section}{Shaman}

\noindent
\begin{wrapfigure}{l}{0.4\textwidth}
\includegraphics[width=0.4\textwidth, keepaspectratio=true]{pieces/14_shaman.png}
\caption{Shaman}
\label{fig:shaman}
% % \centering
\end{wrapfigure}

\clearpage

\section*{Initial setup}
\addcontentsline{toc}{section}{Initial setup}

Initial setup for Light player is (mirrored for Dark one):
\texttt{PPPPPPPPPPPPPPPPPPPPPPPP \\
        TRGAHUWCSNBQKBNSCWUHAGRT}, \\
or more conveniently, as seen in this image:

\noindent
% \begin{figure}[t]
\begin{figure}[h]
\includegraphics[width=1.0\textwidth, keepaspectratio=true]{boards/18_conquest_of_tlalocan.png}
\caption{Conquest of Tlalocan board}
\label{fig:conquest_of_tlalocan}
% \centering
\end{figure}

\clearpage
% ---------------------------------------- Conquest of Tlalocan chapter

