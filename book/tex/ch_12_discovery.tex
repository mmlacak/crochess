
% Copyright (c) 2015 - 2020 Mario Mlačak, mmlacak@gmail.com
% Licensed and published as Public Domain work.

% Discovery chapter ===================================================
\chapter*{Discovery}
\addcontentsline{toc}{chapter}{Discovery}

\begin{flushright}
\parbox{0.8\textwidth}{
\emph{I don’t believe in God but I’m very interested in her. \\
\hspace*{\fill}{\textperiodcentered \textperiodcentered \textperiodcentered \hspace*{0.2em} Arthur C. Clarke} } }
\end{flushright}

\noindent
Discovery is chess variant which is played on 24 x 24 board, with
light (pastel!) yellow and gray fields and darker gray and dark teal
pieces. Star colors are bright orange and dark violet. In algebraic
notation, columns are enumerated from 'a' to 'x', and rows are
enumerated from '1' to '24'. A new piece is introduced, Monolith.

\clearpage % ..........................................................
% Monolith ************************************************************

\section*{Monolith}
\addcontentsline{toc}{section}{Monolith}

\vspace*{-1.1\baselineskip}
\noindent
\begin{wrapfigure}[11]{l}{0.4\textwidth}
\centering
\includegraphics[width=0.4\textwidth, keepaspectratio=true]{pieces/15_monolith.png}
\caption{Monolith}
\label{fig:15_monolith}
\end{wrapfigure}
Monolith does not belong to any player, but can be moved by both of them.
Monolith cannot be captured, converted, nor activated.
Pawns cannot be promoted to Monolith.

Monolith is a teleportation device, much like moveable Star. Piece can
initiate teleportation either by touching a Monolith or a field at which
it stands.

Piece, if not Wave, then reappears on a chosen empty portal-field around
any Star or the other Monolith. Wave teleported from a Monolith can emerge
only from the other Monolith. Kings cannot be teleported.

Piece teleported from a Star, if not Wave, can reappear on a chosen empty
portal-field around the 2 Stars in opposite color, or around any Monolith.
Wave teleported from a Star can only emerge from the Star in the same color.

Monolith cannot capture piece, and thus cannot check, nor checkmate opponent's
King. Monolith cannot activate Wave, nor any other piece.

Monolith moves similar to Knight, but can perform 3 steps in a single ply,
by alternating between left and right steps. All step-fields in a ply must
be empty.

Alternative move for Monolith is syzygy.

In algebraic notation, symbol for Monolith is 'M'.

% \clearpage % ..........................................................

% \vspace*{0.05\textheight}
\noindent
\begin{wrapfigure}[2]{l}{0.4\textwidth}
\centering
\includegraphics[width=0.4\textwidth, keepaspectratio=true]{pieces/bishop/20_discovery.png}
\caption{Bishop}
\label{fig:bishop/20_discovery}
\end{wrapfigure}
Piece colors in this variant are presented on the left.

\vspace*{0.30\textheight}
\noindent
\begin{wrapfigure}[2]{l}{0.4\textwidth}
\centering
\includegraphics[width=0.4\textwidth, keepaspectratio=true]{pieces/star/20_discovery.png}
\caption{Star}
\label{fig:star/20_discovery}
\end{wrapfigure}
Star colors in this variant are presented on the left.

\clearpage % ..........................................................
% Movement ------------------------------------------------------------

\subsection*{Movement}
\addcontentsline{toc}{subsection}{Movement}

\vspace*{-0.3\baselineskip}
\noindent
\begin{wrapfigure}[4]{l}{0.35\textwidth}
\centering
\includegraphics[width=0.291666667\textwidth, keepaspectratio=true]{examples/20_d/scn_d_01_knight_steps.png}
\caption{Knight steps}
\label{fig:scn_d_01_knight_steps}
\end{wrapfigure}
Like in a \hyperref[fig:scn_cot_10_knight_directions]{Conquest of Tlalocan variant},
looking from Knight's position forward, one direction
would be to the left, and the other to the right.

% \vspace*{0.2\textheight}
\vspace*{6.1\baselineskip}
\noindent
\begin{wrapfigure}[9]{l}{0.35\textwidth}
\centering
\includegraphics[width=0.291666667\textwidth, keepaspectratio=true]{examples/20_d/scn_d_02_monolith_steps.png}
\caption{Monolith steps}
\label{fig:scn_d_02_monolith_steps}
\end{wrapfigure}
Here, all left (green) and right (blue) steps of Monolith are marked.

Monolith can freely choose any step-field as its' first step destination. On all
subsequent steps, Monolith has to alternate between left and right steps. Every
step direction can be chosen independently of any previous choice.

% \vspace*{6.1\baselineskip}
\noindent
\begin{wrapfigure}[9]{l}{0.35\textwidth}
\centering
\includegraphics[width=0.291666667\textwidth, keepaspectratio=true]{examples/20_d/scn_d_03_monolith_step_1.png}
\caption{Monolith first step}
\label{fig:scn_d_03_monolith_step_1}
\end{wrapfigure}
Like Knight, Monolith is not obstructed by any piece on unmarked (i.e. non-step)
field. Monolith cannot interact with other pieces on its' own. So, Monolith is
blocked by any piece, except Star or other Monolith, on its' step-field. In this
variant, Monolith is limited to 3 steps in its' ply.

\clearpage % ..........................................................

\noindent
\begin{figure}[!h]
% \begin{figure}[!t]
\includegraphics[width=1.0\textwidth, keepaspectratio=true]{examples/20_d/scn_d_04_monolith_step_2.png}
\caption{Monolith step 2}
\label{fig:scn_d_04_monolith_step_2}
% \centering
\end{figure}

Starting field is marked S. Right step was chosen as a first one, so next step
has to be to the left. Here, Monolith is obstructed by Wave on a step-field.
This is so regardless if player moving Monolith is light or dark.

\clearpage % ..........................................................

\noindent
\begin{figure}[!h]
% \begin{figure}[!t]
\includegraphics[width=1.0\textwidth, keepaspectratio=true]{examples/20_d/scn_d_05_monolith_step_3.png}
\caption{Monolith step 3}
\label{fig:scn_d_05_monolith_step_3}
% \centering
\end{figure}

Previous step was a left one, so next step needs to be one of 4 marked right
steps. This is also last step in a Monolith's ply, since it's third. Monolith
is not obstructed by Pawns on a non-step fields; nor by Bishop on a step-field
in alternate direction (here, left).

% ------------------------------------------------------------ Movement
\clearpage % ..........................................................
% Teleporting ---------------------------------------------------------

\subsection*{Teleporting}
\addcontentsline{toc}{subsection}{Teleporting}

\vspace*{-0.9\baselineskip}
\noindent
\begin{figure}[!h]
% \begin{figure}[!t]
\includegraphics[width=1.0\textwidth, keepaspectratio=true]{examples/20_d/scn_d_06_teleport_via_monolith.png}
\caption{Teleporting piece via Monolith}
\label{fig:scn_d_06_teleport_via_monolith}
% \centering
\end{figure}

Teleportation using Monoliths is similar to one using Stars in \hyperref[fig:scn_n_02_teleport_init]{previous variant, Nineteen}.
Pieces, if not Waves, teleporting from Monolith can reappear near any Star or the other Monolith.
All momentum carried is lost. Again, Kings cannot teleport.
Here, all empty portal-fields where Bishop can be teleported to are enumerated.

\clearpage % ..........................................................

\noindent
\begin{figure}[!h]
% \begin{figure}[!t]
\includegraphics[width=1.0\textwidth, keepaspectratio=true]{examples/20_d/scn_d_07_teleport_via_star.png}
\caption{Teleporting piece via Star}
\label{fig:scn_d_07_teleport_via_star}
% \centering
\end{figure}

All pieces, except Waves, teleporting from a Star can reappear on a empty portal-field
near Stars in opposite color, or near any Monolith.
Here, all empty portal-fields where Bishop can be teleported to are enumerated.

\clearpage % ..........................................................
% Teleporting Wave ....................................................

\subsubsection*{Wave}
\addcontentsline{toc}{subsubsection}{Wave}

\vspace*{-1.2\baselineskip}
\noindent
\begin{figure}[!h]
% \begin{figure}[!t]
\includegraphics[width=1.0\textwidth, keepaspectratio=true]{examples/20_d/scn_d_08_teleport_wave_via_star.png}
\caption{Teleporting Wave via Star}
\label{fig:scn_d_08_teleport_wave_via_star}
% \centering
\end{figure}

Teleporting Wave using Star is the same as in \hyperref[fig:scn_n_04_teleport_move_3]{previous variant, Nineteen}.
Wave teleported from a Star emerges from the other Star in the same color,
and continues to move from position of a destination Star in the same
direction as before teleportation. Teleported Wave retains momentum carried.
Here, light Wave could activate own Bishop after teleporting with 2 momentum.

\clearpage % ..........................................................

\noindent
\begin{figure}[!h]
% \begin{figure}[!t]
\includegraphics[width=1.0\textwidth, keepaspectratio=true]{examples/20_d/scn_d_09_teleport_wave_via_monolith.png}
\caption{Teleporting Wave via Monolith}
\label{fig:scn_d_09_teleport_wave_via_monolith}
% \centering
\end{figure}

Wave teleported from a Monolith emerges from the other Monolith, and continues
movement from position of a destination Monolith in the same direction as before
teleportation, while retaining momentum carried into teleportation.
Here, light Wave could activate own Bishop after teleporting with 2 momentum.

\clearpage % ..........................................................

\noindent
\begin{figure}[!h]
% \begin{figure}[!t]
\includegraphics[width=1.0\textwidth, keepaspectratio=true]{examples/20_d/scn_d_10_teleported_wave_blocked.png}
\caption{Teleported Wave blocked}
\label{fig:scn_d_10_teleported_wave_blocked}
% \centering
\end{figure}

In case where all step-fields of a teleported Wave are blocked, it's removed from
chessboard as if captured by the opponent, described in
\hyperref[fig:scn_n_06_teleport_wave_blocked]{previous variant, Nineteen}.

The same applies to all other (non-Wave) pieces. If all portal-fields where
teleported piece could reappear are occupied, piece is removed from chessboard.

\clearpage % ..........................................................

\noindent
\begin{figure}[!h]
% \begin{figure}[!t]
\includegraphics[width=1.0\textwidth, keepaspectratio=true]{examples/20_d/scn_d_11_wave_teleported_off_board.png}
\caption{Wave teleported off-board}
\label{fig:scn_d_11_wave_teleported_off_board}
% \centering
\end{figure}

Teleported Wave with all of its' step-fields located off-board is also removed from
chessboard as if captured by the opponent.

\clearpage % ..........................................................

\noindent
\begin{figure}[!h]
% \begin{figure}[!t]
\includegraphics[width=1.0\textwidth, keepaspectratio=true]{examples/20_d/scn_d_12_wave_teleport_on_and_off_board.png}
\caption{Teleporting Wave on- and off-board}
\label{fig:scn_d_12_wave_teleport_on_and_off_board}
% \centering
\end{figure}

Before and after teleportation, Wave can step outside of a board.
This is legal as long as Wave ends its' ply on a chessboard.

Here, light Wave could activate own Bishop after teleportation with 3 momentum.

\clearpage % ..........................................................

\subsubsection*{Wave cascade}
\addcontentsline{toc}{subsubsection}{Wave cascade}

\vspace*{-0.9\baselineskip}
\noindent
\begin{figure}[!h]
% \begin{figure}[!t]
\includegraphics[width=1.0\textwidth, keepaspectratio=true]{examples/20_d/scn_d_13_teleporting_wave_cascade.png}
\caption{Cascading teleportations}
\label{fig:scn_d_13_teleporting_wave_cascade}
% \centering
\end{figure}

Teleportation cascade refers to Wave being teleported at least twice in the same ply,
other pieces can't cascade. Unlike in a previous variants, thanks to Monolith,
teleportation cascade is now useful in granting access to otherwise unreachable places.
Here, light Wave can activate own Bishop only after second teleportation.

% .................................................... Teleporting Wave
\clearpage % ..........................................................

\subsubsection*{Monolith}
\addcontentsline{toc}{subsubsection}{Monolith}

\vspace*{-0.9\baselineskip} % just to have it aligned with previous example
\noindent
\begin{figure}[!h]
% \begin{figure}[!t]
\includegraphics[width=1.0\textwidth, keepaspectratio=true]{examples/20_d/scn_d_14_teleporting_monolith_via_star.png}
\caption{Teleporting Monolith via Star}
\label{fig:scn_d_14_teleporting_monolith_via_star}
% \centering
\end{figure}

Monolith can also be teleported. If teleported from a Star, Monolith can reappear
on an empty portal-field near Stars in opposite color, or near the other Monolith.

\clearpage % ..........................................................

\noindent
\begin{figure}[!h]
% \begin{figure}[!t]
\includegraphics[width=1.0\textwidth, keepaspectratio=true]{examples/20_d/scn_d_15_teleporting_monolith_via_monolith.png}
\caption{Teleporting Monolith via Monolith}
\label{fig:scn_d_15_teleporting_monolith_via_monolith}
% \centering
\end{figure}

Monolith teleported from the other Monolith can reappear on an empty portal-field
near any Star.

\clearpage % ..........................................................

\subsubsection*{Trance-journey interaction}
\addcontentsline{toc}{subsubsection}{Trance-journey interaction}

\vspace*{-0.9\baselineskip}
\noindent
\begin{figure}[!h]
% \begin{figure}[!t]
\includegraphics[width=1.0\textwidth, keepaspectratio=true]{examples/20_d/scn_d_16_monolith_shaman_interaction.png}
\caption{Trance-journey interaction}
\label{fig:scn_d_16_monolith_shaman_interaction}
% \centering
\end{figure}

Like with Stars (and Kings) in \hyperref[fig:scn_cot_18_light_light_shaman_interaction_start]{the previous variant},
entranced Shamans cannot interact with Monolith, but can continue to move past it. This is so regardless of colors
of both entrancing (S) and entranced (T) Shamans. Here, entranced light Shaman can displace dark Knight, which it
can reach after passing all non-interacting pieces.

% --------------------------------------------------------- Teleporting
\clearpage % ..........................................................
% Syzygy --------------------------------------------------------------

\subsection*{Syzygy}
\addcontentsline{toc}{subsection}{Syzygy}
...

% -------------------------------------------------------------- Syzygy
% ************************************************************ Monolith
\clearpage % ..........................................................

\section*{Promotion}
\addcontentsline{toc}{section}{Promotion}

Promotion is non enforced, delayed variety, i.e. it's the same as in
\hyperref[sec:Age of Aquarius/Promotion]{previous chess variant}, Age of Aquarius.

Promotion in this variant is polygamous, more than one Queen in the same color
can be present on chessboard at any given time.

Again, Pawn cannot be promoted to Monolith.

\clearpage % ..........................................................

\section*{En passant}
\addcontentsline{toc}{section}{En passant}

\noindent
\begin{wrapfigure}{l}{0.4\textwidth}
\centering
\includegraphics[width=0.125\textwidth, keepaspectratio=true]{en_passants/20_discovery_en_passant.png}
\caption{En passant}
\label{fig:20_discovery_en_passant}
\end{wrapfigure}
Rush and en passant are identical to those in Classic Chess, only difference
is that Pawn can now move longer on initial turn, up to 10 fields in this
variant.

\clearpage % ..........................................................

\section*{Castling}
\addcontentsline{toc}{section}{Castling}

Castling is the same as in Classical Chess, only difference is that King can move between 2 and 9 fields across.
All other constraints from Classical Chess still applies.

\noindent
\begin{figure}[!h]
% \begin{figure}[!t]
\includegraphics[width=1.0\textwidth, keepaspectratio=true]{castlings/20_d/discovery_castling.png}
\caption{Castling}
\label{fig:discovery_castling}
% \centering
\end{figure}

In example above, all valid King's castling moves are numbered.

\noindent
\begin{figure}[!h]
% \begin{figure}[!t]
\includegraphics[width=1.0\textwidth, keepaspectratio=true]{castlings/20_d/discovery_castling_left_07.png}
\caption{Castling long left}
\label{fig:discovery_castling_left_07}
% \centering
\end{figure}

In this example King was castling long to the left. Initial King's position is marked with "K".
After castling is finished, left Rook ends up at field immediately right to the King.

\clearpage % ..........................................................

\section*{Initial setup}
\addcontentsline{toc}{section}{Initial setup}

Compared to initial setup of Conquest of Tlalocan, just 2 Monoliths are placed in to the open,
symetrically, on both sides of chessboard. This can be seen in the image below:

\noindent
% \begin{figure}[t]
\begin{figure}[h]
\includegraphics[width=1.0\textwidth, keepaspectratio=true]{boards/20_discovery.png}
\caption{Discovery board}
\label{fig:20_discovery}
% \centering
\end{figure}

\clearpage % ..........................................................
% =================================================== Discovery chapter
