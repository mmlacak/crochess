
% Copyright (c) 2015 - 2020 Mario Mlačak, mmlacak@gmail.com
% Public Domain work, under CC0 1.0 Universal Public Domain Dedication. See LICENSING, COPYING files for details.

% Mayan Ascendancy chapter ============================================
\chapter*{Mayan Ascendancy}
\addcontentsline{toc}{chapter}{Mayan Ascendancy}
\label{ch:Mayan Ascendancy}

\begin{flushright}
\parbox{0.8\textwidth}{
\emph{The world has achieved brilliance without wisdom, power without
conscience. Our is a world of nuclear giants and ethical infants.\newline
\hspace*{\fill}{\textasciitilde{} Omar Nelson Bradley} } }
\end{flushright}

\noindent
Mayan Ascendancy is chess variant which is played on 12~$\times$~12 board
with yellow and blue fields and with dark yellow and dark blue pieces.
A new piece is introduced, Pyramid.

\clearpage % ..........................................................
% Pyramid *************************************************************

\section*{Pyramid}
\addcontentsline{toc}{section}{Pyramid}
\label{sec:Mayan Ascendancy/Pyramid}

\noindent
\begin{wrapfigure}[12]{l}{0.4\textwidth}
\centering
\includegraphics[width=0.4\textwidth, keepaspectratio=true]{pieces/08_pyramid.png}
\caption{Pyramid}
\label{fig:08_pyramid}
\end{wrapfigure}
Pyramid is passive piece, meaning it can't move on its own, it has to be
activated first. This is done by capturing a field at which Pyramid stands
with own other piece and then move Pyramid further.

Once activated, Pyramid moves similar to Rook, only real difference is that
it can move for only so many fields as piece activating it has moved, i.e.
for at most as momentum received.

\subsection*{Momentum}
\addcontentsline{toc}{subsection}{Momentum}
\label{sec:Mayan Ascendancy/Pyramid/Momentum}

Momentum is count of fields traveled over by a piece. Pyramid receives
momentum from piece which activates it. Momentum is spent by Pyramid when
moving, one for each field traveled. So Pyramid can't move for more
fields than received momentum, i.e. for more than activating piece has
traveled. Momentum can't be saved for later, it is wasted when Pyramid
moves for less than received momentum.

\subsubsection*{Non-negative}
\addcontentsline{toc}{subsubsection}{Non-negative}
\label{sec:Mayan Ascendancy/Pyramid/Momentum/Non-negative}

Piece has momentum if it's equal to or greater than 1. Piece has no momentum
if it's 0. In all cases, momentum cannot become negative, it's not possible
to "borrow" momentum from activating piece to activated piece (Pyramid).

% \clearpage % ..........................................................

\subsubsection*{Fields counting}
\addcontentsline{toc}{subsubsection}{Fields counting}
\label{sec:Mayan Ascendancy/Pyramid/Momentum/Fields counting}

Momentum accumulated (or spent) is always count of traveled fields, regardless
if special move has a step crossing two fields at once. For instance,
\hyperref[fig:04_croatian_ties_en_passant]{rushing Pawn's} first step is always
across 2 fields; still, both fields are counted towards momentum, since rushing
Pawn can be blocked by a piece on both fields.

\subsection*{Pyramid (cont.)}
\addcontentsline{toc}{subsection}{Pyramid (cont.)}
\label{sec:Mayan Ascendancy/Pyramid/Pyramid (cont.)}

Pyramid can't check opponent's King, and consequently can't contribute to
checkmate. Pyramid can capture all the other opponent's pieces after it has
been activated, even if it has no remaining momentum, i.e. can't move any
further.

Pyramid can also promote own Pawns on
\hyperref[sec:Definitions/Chessboard sides, navigation]{opponent's side of the board}.
It can also convert any opponent's piece, except King, on
\hyperref[sec:Definitions/Chessboard sides, navigation]{own side of the board}.
To do either of these things, Pyramid does not have to have any remaining
momentum, it's enough if piece in question is within reach.

Pyramid can also activate other Pyramid, and transfer remaining momentum to it.
There has to be remaining momentum, it must be greater than 0 for cascading
to be permitted. Pyramid cannot activate any other piece, neither own nor
opponent's.

\clearpage % ..........................................................
% Activation ----------------------------------------------------------

\subsection*{Activation}
\addcontentsline{toc}{subsection}{Activation}
\label{sec:Mayan Ascendancy/Pyramid/Activation}

\noindent
\begin{figure}[!h]
\includegraphics[width=1.0\textwidth, keepaspectratio=true]{examples/06_ma/scn_ma_01_pyramid_activation_init.png}
\caption{Pyramid activation}
\label{fig:scn_ma_01_pyramid_activation_init}
\end{figure}

Here Pegasus is about to capture field on which Pyramid stands. Note, only
step-fields are counted towards momentum. After activation Pyramid would be
limited to move at most 4 fields across, i.e. at most the momentum it received
from Pegasus.

\clearpage % ..........................................................

\noindent
\begin{figure}[!h]
\includegraphics[width=1.0\textwidth, keepaspectratio=true]{examples/06_ma/scn_ma_02_pyramid_activated.png}
\caption{Pyramid activated}
\label{fig:scn_ma_02_pyramid_activated}
\end{figure}

Above, arrows show all possible moves by Pyramid. Just like Rook, Pyramid has to
stop before own Bishop. Pyramid can capture opponent's Knight, but can't move any
further after capture. Pyramid can also capture opponent's Bishop, despite being
barely reachable.

% ---------------------------------------------------------- Activation
\clearpage % ..........................................................
% Promotion -----------------------------------------------------------

\subsection*{Promotion}
\addcontentsline{toc}{subsection}{Promotion}
\label{sec:Mayan Ascendancy/Pyramid/Promotion}

Pyramid can promote own Pawns, but only on opponent's side of the board.
Promotion is done by activating Pyramid which then marks Pawn for promotion
by touching either Pawn or field at which it stands. Pyramid then leaves
board as if captured by the opponent, and Pawn is replaced by desired piece,
for instance Queen.

Both Pyramid and Pawn in question has to reside on opponent's side of the
board before promotion can take place. Piece which activates Pyramid need
not to be on opponent's side of the board.

Piece which Pawn can be promoted to is from the set of all starting pieces,
except King. This promoting-to piece is not limited to pieces already being
captured.

\clearpage % ..........................................................

\noindent
\begin{figure}[!h]
\includegraphics[width=1.0\textwidth, keepaspectratio=true]{examples/06_ma/scn_ma_05_promo_init.png}
\caption{Promotion start}
\label{fig:scn_ma_05_promo_init}
\end{figure}

Here, Pegasus is accumulating momentum while travelling over step-fields. After
activation Pyramid would be limited to move at most 4 fields across, i.e. at most
the momentum it received from Pegasus.

\clearpage % ..........................................................

\noindent
\begin{figure}[!h]
\includegraphics[width=1.0\textwidth, keepaspectratio=true]{examples/06_ma/scn_ma_06_promo_pyramid_activated.png}
\caption{Promotion, Pyramid activated}
\label{fig:scn_ma_06_promo_pyramid_activated}
\end{figure}

Above, Pegasus captured field at which Pyramid was situated, arrows now show
all possible moves by Pyramid. Pyramid can't promote Pawn 2, as it is still
located on own half of the chessboard. Just as Rook, Pyramid can't advance
past Pawn 2. Only full movement to the right leads to promotion of Pawn 1,
shown in blue.

\clearpage % ..........................................................

\noindent
\begin{figure}[!h]
\includegraphics[width=1.0\textwidth, keepaspectratio=true]{examples/06_ma/scn_ma_07_promo_end.png}
\caption{Promotion end}
\label{fig:scn_ma_07_promo_end}
\end{figure}

Now that Pyramid has reached Pawn 1, it is removed from the board and piece of
choice, in this instance Queen, replaces Pawn. Just as with ordinary promotion,
this can take place regardless of which pieces has been captured, e.g. even if
own Queen is still present on chessboard.

% ----------------------------------------------------------- Promotion
\clearpage % ..........................................................
% Conversion ----------------------------------------------------------

\subsection*{Conversion}
\addcontentsline{toc}{subsection}{Conversion}
\label{sec:Mayan Ascendancy/Pyramid/Conversion}

Pyramid can convert opponent's pieces, except King, but only on own side of
the board. Conversion is done by activating Pyramid which then marks opponent's
piece for conversion by touching either piece or field at which it stands. Now
Pyramid leaves the board as if captured by the opponent, and opponent's piece
is replaced by own piece of the same type.

Both Pyramid and opponent's piece has to reside on own side of the board before
conversion can take place. Piece which activates Pyramid need not to be on own
side of the board. Conversion is not limited to pieces which has been captured.

Note that Pyramid might just as well capture opponent's piece. Differences are
what leaves chessboard, and what remains on captured field. Capture itself with
Pyramid is in no way different than that with Rook. In either case, converting
or capturing, it is enough if Pyramid can reach opponent's piece, i.e. has
enough momentum.

\clearpage % ..........................................................

\noindent
\begin{figure}[!h]
\includegraphics[width=1.0\textwidth, keepaspectratio=true]{examples/06_ma/scn_ma_08_conversion_init.png}
\caption{Conversion start}
\label{fig:scn_ma_08_conversion_init}
\end{figure}

In example above, Bishop is travelling over 4 step-fields to reach for Pyramid,
and so that is momentum Pyramid will receive when activated by the Bishop.
This is also limit how far Pyramid could move after being activated.

\clearpage % ..........................................................

\noindent
\begin{figure}[!h]
\includegraphics[width=1.0\textwidth, keepaspectratio=true]{examples/06_ma/scn_ma_09_conversion_pyramid_activated.png}
\caption{Conversion, Pyramid activated}
\label{fig:scn_ma_09_conversion_pyramid_activated}
\end{figure}

Above, Bishop captured field at which Pyramid was situated, arrows now show all
possible moves by Pyramid. Pyramid can't convert opponent's Bishop, as it is still
located on opponent's side of chessboard. Pyramid could capture opponent's Bishop.
Again, just like Rook, Pyramid can't advance past opponent's Bishop. Only full
movement to the right leads to conversion of opponent's Rook, shown in blue.

\clearpage % ..........................................................

\noindent
\begin{figure}[!h]
\includegraphics[width=1.0\textwidth, keepaspectratio=true]{examples/06_ma/scn_ma_10_conversion_end.png}
\caption{Conversion end}
\label{fig:scn_ma_10_conversion_end}
\end{figure}

Now that Pyramid has reached opponent's Rook, it is removed from the board and
own Rook replaces opponent's Rook. This conversion can still take place, regardless
if any light Rook has been captured or not, i.e. even with both light Rooks still
present on chessboard. Capturing opponent's Rook would simply leave Pyramid in
place of it.

\clearpage % ..........................................................

\subsubsection*{Converting Rooks}
\addcontentsline{toc}{subsubsection}{Converting Rooks}
\label{sec:Mayan Ascendancy/Pyramid/Conversion/Converting Rooks}

\vspace*{-1.1\baselineskip}
\noindent
\begin{figure}[!h]
\includegraphics[width=1.0\textwidth, keepaspectratio=true]{examples/06_ma/scn_ma_11_converting_rook_init.png}
\caption{Converting Rook start}
\label{fig:scn_ma_11_converting_rook_init}
\end{figure}

Converting opponent's Rook does not grant it an option to castle, even if it's
converted at initial position of own Rook, and hasn't moved yet.

Here, dark Rook moved into initial position of light Rook on previous move
(grey arrows); light player is about to convert dark Rook.

\clearpage % ..........................................................

\noindent
\begin{figure}[!h]
\includegraphics[width=1.0\textwidth, keepaspectratio=true]{examples/06_ma/scn_ma_12_converting_rook_end.png}
\caption{Converting Rook end}
\label{fig:scn_ma_12_converting_rook_end}
\end{figure}

Here, light Queen moved out of the way after conversion. Dark Rook has been
converted at light Rook's initial position, and hasn't been moved; still,
light King can't castle with converted Rook.

\clearpage % ..........................................................

\subsubsection*{Converting Pawns}
\addcontentsline{toc}{subsubsection}{Converting Pawns}
\label{sec:Mayan Ascendancy/Pyramid/Conversion/Converting Pawns}

\vspace*{-1.1\baselineskip}
\noindent
\begin{figure}[!h]
\includegraphics[width=1.0\textwidth, keepaspectratio=true]{examples/06_ma/scn_ma_13_converting_pawn_init.png}
\caption{Converting Pawn start}
\label{fig:scn_ma_13_converting_pawn_init}
\end{figure}

Converting opponent's Pawn does not grant it an option to rush, even if it's
converted at initial position of own Pawn, and hasn't moved yet.

Here, dark Pawn moved into initial position of light Pawn on previous move
(grey arrow); light player is about to convert dark Pawn.

\clearpage % ..........................................................

\noindent
\begin{figure}[!h]
\includegraphics[width=1.0\textwidth, keepaspectratio=true]{examples/06_ma/scn_ma_14_converting_pawn_end.png}
\caption{Converting Pawn end}
\label{fig:scn_ma_14_converting_pawn_end}
\end{figure}

Here, dark Pawn has been converted at light Pawn's initial position, and hasn't
been moved; still, converted Pawn cannot rush (move forward for two or more fields).

% ---------------------------------------------------------- Conversion
\clearpage % ..........................................................
% Cascading -----------------------------------------------------------

\subsection*{Cascading}
\addcontentsline{toc}{subsection}{Cascading}
\label{sec:Mayan Ascendancy/Pyramid/Cascading}

\noindent
\begin{figure}[!h]
\includegraphics[width=1.0\textwidth, keepaspectratio=true]{examples/06_ma/scn_ma_15_cascading_init.png}
\caption{Cascading start}
\label{fig:scn_ma_15_cascading_init}
\end{figure}

Once activated, Pyramid can also activate another Pyramid. To do so, activated
Pyramid has to have at least 1 remaining momentum to transfer it to another
Pyramid. If all momentum received was spent moving, Pyramid cannot cascade, i.e.
cannot activate another Pyramid.

\clearpage % ..........................................................

\noindent
\begin{figure}[!h]
\includegraphics[width=1.0\textwidth, keepaspectratio=true]{examples/06_ma/scn_ma_16_cascading_pyramid_1_activated.png}
\caption{Cascading, 1st Pyramid activated}
\label{fig:scn_ma_16_cascading_pyramid_1_activated}
\end{figure}

Pyramid 1 has been activated by Queen and received momentum of 5, arrows now show
its all possible moves. Note, Pyramid 3 can't be activated, it's on the very end
of fields reachable by Pyramid 1. Note also that Pyramid 1 can't activate, nor move
past light Bishop on the left.

\clearpage % ..........................................................

\noindent
\begin{figure}[!h]
\includegraphics[width=1.0\textwidth, keepaspectratio=true]{examples/06_ma/scn_ma_17_cascading_pyramid_2_activated.png}
\caption{Cascading, 2nd Pyramid activated}
\label{fig:scn_ma_17_cascading_pyramid_2_activated}
\end{figure}

Pyramid 2 has been activated by Pyramid 1 and in the process received momentum of 2,
arrows now show all possible moves by Pyramid 2.

\clearpage % ..........................................................

\noindent
\begin{figure}[!h]
\includegraphics[width=1.0\textwidth, keepaspectratio=true]{examples/06_ma/scn_ma_18_cascading_end.png}
\caption{Cascading end}
\label{fig:scn_ma_18_cascading_end}
\end{figure}

Pyramid 2 has finished its movement, and so it ends light player's complete move,
which consisted of 3 plies, i.e. 3 pieces has been moved.

% ----------------------------------------------------------- Cascading
\clearpage % ..........................................................

\subsection*{Against King}
\addcontentsline{toc}{subsection}{Against King}
\label{sec:Mayan Ascendancy/Pyramid/Against King}

Pyramid can't check opponent's King, meaning that King is not under check even if
Pyramid could capture any other piece on the same field.

\noindent
\begin{figure}[!h]
\includegraphics[width=1.0\textwidth, keepaspectratio=true]{examples/06_ma/scn_ma_19_pyramid_vs_king.png}
\caption{Pyramid vs. King}
\label{fig:scn_ma_19_pyramid_vs_king}
\end{figure}

Above, King does not have to move/defend, as it is not under check from Pyramid.

\noindent
\begin{figure}[!h]
\includegraphics[width=1.0\textwidth, keepaspectratio=true]{examples/06_ma/scn_ma_20_pyramid_vs_bishop.png}
\caption{Pyramid vs. Bishop}
\label{fig:scn_ma_20_pyramid_vs_bishop}
\end{figure}

Bishop in the same place, however, could be captured without any hindrance.

\clearpage % ..........................................................

\subsection*{Activation by Pawn}
\addcontentsline{toc}{subsection}{Activation by Pawn}
\label{sec:Mayan Ascendancy/Pyramid/Activation by Pawn}

\noindent
\begin{figure}[!h]
\includegraphics[width=1.0\textwidth, keepaspectratio=true]{examples/06_ma/scn_ma_04_pyramid_activation_by_pawn.png}
\caption{Pyramid activation by Pawns}
\label{fig:scn_ma_04_pyramid_activation_by_pawn}
\end{figure}

Pawns can activate Pyramid on own capture-field giving it 1 momentum, see Pawn 1.
Pawns can't activate Pyramids on step-fields, and are blocked from moving further,
see Pawns 2 and 3.

% ************************************************************* Pyramid
\clearpage % ..........................................................

\section*{Rush, en passant}
\addcontentsline{toc}{section}{Rush, en passant}
\label{sec:Mayan Ascendancy/Rush, en passant}

\noindent
\begin{wrapfigure}{l}{0.4\textwidth}
\centering
\includegraphics[width=0.275\textwidth, keepaspectratio=true]{en_passants/06_mayan_ascendancy_en_passant.png}
\caption{En passant}
\label{fig:06_mayan_ascendancy_en_passant}
\end{wrapfigure}
Rush and en passant are identical to those in Classic Chess, only difference
is that Pawn can now move longer on initial turn, up to 4 fields in this
variant.

Again, converted opponent's Pawns cannot be rushed, even if converted on
initial positions of own Pawns.

\clearpage % ..........................................................

\section*{Castling}
\addcontentsline{toc}{section}{Castling}
\label{sec:Mayan Ascendancy/Castling}

Castling is the same as in Classical Chess, only difference is that King can move 2, 3 or 4 fields across.
All other constraints from Classical Chess still applies.

\noindent
\begin{figure}[!h]
\includegraphics[width=1.0\textwidth, keepaspectratio=true]{castlings/06_ma/mayan_ascendancy_castling_00.png}
\caption{Castling}
\label{fig:mayan_ascendancy_castling_00}
\end{figure}

In example above, all valid King's castling moves are numbered. After any castling, Rook
ends on a field next to King closer to center, i.e. closer to King's initial position.

\noindent
\begin{figure}[!h]
\includegraphics[width=1.0\textwidth, keepaspectratio=true]{castlings/06_ma/mayan_ascendancy_castling_right_11_04.png}
\caption{Castling long right}
\label{fig:mayan_ascendancy_castling_right_11_04}
\end{figure}

In this example King was castling long to the right. Initial King's position is marked with "K".
After castling is finished, right Rook ends up on the field immediately left to the King.

Again, converted opponent's Rooks cannot be castled, even if converted on initial
positions of own Rooks.

\clearpage % ..........................................................

\section*{Initial setup}
\addcontentsline{toc}{section}{Initial setup}
\label{sec:Mayan Ascendancy/Initial setup}

Compared to initial setup of Croatian Ties, Pyramid is inserted between Pegasus and Knight
symmetrically, on both sides of chessboard. This can be seen in the image below:

\noindent
\begin{figure}[h]
\includegraphics[width=1.0\textwidth, keepaspectratio=true]{boards/06_mayan_ascendancy.png}
\caption{Mayan Ascendancy board}
\label{fig:06_mayan_ascendancy}
\end{figure}

\clearpage % ..........................................................
% ============================================ Mayan Ascendancy chapter
