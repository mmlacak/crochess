
% Copyright (c) 2015 - 2020 Mario Mlačak, mmlacak@gmail.com
% Licensed and published as Public Domain work.

% One chapter =========================================================
\chapter*{One}
\addcontentsline{toc}{chapter}{One}

\begin{flushright}
\parbox{0.8\textwidth}{
\emph{God is not external to anyone, but is present with all things, though
they are ignorant that he is so. \\
\hspace*{\fill}{\textperiodcentered \textperiodcentered \textperiodcentered \hspace*{0.2em} Plotinus} } }
\end{flushright}

\noindent
One is chess variant which is played on 26 x 26 board, with white and
darker violet fields, and with light purple and fuchsia pieces. Star
colors are reversed colors of ordinary pieces, i.e. fuchsia and light
purple. In algebraic notation, columns are enumerated from 'a' to 'z',
and rows are enumerated from '1' to '26'. A new piece is introduced,
Starchild.

\clearpage % ..........................................................
% Starchild ***********************************************************

\section*{Starchild}
\addcontentsline{toc}{section}{Starchild}

\noindent
\begin{wrapfigure}[11]{l}{0.4\textwidth}
\centering
\includegraphics[width=0.4\textwidth, keepaspectratio=true]{pieces/16_starchild.png}
\caption{Starchild}
\label{fig:16_starchild}
\end{wrapfigure}
Starchild is supporting piece, it cannot capture any piece, cannot check or checkmate
opponent's King.

Starchild cannot be converted, but can be activated. If activated, it does not spend
momentum while moving. Starchild can activate any own piece, except King, and can
activate opponent's Starchilds. Starchild can also activate and move Stars, but not
Monoliths.

Starchild can take any piece, except Kings, Stars and Monoliths, for a non-interactive,
viewing-only trance-journey.

% \vspace*{0.15\textheight}
\noindent
\begin{wrapfigure}[10]{l}{0.4\textwidth}
\centering
\includegraphics[width=0.4\textwidth, keepaspectratio=true]{pieces/star/22_one.png}
\caption{Star}
\label{fig:star/22_one}
\end{wrapfigure}
Starchild can't teleport. Starchild moves from starting to empty destination field in
an instant, without ever stepping on any intermediate fields and without interacting
with any pieces on them.

Starchild can ressurect any captured piece.

In algebraic notation, symbol for Starchild is 'I'.

Star colors in this variant are presented above, on the left.

\clearpage % ..........................................................
% Movement ------------------------------------------------------------

\subsection*{Movement}
\addcontentsline{toc}{subsection}{Movement}

\vspace*{-1.1\baselineskip}
\noindent
\begin{figure}[!h]
% \begin{figure}[!t]
\includegraphics[width=1.0\textwidth, keepaspectratio=true]{examples/22_o/scn_o_01_starchild_movement.png}
\caption{Starchild movement}
\label{fig:scn_o_01_starchild_movement}
% \centering
\end{figure}

Starchild can move to any empty field in opposite color to the one it's located on.
Starchild is not hampered by any piece between starting and destination field.

Here, light Starchild in the middle is positioned at dark field, and so can access
any empty light field on chessboard, in a single move.

\clearpage % ..........................................................

\subsection*{Activating piece}
\addcontentsline{toc}{subsection}{Activating piece}

% \vspace*{-1.1\baselineskip}
\noindent
\begin{wrapfigure}{l}{0.4\textwidth}
\centering
\includegraphics[width=0.4\textwidth, keepaspectratio=true]{examples/22_o/scn_o_02_starchild_activating_fields.png}
\caption{Activating fields}
\label{fig:scn_o_02_starchild_activating_fields}
\end{wrapfigure}
Fields at which Starchild can activate a piece are all fields immediately surrounding
it horizontally, vertically and diagonally. They are the same as step-fields of a King.
Any piece activated by Starchild receives 1 momentum.

Here, opponent's Bishop and own King can't be activated, so only own Pegasus can be
activated with 1 momentum.

\clearpage % ..........................................................

[*] Syzygy (+?) ... --\textgreater ressurection [*]

[*] TODO :: Star moved \textgreater\textgreater Wave cannot pass through ... teleportation is mandatory (not optional!) [*]

[*] moved Star ---\textgreater Wave \textgreater\textgreater if no board field beyond other Star \textgreater\textgreater sacrified [*]
see \hyperref[fig:scn_d_11_wave_teleported_off_board]{see "Wave teleported off-board"}

% ------------------------------------------------------------ Movement
% *********************************************************** Starchild
\clearpage % ..........................................................

\section*{Promotion}
\addcontentsline{toc}{section}{Promotion}

Promotion is non enforced, delayed variety, i.e. it's the same as in
\hyperref[sec:Age of Aquarius/Promotion]{previous chess variant}, Age of Aquarius.

Additionaly, promotion in this variant is monogamous.
Only one Queen in the same color can be present on chessboard at any given time.

\clearpage % ..........................................................

\section*{Rush, en passant}
\addcontentsline{toc}{section}{Rush, en passant}

\vspace*{-1.2\baselineskip}
\noindent
\begin{figure}[!h]
% \begin{figure}[!t]
\includegraphics[width=1.0\textwidth, keepaspectratio=true]{en_passants/22_one_en_passant.png}
\caption{En passant}
\label{fig:22_one_en_passant}
% \centering
\end{figure}

Rush and en passant are identical to those in \hyperref[fig:14_hemera_s_dawn_en_passant]{Hemera's Dawn variant}.
Own Pawns can be rushed for up to 11 fields in this variant.

\clearpage % ..........................................................

\section*{Castling}
\addcontentsline{toc}{section}{Castling}

Castling is the same as in Classical Chess, only difference is that King can move between 2 and 10 fields across.
All other constraints from Classical Chess still applies.

\noindent
\begin{figure}[!h]
% \begin{figure}[!t]
\includegraphics[width=1.0\textwidth, keepaspectratio=true]{castlings/22_o/one_castling.png}
\caption{Castling}
\label{fig:one_castling}
% \centering
\end{figure}

In example above, all valid King's castling moves are numbered.

\noindent
\begin{figure}[!h]
% \begin{figure}[!t]
\includegraphics[width=1.0\textwidth, keepaspectratio=true]{castlings/22_o/one_castling_right_04.png}
\caption{Castling short right}
\label{fig:one_castling_right_04}
% \centering
\end{figure}

In this example King was castling short to the right. Initial King's position is marked with "K".
After castling is finished, right Rook ends up at field immediately left to the King.

\clearpage % ..........................................................

\section*{Initial setup}
\addcontentsline{toc}{section}{Initial setup}

Compared to initial setup of Discovery, Starchild is inserted between Unicorn and Shaman
symmetrically, on both sides of chessboard. This can be seen in the image below:

\noindent
% \begin{figure}[t]
\begin{figure}[h]
\includegraphics[width=1.0\textwidth, keepaspectratio=true]{boards/22_one.png}
\caption{One board}
\label{fig:22_one}
% \centering
\end{figure}

\clearpage % ..........................................................
% ========================================================= One chapter
