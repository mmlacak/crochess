
% Copyright (c) 2015 - 2020 Mario Mlačak, mmlacak@gmail.com
% Licensed and published as Public Domain work.

% One chapter =========================================================
\chapter*{One}
\addcontentsline{toc}{chapter}{One}

\begin{flushright}
\parbox{0.8\textwidth}{
\emph{God is not external to anyone, but is present with all things, though
they are ignorant that he is so. \\
\hspace*{\fill}{\textperiodcentered \textperiodcentered \textperiodcentered \hspace*{0.2em} Plotinus} } }
\end{flushright}

\noindent
One is chess variant which is played on 26 x 26 board, with white and
darker violet fields, and with light purple and fuchsia pieces. Star
colors are reversed colors of ordinary pieces, i.e. fuchsia and light
purple. In algebraic notation, columns are enumerated from 'a' to 'z',
and rows are enumerated from '1' to '26'. A new piece is introduced,
Starchild.

\clearpage % ..........................................................
% Starchild ***********************************************************

\section*{Starchild}
\addcontentsline{toc}{section}{Starchild}

% \vspace*{-1.5\baselineskip}
\noindent
\begin{wrapfigure}[11]{l}{0.4\textwidth}
\centering
\includegraphics[width=0.4\textwidth, keepaspectratio=true]{pieces/16_starchild.png}
\caption{Starchild}
\label{fig:16_starchild}
\end{wrapfigure}
Starchild cannot capture any piece, cannot check or checkmate opponent's King.
Starchild is celestial piece, it can participate in demoting-to-Pawn syzygy.
Starchild can be demoted to Pawn.

Starchild cannot be converted, but can be activated. If activated, it does not spend
momentum while moving. In addition to own Waves, Pyramids, Starchild can activate
opponent's Starchilds. Starchild can also activate and move Stars, but not Monoliths.

% \vspace*{0.15\textheight}
\noindent
\begin{wrapfigure}[11]{l}{0.4\textwidth}
\centering
\includegraphics[width=0.4\textwidth, keepaspectratio=true]{pieces/star/22_one.png}
\caption{Star}
\label{fig:star/22_one}
\end{wrapfigure}
Starchild can't teleport. Starchild moves from starting to destination field in opposite
color in one step, without interacting with any piece on chessboard.

Starchild can ressurect any captured piece, except Kings, Stars, Monoliths. Waves and
Starchilds can be ressurected without ressurecting Starchild being oblationed.
Starchild can take any piece, except Kings, Stars and Monoliths, for a non-interactive,
viewing-only trance-journey.

In algebraic notation, symbol for Starchild is 'I'.

\clearpage % ..........................................................
% Movement ------------------------------------------------------------

\subsection*{Movement}
\addcontentsline{toc}{subsection}{Movement}

\vspace*{-1.1\baselineskip}
\noindent
\begin{figure}[!h]
% \begin{figure}[!t]
\includegraphics[width=1.0\textwidth, keepaspectratio=true]{examples/22_o/scn_o_01_starchild_movement.png}
\caption{Starchild movement}
\label{fig:scn_o_01_starchild_movement}
% \centering
\end{figure}

Starchild can move to any empty field in opposite color to the one it's located on.
Starchild is not hampered by any piece between starting and destination field.

Here, light Starchild in the middle moved from its starting position in one step.
It is now positioned at dark field, and so can access any empty light field in a
single step.

\clearpage % ..........................................................

\subsubsection*{Activating Wave}
\addcontentsline{toc}{subsubsection}{Activating Wave}

\vspace*{-1.1\baselineskip}
\noindent
\begin{figure}[!h]
% \begin{figure}[!t]
\includegraphics[width=1.0\textwidth, keepaspectratio=true]{examples/22_o/scn_o_02_starchild_activating_own_piece_init.png}
\caption{Activating Wave}
\label{fig:scn_o_02_starchild_activating_own_piece_init}
% \centering
\end{figure}

Starchild can activate own Waves and Pyramids, and opponent's Starchilds on its step-fields,
with 1 momentum. Here, both light Waves and dark Starchild can be activated. Neither light
Rook nor any of other opponent's pieces can be activated; some of them are also on the same
color field as activating Starchild.

\clearpage % ..........................................................

\vspace*{-2.1\baselineskip}
\noindent
\begin{figure}[!h]
% \begin{figure}[!t]
\includegraphics[width=1.0\textwidth, keepaspectratio=true]{examples/22_o/scn_o_03_starchild_activating_own_piece_end.png}
\caption{Wave activated}
\label{fig:scn_o_03_starchild_activating_own_piece_end}
% \centering
\end{figure}

Activated Wave moves the same as Starchild, i.e. to any field in opposite color to its starting
position. There it can activate any own piece (except King), and opponent's Wave, with 1 received
momentum. Here, light Wave A is now activated, and it can activate dark Wave or light Rook. It
cannot activate other opponent's pieces (here, dark Knight, Pegasus and dark Starchild). Light
Wave B can't be activated because it's on the same dark field as activating Wave.

\clearpage % ..........................................................

\subsubsection*{Activating Starchild}
\addcontentsline{toc}{subsubsection}{Activating Starchild}

\vspace*{-1.3\baselineskip}
\noindent
\begin{figure}[!h]
% \begin{figure}[!t]
\includegraphics[width=1.0\textwidth, keepaspectratio=true]{examples/22_o/scn_o_04_activating_starchild.png}
\caption{Activating Starchild}
\label{fig:scn_o_04_activating_starchild}
% \centering
\end{figure}

\hyperref[fig:10_wave]{Like Wave}, activated Starchild does not spend received momentum for moving;
if Starchild activates a piece, it transfers all of received momentum to it. In example similar to
previous, light Starchild is receiving 5 momentum after being activated by light Queen, received
momentum is then transfered from light Wave to dark Wave to dark Knight in its entirety.

\clearpage % ..........................................................

\subsubsection*{Neighboring-fields}
\addcontentsline{toc}{subsubsection}{Neighboring-fields}

% \vspace*{-0.9\baselineskip}
\noindent
\begin{wrapfigure}[5]{l}{0.4\textwidth}
\centering
\includegraphics[width=0.192307692\textwidth, keepaspectratio=true]{examples/22_o/scn_o_05_neighboring_fields.png}
\caption{Neighboring-fields}
\label{fig:scn_o_05_neighboring_fields}
\end{wrapfigure}
Neighboring-fields are all fields immediately surrounding a piece horizontally, vertically
and diagonally. They are the same as step-fields of a King.

% \clearpage % ..........................................................

\vspace*{2.1\baselineskip}
\subsubsection*{Moving a Star}
\addcontentsline{toc}{subsubsection}{Moving a Star}

% \vspace*{-0.9\baselineskip}
\noindent
\begin{wrapfigure}[5]{l}{0.4\textwidth}
\centering
\includegraphics[width=0.192307692\textwidth, keepaspectratio=true]{examples/22_o/scn_o_06_starchild_moving_star_init.png}
\caption{Moving into a Star}
\label{fig:scn_o_06_starchild_moving_star_init}
\end{wrapfigure}
Starchild can activate a Star the same way as any other piece, i.e. by capturing field at
which Star is located. Activated Star receives 1 momentum.

\vspace*{2.1\baselineskip}
\noindent
\begin{wrapfigure}[7]{l}{0.4\textwidth}
\centering
\includegraphics[width=0.192307692\textwidth, keepaspectratio=true]{examples/22_o/scn_o_07_starchild_moving_star_end.png}
\caption{Moving a Star}
\label{fig:scn_o_07_starchild_moving_star_end}
\end{wrapfigure}
Once activated, Star can move to any empty neighboring-field, which all are enumerated in
example on the left.

Note, even if activated Starchild received more than 1 momentum, Star can move for only one step.

\clearpage % ..........................................................

% \vspace*{1.1\baselineskip}
\subsubsection*{Not moving a Monolith}
\addcontentsline{toc}{subsubsection}{Not moving a Monolith}

% \vspace*{-0.9\baselineskip}
\noindent
\begin{wrapfigure}[4]{l}{0.4\textwidth}
\centering
\includegraphics[width=0.346153846\textwidth, keepaspectratio=true]{examples/22_o/scn_o_08_starchild_not_moving_monolith_init.png}
\caption{Moving into a Monolith}
\label{fig:scn_o_08_starchild_not_moving_monolith_init}
\end{wrapfigure}
Starchild can try to capture field at which Monolith is located, as if trying to activate and move
it, like a Star.

\vspace*{7.1\baselineskip}
\noindent
\begin{wrapfigure}[5]{l}{0.4\textwidth}
\centering
\includegraphics[width=0.346153846\textwidth, keepaspectratio=true]{examples/22_o/scn_o_09_starchild_not_moving_monolith_end.png}
\caption{Moving out of a Monolith}
\label{fig:scn_o_09_starchild_not_moving_monolith_end}
\end{wrapfigure}
Instead of moving a Monolith, Starchild then emerges on any empty portal-field around Monolith it
tried to move. If there is none, Starchild is oblationed.

\clearpage % ..........................................................

% \vspace*{1.1\baselineskip}
\subsubsection*{Not teleporting Wave}
\addcontentsline{toc}{subsubsection}{Not teleporting Wave}

% \vspace*{-0.9\baselineskip}
\noindent
\begin{wrapfigure}[6]{l}{0.4\textwidth}
\centering
\includegraphics[width=0.346153846\textwidth, keepaspectratio=true]{examples/22_o/scn_o_10_starchild_activated_wave_not_teleporting_init.png}
\caption{Moving into a Star}
\label{fig:scn_o_10_starchild_activated_wave_not_teleporting_init}
\end{wrapfigure}
Wave activated by Starchild cannot teleport. Similar to Starchild, Wave can try to teleport;
instead, it emerges on an empty portal-field around Monolith or a Star through which it tried
to teleport.

\vspace*{5.1\baselineskip}
\noindent
\begin{wrapfigure}[5]{l}{0.4\textwidth}
\centering
\includegraphics[width=0.346153846\textwidth, keepaspectratio=true]{examples/22_o/scn_o_11_starchild_activated_wave_not_teleporting_end.png}
\caption{Moving out of a Star}
\label{fig:scn_o_11_starchild_activated_wave_not_teleporting_end}
\end{wrapfigure}
Also similar to Starchild, if there is no empty portal-field around Monolith (or a Star), Wave
is oblationed.

\clearpage % ..........................................................

% \vspace*{1.1\baselineskip}
\subsubsection*{Teleporting Wave}
\addcontentsline{toc}{subsubsection}{Teleporting Wave}

\vspace*{-0.9\baselineskip}
\noindent
\begin{figure}[!h]
% \begin{figure}[!t]
\includegraphics[width=1.0\textwidth, keepaspectratio=true]{examples/22_o/scn_o_12_star_moved_wave_teleportation.png}
\caption{Optional Wave teleportation}
\label{fig:scn_o_12_star_moved_wave_teleportation}
% \centering
\end{figure}

\vspace*{-0.3\baselineskip}
Wave activated by pieces other than Starchild can still teleport as usual. Stars in this variant
can be moved out of their default positions. Teleportation for Wave reaching a Star is optional,
step-fields behind a Star are still accessible. Here, light Wave could also activate light Queen. So,
\hyperref[fig:scn_d_09_teleport_wave_via_monolith]{Monolith is the only piece Wave cannot "pass-through"},
i.e. ignore as all the other pieces on chessboard.

\clearpage % ..........................................................

\vspace*{-2.1\baselineskip}
\noindent
\begin{figure}[!h]
% \begin{figure}[!t]
\includegraphics[width=1.0\textwidth, keepaspectratio=true]{examples/22_o/scn_o_13_star_moved_wave_off_board.png}
\caption{Wave teleported off-board}
\label{fig:scn_o_13_star_moved_wave_off_board}
% \centering
\end{figure}

Wave can end up with all step-fields off-board after teleportation, due to one or both Stars
moved out of their initial positions. In such a case, Wave is oblationed, the same as in
\hyperref[fig:scn_d_11_wave_teleported_off_board]{previous variant, Discovery}.

Wave is also removed from chessboard if, after teleportation, all of its step-fields are
blocked; this is again similar to
\hyperref[fig:scn_d_10_teleported_wave_blocked]{previous variant, Discovery}.

\clearpage % ..........................................................

% \vspace*{1.1\baselineskip}
\subsubsection*{Conversion immunity}
\addcontentsline{toc}{subsubsection}{Conversion immunity}

\vspace*{-0.9\baselineskip}
\noindent
\begin{figure}[!h]
% \begin{figure}[!t]
\includegraphics[width=1.0\textwidth, keepaspectratio=true]{examples/22_o/scn_o_14_starchild_conversion_immunity_init.png}
\caption{Conversion immunity}
\label{fig:scn_o_14_starchild_conversion_immunity_init}
% \centering
\end{figure}

\hyperref[sec:Mayan Ascendancy/Pyramid/Conversion]{Conversion} is a move in which activated
Pyramid reaches opponent's piece, if it's not King, on own side of board. Pyramid is then
oblationed, and reached piece is replaced by the same piece in opposite color.
Starchild cannot be converted, instead, original Starchild remains on chessboard;
conversioning Pyramid is still oblationed.

% ------------------------------------------------------------ Movement
\clearpage % ..........................................................
% Trance-journey ------------------------------------------------------

\subsection*{Trance-journey}
\addcontentsline{toc}{subsection}{Trance-journey}

% \vspace*{-1.3\baselineskip}
\vspace*{-0.9\baselineskip}
\noindent
\begin{wrapfigure}[11]{l}{0.4\textwidth}
\centering
\includegraphics[width=0.346153846\textwidth, keepaspectratio=true]{examples/22_o/scn_o_15_trance_journey_init_starchild.png}
\caption{Starchild initiating}
\label{fig:scn_o_15_trance_journey_init_starchild}
\end{wrapfigure}
Trance-journey is initiated by either Shaman or Starchild, by activating another Starchild.
Activated Starchild then activates a piece, entranced piece then leaves onto trance-journey.
Any piece, own or opponent's, can be entranced, except Kings, Stars and Monoliths.

Between initiating piece and entrancing Starchild only Waves are allowed, but no other pieces.
Initiating Shaman or Starchild can themselves be activated by some other piece(s), not necessarily
in the same color.

% \vspace*{-0.3\baselineskip}
\vspace*{-0.1\baselineskip}
\noindent
\begin{wrapfigure}[10]{l}{0.4\textwidth}
\centering
\includegraphics[width=0.346153846\textwidth, keepaspectratio=true]{examples/22_o/scn_o_16_trance_journey_init_shaman.png}
\caption{Shaman initiating}
\label{fig:scn_o_16_trance_journey_init_shaman}
\end{wrapfigure}
Entranced piece must have the same color as initiating Shaman or Starchild, color of
entrancing Starchild do not need to match.

Here, entranced piece is dark Bishop, both initiating pieces are also dark, i.e. dark Shaman
in this example, and dark Starchild in previous example. Entrancing piece in both examples
is light Starchild.

Entranced piece can end its trance-journey on any empty step-field. If all are occupied, then it emerges
on any empty entrancing Starchild's neighboring-field. If there's none, then it emerges on empty initiating
Shaman or Starchild's neighboring-fields. If all are occupied, entranced piece is oblationed.

\clearpage % ..........................................................

\vspace*{-2.1\baselineskip}
\noindent
\begin{figure}[!h]
% \begin{figure}[!t]
\includegraphics[width=1.0\textwidth, keepaspectratio=true]{examples/22_o/scn_o_17_trance_journey_started_by_shaman.png}
\caption{Trance-journey}
\label{fig:scn_o_17_trance_journey_started_by_shaman}
% \centering
\end{figure}

Trance-journey pattern depends on color of entrancing Starchild. Here, light Starchild features
\hyperref[fig:scn_cot_14_light_shaman_trance_journey]{light Shaman's pattern}. Should entrancing
Starchild be dark, it would also produce
\hyperref[fig:scn_cot_16_dark_shaman_trance_journey]{dark Shaman's pattern}.

Trance-journey is optional, entranced piece could just move, with received momentum.
Here, dark Bishop is receiving 1 momentum, so it could move for 1 step.

\clearpage % ..........................................................

% \vspace*{-1.1\baselineskip}
\subsubsection*{Push-pull entrancement}
\addcontentsline{toc}{subsubsection}{Push-pull entrancement}

\vspace*{-0.9\baselineskip}
\noindent
\begin{wrapfigure}[7]{l}{0.4\textwidth}
\centering
\includegraphics[width=0.346153846\textwidth, keepaspectratio=true]{examples/22_o/scn_o_18_push_pull_trance_journey_init.png}
\caption{Initiating trance-journey}
\label{fig:scn_o_18_push_pull_trance_journey_init}
\end{wrapfigure}
Starchild initiating trance-journey could also be activated later in the same cascade, and act as an
entrancing Shaman. This is similar to push-pull entrancement in the
\hyperref[fig:star/scn_cot_33_push_pull_entrancement_start]{previous variant, Conquest of Tlalocan}.

\vspace*{3.9\baselineskip}
% \vspace*{-0.9\baselineskip}
\noindent
\begin{wrapfigure}[10]{l}{0.4\textwidth}
\centering
\includegraphics[width=0.346153846\textwidth, keepaspectratio=true]{examples/22_o/scn_o_19_push_pull_trance_journey_entrancing.png}
\caption{Push-pull entrancing}
\label{fig:scn_o_19_push_pull_trance_journey_entrancing}
\end{wrapfigure}
In previous eample, dark Starchild activated Wave A, which then activated Wave B. Here, Wave B is
"in the air", about to activate dark Starchild, which will then entrance dark Knight.

\clearpage % ..........................................................

% \vspace*{-3.1\baselineskip}
\noindent
\begin{figure}[!h]
% \begin{figure}[!t]
\includegraphics[width=1.0\textwidth, keepaspectratio=true]{examples/22_o/scn_o_20_push_pull_trance_journey_entranced.png}
\caption{Dark-pattern trance-journey}
\label{fig:scn_o_20_push_pull_trance_journey_entranced}
% \centering
\end{figure}

Starchild can initiate trance-journey by
\hyperref[sec:Terms/Push-pull activation]{push-pull activation},
if its color is the same as color of entranced piece; here both Starchild and Knight are dark.

Push-pull activation would work even if initiating Starchild has been activated by some other pieces,
which don't have to be in the same color.

\clearpage % ..........................................................

% \vspace*{-1.1\baselineskip}
\subsubsection*{Failed trance-journey}
\addcontentsline{toc}{subsubsection}{Failed trance-journey}

\vspace*{-1.1\baselineskip}
\noindent
\begin{figure}[!h]
% \begin{figure}[!t]
\includegraphics[width=1.0\textwidth, keepaspectratio=true]{examples/22_o/scn_o_21_trance_journey_failed.png}
\caption{Failed trance-journey}
\label{fig:scn_o_21_trance_journey_failed}
% \centering
\end{figure}

If all step-fields in a trance-journey are blocked by Kings, Stars or Monoliths, entranced piece is
\hyperref[sec:Terms/Oblation]{oblationed}, i.e. removed from chessboard as if captured by the opponent.

Note, dark Bishop is forced on taking trance-journey, because after activation it's blocked from performing
normal, diagonal move by own, initiating dark Starchild.

% ------------------------------------------------------ Trance-journey
\clearpage % ..........................................................

\subsection*{Syzygy}
\addcontentsline{toc}{subsection}{Syzygy}

\vspace*{-1.3\baselineskip}
\noindent
\begin{figure}[!h]
% \begin{figure}[!t]
\includegraphics[width=1.0\textwidth, keepaspectratio=true]{examples/22_o/scn_o_22_syzygy_monolith.png}
\caption{Demoting-to-Pawn syzygy}
\label{fig:scn_o_22_syzygy_monolith}
% \centering
\end{figure}

Starchild is celestial piece, it can participate in
\hyperref[fig:scn_d_15_syzygy_2_stars_init]{demoting-to-Pawn syzygy} in place of Stars and Monoliths.
\hyperref[fig:scn_d_17_syzygy_2_monoliths_init]{Again}, shortest step connecting Stars, Monoliths,
Starchilds is called syzygy-step, fields which are connected by syzygy-steps are called syzygy-fields.
For horizontal and vertical syzygy, syzygy-steps are the same as steps of Rook; for diagonal it’s
Bishop steps. Starchilds are also eligible to demotion.

\clearpage % ..........................................................

\vspace*{-2.1\baselineskip}
\noindent
\begin{figure}[!h]
% \begin{figure}[!t]
\includegraphics[width=1.0\textwidth, keepaspectratio=true]{examples/22_o/scn_o_23_syzygy_starchild_init.png}
\caption{Ressurection syzygy start}
\label{fig:scn_o_23_syzygy_starchild_init}
% \centering
\end{figure}

Starchild-initiated syzygy is ressurection. One captured piece, if it's not Starchild, can be (but
don't have to be) ressurected by replacing initiating Starchild, Starchild itself is then oblationed.
Only captured pieces can be ressurected. Kings, Stars and Monoliths cannot be ressurected.

\clearpage % ..........................................................

\vspace*{-2.1\baselineskip}
\noindent
\begin{figure}[!h]
% \begin{figure}[!t]
\includegraphics[width=1.0\textwidth, keepaspectratio=true]{examples/22_o/scn_o_24_syzygy_starchild_end.png}
\caption{Queen ressurected}
\label{fig:scn_o_24_syzygy_starchild_end}
% \centering
\end{figure}

Here, ressurected Queen replaced initiating Starchild. Note, in this variant
\hyperref[sec:One/Promotion]{promotion is monogamous}, so the only light Queen
had to be captured, before it could be ressurected.

\clearpage % ..........................................................

\vspace*{-2.1\baselineskip}
\noindent
\begin{figure}[!h]
% \begin{figure}[!t]
\includegraphics[width=1.0\textwidth, keepaspectratio=true]{examples/22_o/scn_o_25_syzygy_starchild_ressurection.png}
\caption{Starchild ressurected}
\label{fig:scn_o_25_syzygy_starchild_ressurection}
% \centering
\end{figure}

Waves and Starchilds can be ressurected, without initiating Starchild being oblationed. Chosen piece can emerge
on any empty neighboring-field around Starchilds in syzygy. If neighboring-fields are all occupied, piece emerges
on any empty portal-field around Stars, Monoliths in syzygy. If portal-fields are occupied, piece emerges on any
empty syzygy-field. If all are occupied, ressurection is not performed.

% *********************************************************** Starchild
\clearpage % ..........................................................

\hyperref[fig:scn_d_16_syzygy_2_stars_steps]{Similar to demoting-to-Pawn}, only one ressurection per syzygy is
allowed; to ressurect another piece Starchild in syzygy has to move out of alignment, and then back in.

\section*{Promotion}
\addcontentsline{toc}{section}{Promotion}
\label{sec:One/Promotion}

Promotion is non enforced, delayed variety, i.e. it's the same as in
\hyperref[sec:Age of Aquarius/Promotion]{previous chess variant}, Age of Aquarius.

Additionaly, promotion in this variant is monogamous.
Only one Queen in the same color can be present on chessboard at any given time.

\clearpage % ..........................................................

\section*{Rush, en passant}
\addcontentsline{toc}{section}{Rush, en passant}

\vspace*{-1.2\baselineskip}
\noindent
\begin{figure}[!h]
% \begin{figure}[!t]
\includegraphics[width=1.0\textwidth, keepaspectratio=true]{en_passants/22_one_en_passant.png}
\caption{En passant}
\label{fig:22_one_en_passant}
% \centering
\end{figure}

Rush and en passant are identical to those in \hyperref[fig:14_hemera_s_dawn_en_passant]{Hemera's Dawn variant}.
Own Pawns can be rushed for up to 11 fields in this variant.

\clearpage % ..........................................................

\section*{Castling}
\addcontentsline{toc}{section}{Castling}

Castling is the same as in Classical Chess, only difference is that King can move between 2 and 10 fields across.
All other constraints from Classical Chess still applies.

\noindent
\begin{figure}[!h]
% \begin{figure}[!t]
\includegraphics[width=1.0\textwidth, keepaspectratio=true]{castlings/22_o/one_castling.png}
\caption{Castling}
\label{fig:one_castling}
% \centering
\end{figure}

In example above, all valid King's castling moves are numbered.

\noindent
\begin{figure}[!h]
% \begin{figure}[!t]
\includegraphics[width=1.0\textwidth, keepaspectratio=true]{castlings/22_o/one_castling_right_04.png}
\caption{Castling short right}
\label{fig:one_castling_right_04}
% \centering
\end{figure}

In this example King was castling short to the right. Initial King's position is marked with "K".
After castling is finished, right Rook ends up at field immediately left to the King.

\clearpage % ..........................................................

\section*{Initial setup}
\addcontentsline{toc}{section}{Initial setup}

Compared to initial setup of Discovery, Starchild is inserted between Unicorn and Shaman
symmetrically, on both sides of chessboard. This can be seen in the image below:

\noindent
% \begin{figure}[t]
\begin{figure}[h]
\includegraphics[width=1.0\textwidth, keepaspectratio=true]{boards/22_one.png}
\caption{One board}
\label{fig:22_one}
% \centering
\end{figure}

\clearpage % ..........................................................
% ========================================================= One chapter
