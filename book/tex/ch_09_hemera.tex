
% Hemera's Dawn chapter ===============================================
\chapter*{Hemera's Dawn}
\addcontentsline{toc}{chapter}{Hemera's Dawn}

\begin{flushright}
\parbox{0.8\textwidth}{
\emph{Then assuredly the world was made, not in time, but simultaneously with time. \\
\hspace*{\fill}{\textperiodcentered \textperiodcentered \textperiodcentered \hspace*{0.2em} St. Augustine} } }
\end{flushright}

\noindent
Hemera's Dawn is chess variant which is played on 20 x 20 board, with
darkish red-brown and gray fields and pure red and bright green-tinted
yellow pieces. Star colors are bright blue and white. In algebraic
notation, columns are enumerated from 'a' to 't', and rows are enumerated
from '1' to '20'. A new piece is introduced, Centaur.

\clearpage % ..........................................................
% Centaur *************************************************************

\section*{Centaur}
\addcontentsline{toc}{section}{Centaur}

\noindent
\begin{wrapfigure}[11]{l}{0.4\textwidth}
\centering
\includegraphics[width=0.4\textwidth, keepaspectratio=true]{pieces/12_centaur.png}
\caption{Centaur}
\label{fig:12_centaur}
\end{wrapfigure}
Centaur is similar to Unicorn, only it can continue its' jumpy movement
in two chosen directions until another piece is encountered, or it runs
out of a chessboard.

Note that Centaur cannot change its’ heading after first two jumps, i.e.
once long and short jump directions are chosen.

In algebraic notation symbol for Centaur is ’C’.

% \vspace*{0.05\textheight}
\noindent
\begin{wrapfigure}{l}{0.4\textwidth}
\centering
\includegraphics[width=0.4\textwidth, keepaspectratio=true]{pieces/star/14_hemera_s_dawn.png}
\caption{Star}
\label{fig:star/14_hemera_s_dawn}
\end{wrapfigure}
Star colors in this variant are different to colors of light and dark pieces.

\clearpage % ..........................................................
% Movement ------------------------------------------------------------

% \vspace{7\baselineskip}
\subsection*{Movement}
\addcontentsline{toc}{subsection}{Movement}

\noindent
\begin{wrapfigure}{l}{0.4\textwidth}
\centering
\includegraphics[width=0.4\textwidth, keepaspectratio=true]{examples/14_hd/scn_hd_01_centaur_same_color.png}
\caption{Centaur short jump}
\label{fig:scn_hd_01_centaur_same_color}
\end{wrapfigure}
On fields with the same color as Centaur, it can move exactly the
same way Knight does.

\clearpage % ..........................................................

\noindent
\begin{figure}[!h]
% \begin{figure}[!t]
\includegraphics[width=1.0\textwidth, keepaspectratio=true]{examples/14_hd/scn_hd_02_centaur_opposite_color.png}
\caption{Centaur long jump}
\label{fig:scn_hd_02_centaur_opposite_color}
% \centering
\end{figure}

On fields in opposite color, Centaur can jump much longer, exactly the
same way Unicorn does. Again, just as Knight (and Unicorn), Centaur is
not hampered by surrouding pieces. Only own pieces on marked (i.e. step)
fields can prevent Centaur to move, opponent's pieces could be captured.

For comparison, Knight's step-fields are also numbered (grey).

\clearpage % ..........................................................

\noindent
\begin{figure}[!h]
% \begin{figure}[!t]
\includegraphics[width=1.0\textwidth, keepaspectratio=true]{examples/14_hd/scn_hd_03_centaur_multi_step.png}
\caption{Centaur multi-step move}
\label{fig:scn_hd_03_centaur_multi_step}
% \centering
\end{figure}

On first two jumps, Centaur can choose freely any direction (blue). After
two directions are chosen, Centaur for the rest of a move has to follow
them (green), e.g. after reaching field 6, it cannot move to any 7a to 7g
fields (grey).

Here, step-fields are numbered 1 to 11, they are also capture-fields,
Centaur could capture dark Bishop if in place of the light one. Pieces on
all other fields are ignored (Pawns).

% ------------------------------------------------------------ Movement
% ************************************************************* Centaur
\clearpage % ..........................................................

\section*{Promotion}
\addcontentsline{toc}{section}{Promotion}

Promotion is non enforced, delayed variety, i.e. it's the same as in
\hyperref[sec:Age of Aquarius/Promotion]{previous chess variant}, Age of Aquarius.

Promotion in this variant is polygamous, more than one Queen in the same color
can be present on chessboard at any given time.

\clearpage % ..........................................................

\section*{En passant}
\addcontentsline{toc}{section}{En passant}

\noindent
\begin{wrapfigure}{l}{0.4\textwidth}
\centering
\includegraphics[width=0.15\textwidth, keepaspectratio=true]{en_passants/14_hemera_s_dawn_en_passant.png}
\caption{En passant}
\label{fig:14_hemera_s_dawn_en_passant}
\end{wrapfigure}
Rush and en passant are identical to those in Classic Chess, only difference
is that Pawn can now move longer on initial turn, up to 8 fields in this
variant.

\clearpage % ..........................................................

\section*{Castling}
\addcontentsline{toc}{section}{Castling}

Castling is the same as in Classical Chess, only difference is that King can move between 2 and 7 fields across.
All other constraints from Classical Chess still applies.

\noindent
\begin{figure}[!h]
% \begin{figure}[!t]
\includegraphics[width=1.0\textwidth, keepaspectratio=true]{castlings/14_hd/hemera_s_dawn_castling.png}
\caption{Castling}
\label{fig:hemera_s_dawn_castling}
% \centering
\end{figure}

In example above, all valid King's castling moves are numbered.

\noindent
\begin{figure}[!h]
% \begin{figure}[!t]
\includegraphics[width=1.0\textwidth, keepaspectratio=true]{castlings/14_hd/hemera_s_dawn_castling_right_03.png}
\caption{Castling short right}
\label{fig:hemera_s_dawn_castling_right_03}
% \centering
\end{figure}

In this example King was castling short to the right. Initial King's position is marked with "K".
After castling is finished, right Rook ends up at field immediately left to the King.

\clearpage % ..........................................................

\section*{Initial setup}
\addcontentsline{toc}{section}{Initial setup}

Initial setup can be seen in image below:

\noindent
% \begin{figure}[t]
\begin{figure}[h]
\includegraphics[width=1.0\textwidth, keepaspectratio=true]{boards/14_hemera_s_dawn.png}
\caption{Hemera's Dawn board}
\label{fig:14_hemera_s_dawn}
% \centering
\end{figure}

\clearpage % ..........................................................
% =============================================== Hemera's Dawn chapter
