
% Introduction chapter ------------------------------------------------
\chapter*{Introduction}
\addcontentsline{toc}{chapter}{Introduction}

\begin{flushright}
\parbox{0.6\textwidth}{
\emph{Life's too short for chess. \\
\hspace*{\fill}{\textperiodcentered \textperiodcentered \textperiodcentered \hspace*{0.2em} Henry James Byron} } }
\end{flushright}

\noindent
I was in my aunt's house, on the border of a small village.
Through window, walled garden was visible just behind the house.
Behind the garden, a tiny brook. And hills behind the brook.
Afternoon Sun was casting its' orange rays into warm room. It
was frosty outside.

My cousin approached me with some nifty gizmo. He was a
few years older than me and was already going to school.

\noindent
"Here, look at what I got." \\
\hspace*{\fill}"What's that?" \\
"Chess set. Wanna try? Lemme show you." \\
\hspace*{\fill}"Sure."

It was small, plasticky, fiddly thing designed to fit into winter's
coat pocket, to be used on the go. Folding board was also used to
hold all pieces in it. Each piece was as small as humanely usable.
Each field had a hole in the middle. At the bottom of each piece
there was small rod fitting into those holes. It was colored all
in red and ivory.

Short lesson revealed it's not that difficult to grasp what's going
on. Within minutes I picked it up. First match was, predictably, a
complete disaster. On the second go my cousin forgot about a piece,
and I grabbed his Queen gleefully. He surrendered.

After he left me with a new widget, I was intrigued. I wasn't
about playing the game, though. I was more into redesign it. Could it
be made better, more challenging, or just different?

\noindent
'Why not make Knight jump longer, say 3 by 1 fields?' \\
'Hmmmm...' \\
'Nah, that would make jump too long for such a small board.'

Outside, the setting Sun was shining red.

\clearpage % ..........................................................
% ------------------------------------------------ Introduction chapter
