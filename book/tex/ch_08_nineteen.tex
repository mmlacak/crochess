
% Copyright (c) 2015 - 2020 Mario Mlačak, mmlacak@gmail.com
% Licensed and published as Public Domain work.

% Nineteen chapter ====================================================
\chapter*{Nineteen}
\addcontentsline{toc}{chapter}{Nineteen}

\begin{flushright}
\parbox{0.8\textwidth}{
\emph{The truth is at the beginning of anything and its end are alike touching. \\
\hspace*{\fill}{\textperiodcentered \textperiodcentered \textperiodcentered \hspace*{0.2em} Yoshida Kenko} } }
\end{flushright}

\noindent
Nineteen is chess variant which is played on 18 x 18 board, with
light gold-yellow and white fields and gold-yellow and dark gray
pieces. In algebraic notation, columns are enumerated from 'a' to 'r',
and rows are enumerated from '1' to '18'. A new piece is introduced,
Star.

\clearpage % ..........................................................
% Star ****************************************************************

\section*{Star}
\addcontentsline{toc}{section}{Star}

% \vspace*{-1.3\baselineskip}
\noindent
\begin{wrapfigure}[11]{l}{0.4\textwidth}
\centering
\includegraphics[width=0.4\textwidth, keepaspectratio=true]{pieces/11_star.png}
\caption{Star}
\label{fig:11_star}
\end{wrapfigure}
Star does not belong to any player, and cannot be moved, activated, captured or
converted. Light Stars are positioned in lower left and upper right corners, dark
Stars in lower right and upper left corners.

Star is a teleporting piece. Teleportation is initiated by touching a field (or a
Star) at which it stands with a piece, using either normal or capturing step. Piece
in question, if it's not Wave, then reappears on any empty portal-field near Star in
opposite color. Any momentum carried is lost. Teleportation is not limited by matching
colors of a piece and a Star, any piece can use any Star to start teleporting.

Player initiating teleportation can choose which opposite color Star will be
destination, and at which empty portal-field piece will reappear. If there is no empty
portal-field near both Stars of opposite color piece is oblationed, i.e. removed from
chessboard as if it has been captured.

If teleported piece is Wave, it continues movement from a field occupied by the other
Star in the same color. Wave retains all of momentum carried into teleportation. The
way and direction of movement of Wave is the same as before teleportation.

Kings cannot be teleported. Pawns cannot be promoted to a Star. In algebraic notation
symbol for Star is 'T'.

\clearpage % ..........................................................

\subsection*{Portal-fields}
\addcontentsline{toc}{subsection}{Portal-fields}

% \vspace{-1.1\baselineskip}
\noindent
\begin{figure}[!h]
\includegraphics[width=1.0\textwidth, keepaspectratio=true]{examples/12_n/scn_n_01_portal_fields.png}
\caption{Portal-fields}
\label{fig:scn_n_01_portal_fields}
\end{figure}

Portal-fields are all fields immediately surrounding a particular field
horizontally, vertically and diagonally. They are the same as step-fields
of a King.

Since all Stars are pinned into the corners of a chessboard, there are always
exactly 3 portal-fields around each one.

\clearpage % ..........................................................
% Teleporting pieces ==================================================

\subsection*{Teleporting pieces}
\addcontentsline{toc}{subsection}{Teleporting pieces}

\noindent
\begin{figure}[!h]
\includegraphics[width=1.0\textwidth, keepaspectratio=true]{examples/12_n/scn_n_02_teleport_init.png}
\caption{Teleportation start}
\label{fig:scn_n_02_teleport_init}
\end{figure}

Light Bishop is about to teleport by diving into dark Star. Normaly, teleport
destination could be chosen among empty portal-fields of both light Stars.
Here, portal-field are all blocked, except field 1.

\clearpage % ..........................................................

\noindent
\begin{figure}[!h]
\includegraphics[width=1.0\textwidth, keepaspectratio=true]{examples/12_n/scn_n_03_teleport_move_2.png}
\caption{Teleporting dark Rook}
\label{fig:scn_n_03_teleport_move_2}
\end{figure}

Dark Rook after teleportation is about to be removed from chessboard, since there
is no empty portal-fields around both opposite-color Stars. Note that own, dark
Wave (on field 1) is also blocking emergance of teleported dark Rook, even though
it could be activated by it in a normal, non-teleporting ply.

\clearpage % ..........................................................

\noindent
\begin{figure}[!h]
\includegraphics[width=1.0\textwidth, keepaspectratio=true]{examples/12_n/scn_n_04_teleport_move_3.png}
\caption{Teleporting light Wave}
\label{fig:scn_n_04_teleport_move_3}
\end{figure}

Wave can reach a Star and start teleporting even if activating piece (here,
Pegasus) would be blocked.

\clearpage % ..........................................................

\noindent
\begin{figure}[!h]
\includegraphics[width=1.0\textwidth, keepaspectratio=true]{examples/12_n/scn_n_05_teleport_end.png}
\caption{Teleportation end}
\label{fig:scn_n_05_teleport_end}
\end{figure}

Teleported Wave emerges from the other Star in the same color as the starting one.
Wave has to continue movement in the same direction as it did before teleportation,
direction cannot be changed. Wave also retains momentum it had before teleportation,
so here it can activate Pyramid, or
\hyperref[fig:scn_mv_28_activating_rush_pawn_init]{rush light Pawn for 2 fields}.

\clearpage % ..........................................................

\noindent
\begin{figure}[!h]
\includegraphics[width=1.0\textwidth, keepaspectratio=true]{examples/12_n/scn_n_06_teleport_wave_blocked.png}
\caption{Teleported Wave blocked}
\label{fig:scn_n_06_teleport_wave_blocked}
\end{figure}

If teleported Wave has all of its step-fields blocked (here, by dark Pawns), it is
removed from chessboard, just like any other
\hyperref[fig:scn_n_03_teleport_move_2]{teleported piece which has all portal-fields blocked}.

\clearpage % ..........................................................
% Teleporting Wave ----------------------------------------------------

\subsubsection*{Teleporting Wave}
\addcontentsline{toc}{subsubsection}{Teleporting Wave}

\vspace*{-1.2\baselineskip}
\noindent
\begin{figure}[!h]
\includegraphics[width=1.0\textwidth, keepaspectratio=true]{examples/12_n/scn_n_07_teleport_wave_init.png}
\caption{Wave out-of-board before teleportation}
\label{fig:scn_n_07_teleport_wave_init}
\end{figure}

Here, light grey fields are virtual fields extending existing chessboard.
\hyperref[fig:scn_mv_20_wave_activation_by_unicorn_first_step]{Wave activated by Unicorn}
has to choose 2 different steps at the beginning of its movement, and follow
them for the remainder of a ply. Wave's movement is legal as long as its
\hyperref[fig:scn_mv_23_wave_off_board]{ply ends on a chessboard}. So, light
Wave can reach light Star and start teleporting, even though it stepped
outside of a board.

\clearpage % ..........................................................

\noindent
\begin{figure}[!h]
\includegraphics[width=1.0\textwidth, keepaspectratio=true]{examples/12_n/scn_n_08_teleport_wave_end.png}
\caption{Wave teleported}
\label{fig:scn_n_08_teleport_wave_end}
\end{figure}

Teleported Wave has to continue its movement performing the same step(s) as
before teleportation. That means, teleported Wave has to continue alternating
between 2 initially chosen steps, according to a color of a current field. So,
emerging step (here, long jump) is different from a step starting teleportation
(short jump).

\clearpage % ..........................................................

\noindent
\begin{figure}[!h]
\includegraphics[width=1.0\textwidth, keepaspectratio=true]{examples/12_n/scn_n_09_teleport_wave_2_init.png}
\caption{Wave before teleportation}
\label{fig:scn_n_09_teleport_wave_2_init}
\end{figure}

Similar example as previous, with dark Wave which has the same steps (short,
long jump) over the same colored fields (dark, light fields) switched. So,
teleporting step is also different (here, long jump) from previous example
(short jump).

\clearpage % ..........................................................

\noindent
\begin{figure}[!h]
\includegraphics[width=1.0\textwidth, keepaspectratio=true]{examples/12_n/scn_n_10_teleport_wave_2_end.png}
\caption{Wave out-of-board after teleportation}
\label{fig:scn_n_10_teleport_wave_2_end}
\end{figure}

\hyperref[fig:scn_n_08_teleport_wave_end]{Again},
teleported Wave has to continue alternating between 2 initially chosen steps,
according to a color of a current field, i.e. color of starting field of each
step. Wave's movement is legal as long as its
\hyperref[fig:scn_mv_23_wave_off_board]{ply ends on a chessboard}. So, dark
Wave can e.g. activate dark Pawn (with 1 momentum carried through teleportation),
even though it stepped outside of a board.

% ---------------------------------------------------- Teleporting Wave
\clearpage % ..........................................................
% Teleporting Pawn ----------------------------------------------------

\subsubsection*{Teleporting Pawn}
\addcontentsline{toc}{subsubsection}{Teleporting Pawn}

\vspace*{-1.4\baselineskip}
\noindent
\begin{figure}[!h]
\includegraphics[width=1.0\textwidth, keepaspectratio=true]{examples/12_n/scn_n_11_teleport_pawns_init.png}
\caption{Pawn teleporting on step-field}
\label{fig:scn_n_11_teleport_pawns_init}
\end{figure}

All pieces can access a Star on own step- or capture-field. So, light Pawn in
the same column as dark Star (here, a) can step into it, and teleport away. If
destination Star is on
\hyperref[sec:Definitions/Sides of a chessboard]{opponent's side of a board},
teleported Pawn is tagged for
promotion (green, blue fields). If destination portal-field is on opponent's
\hyperref[sec:Terms/Figure row]{figure row} (blue), player can choose between
promoting Pawn outright, or keeping it tagged for promotion.

\clearpage % ..........................................................

\noindent
\begin{figure}[!h]
\includegraphics[width=1.0\textwidth, keepaspectratio=true]{examples/12_n/scn_n_12_teleport_pawns_step_1.png}
\caption{Pawn teleporting on capture-field}
\label{fig:scn_n_12_teleport_pawns_step_1}
\end{figure}

Pawn can also dive into a Star located at its capture-field, and teleport away.
If destination Star is on
\hyperref[sec:Definitions/Sides of a chessboard]{own side of a board} (portal-fields
4, 5, 6), teleported Pawn loses options to promote, and does not gain opportunity
to rush on an initial move.

\clearpage % ..........................................................

\noindent
\begin{figure}[!h]
\includegraphics[width=1.0\textwidth, keepaspectratio=true]{examples/12_n/scn_n_13_teleport_pawns_end.png}
\caption{Pawn teleporting end}
\label{fig:scn_n_13_teleport_pawns_end}
\end{figure}

Light Pawn teleported onto own side of chessboard cannot rush, even though
destination field is on own \hyperref[sec:Terms/Pawn row]{Pawn row}.

% ---------------------------------------------------- Teleporting Pawn
\clearpage % ..........................................................

\subsubsection*{Teleporting Bishop}
\addcontentsline{toc}{subsubsection}{Teleporting Bishop}

\vspace*{-1.1\baselineskip}
\noindent
\begin{figure}[!h]
\includegraphics[width=1.0\textwidth, keepaspectratio=true]{examples/12_n/scn_n_14_teleport_bishop.png}
\caption{Bishop teleportation}
\label{fig:scn_n_14_teleport_bishop}
\end{figure}

Teleporting Bishop, like any other piece, can choose any empty portal-field
around opposite-color Star as a destination, regardless of a field's color.
Teleporting to a field in a different color changes (color of) accessible
fields for teleported Bishop, for the remainder of a game. Here, such
color-changing portal-fields are enumerated, 1 and 2.

% ================================================== Teleporting pieces
% **************************************************************** Star
\clearpage % ..........................................................

\section*{Pawn ranks, rows}
\addcontentsline{toc}{section}{Pawn ranks, rows}

\vspace*{-1.1\baselineskip}
\noindent
\begin{figure}[!h]
\includegraphics[width=1.0\textwidth, keepaspectratio=true]{examples/12_n/scn_n_15_pawn_ranks.png}
\caption{Pawn rows}
\label{fig:scn_n_15_pawn_ranks}
\end{figure}

In this variant, an additional rank of light (blue arrow) and dark (red) Pawns has
been added to \hyperref[fig:12_nineteen]{initial setup}. Ranks of Pawns are enumerated
starting with one closest to opponent; the closest rank being the first one (blue,
red arrows), while the standard rank of Pawns is the second rank (green, grey).

\clearpage % ..........................................................

\section*{Rush, en passant}
\addcontentsline{toc}{section}{Rush, en passant}

\noindent
\begin{wrapfigure}[14]{l}{0.4\textwidth} % 14 bch % 12 lmc
\centering
\includegraphics[width=0.388888889\textwidth, keepaspectratio=true]{en_passants/12_nineteen_en_passant.png}
\caption{En passant}
\label{fig:12_nineteen_en_passant}
\end{wrapfigure}
Rush and en passant are very similar to those in Classic Chess.

Pawns from both ranks can be rushed, up to the other end of
\hyperref[sec:Definitions/Sides of a chessboard]{own side of the chessboard}.

In this variant, Pawns in the first row (Pawn A) can be rushed for up to 6 fields,
while those in second row (B) can go up to 7 fields forward.

Converted opponent's Pawns cannot be rushed, even if converted on an initial positions
of own Pawns.

% \clearpage % ..........................................................

\section*{Promotion}
\addcontentsline{toc}{section}{Promotion}

Promotion is non enforced, delayed variety, i.e. it's the same as in
\hyperref[sec:Age of Aquarius/Promotion]{previous chess variant}, Age of Aquarius.

Again, Pawns cannot be promoted to a Star.

Additionaly, promotion in this variant is monogamous.
Only one Queen in the same color can be present on chessboard at any given time.

\clearpage % ..........................................................

\section*{Castling}
\addcontentsline{toc}{section}{Castling}

Castling is the same as in Classical Chess, only difference is that King can move between 2 and 6 fields across.
All other constraints from Classical Chess still applies.

\noindent
\begin{figure}[!h]
\includegraphics[width=1.0\textwidth, keepaspectratio=true]{castlings/12_n/nineteen_castling.png}
\caption{Castling}
\label{fig:nineteen_castling}
\end{figure}

In example above, all valid King's castling moves are numbered.

\noindent
\begin{figure}[!h]
\includegraphics[width=1.0\textwidth, keepaspectratio=true]{castlings/12_n/nineteen_castling_left_05.png}
\caption{Castling long left}
\label{fig:nineteen_castling_left_05}
\end{figure}

In this example King was castling long to the left. Initial King's position is marked with "K".
After castling is finished, left Rook ends up at field immediately right to the King.

Converted opponent's Rooks cannot be castled, even if converted on an initial positions
of own Rooks.

\clearpage % ..........................................................

\section*{Initial setup}
\addcontentsline{toc}{section}{Initial setup}

Stars are positioned in very corners of chessboard, light Stars in lower left and upper right
corners, dark Stars in lower right and upper left corners. Additional rank of light and dark
Pawns has been added. All other figures are also repositioned.

\noindent
\begin{figure}[h]
\includegraphics[width=1.0\textwidth, keepaspectratio=true]{boards/12_nineteen.png}
\caption{Nineteen board}
\label{fig:12_nineteen}
\end{figure}

\clearpage % ..........................................................
% ==================================================== Nineteen chapter
