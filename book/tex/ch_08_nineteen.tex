
% Nineteen chapter ----------------------------------------------------
\chapter*{Nineteen}
\addcontentsline{toc}{chapter}{Nineteen}

\begin{flushright}
\parbox{0.8\textwidth}{
\emph{The truth is at the beginning of anything and its end are alike touching. \\
\hspace*{\fill}{\textperiodcentered \textperiodcentered \textperiodcentered \hspace*{0.2em} Yoshida Kenko} } }
\end{flushright}

\noindent
Nineteen is chess variant which is played on 18 x 18 board, with
light gold-yellow and white fields and gold-yellow and dark gray
pieces. In algebraic notation, columns are enumerated from 'a' to 'r',
and rows are enumerated from '1' to '18'. A new piece is introduced,
Star.

\clearpage % ..........................................................

\section*{Star}
\addcontentsline{toc}{section}{Star}

\noindent
\begin{wrapfigure}{l}{0.4\textwidth}
\includegraphics[width=0.4\textwidth, keepaspectratio=true]{pieces/11_star.png}
\caption{Star}
\label{fig:star}
% % \centering
\end{wrapfigure}

Pawn cannot be promoted to Star.

\clearpage % ..........................................................

\section*{Initial setup}
\addcontentsline{toc}{section}{Initial setup}

Initial setup for Light player is (mirrored for Dark one):
\texttt{PPPPPPPPPPPPPPPPPP \\
        TRGAUWNBQKBNWUAGRT}, \\
or more conveniently, as seen in this image:

\noindent
% \begin{figure}[t]
\begin{figure}[h]
\includegraphics[width=1.0\textwidth, keepaspectratio=true]{boards/12_nineteen.png}
\caption{Nineteen board}
\label{fig:nineteen}
% \centering
\end{figure}

\clearpage % ..........................................................
% ---------------------------------------------------- Nineteen chapter
